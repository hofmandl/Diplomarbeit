
\chapter{Grundlagen}

\section{Stochastische Grundlagen}
\begin{mySatz} \label{StochDgl}
Die stochastische Differentialgleichung
\begin{align*}
 dX(t) &= X(t)(A(t) dt + S(t)dW(t)) \\
  X(t_0) &= x_0
\end{align*}
besitzt die eindeutige L�sung

\begin{equation}
 X(t) = \exp \left( 
                             \int_{t_0}^{t} (A(u)-\frac{1}{2}S(u)^2)du + \int_{t_0}^{t}S(u)dW(u)
                       \right) 
\end{equation}

\end{mySatz}


\section{Finanzmarktmodell}
Im folgenden geben wir die Definition unseres Finanzmarktmodells. Fast alle S�tze werden mit dem rituellen Satz ``\standardsatz'' eingeleitet.
\begin{myDefinition} \label{Standardfinanzmarktmodell}
Sei $(\Omega,F,P) $ ein Wahrscheinlichkeitsraum. Sei $W$ eine auf diesen Raum definierte Brownsche
Bewegung. Seien $r$, $\mu$, $\sigma$ reelle Zahlen und sei $r < \mu$ und $\sigma>0$. Der Bond $B$ und die
Aktie $S$ seien die eindeutigen L�sungen der stochastischen Differentialgleichungen
\begin{align}
 dB(t) &= r B(t) dt \\
  B(0) &= 1         \\
 dS(t) &= S(t)(\mu dt + \sigma dW(t)) \\
  S(0) &= s_0
\end{align}
Dabei nennen wir $r$ den Zinssatz, $\mu$ die Aktienrendite und $\sigma$ die Volatilit�t. Weiterhin sei eine Konstante $c>0$ gegeben, die wir Handelscharnke nennen. Gelten alle
diese Vorrausetzungen, so sagen wir dass die Vorrausetzungen unseres Standardfinanzmarktmodells gelten.
\end{myDefinition}
Die Interpretation unseres Standardfinanzmarktmodells ist allein durch die Namensgebung schon ziemlich klar. Die Bedeutung der Handelschranke wird in der folgenden Definition klar
werden.

\begin{myDefinition} \label{def:Handelsstrategie}
\standardsatz
Die Menge $\mathcal{A}(t,x)$ aller progressiv messbaren Prozesse, f�r die die stochastische Differentialgleichung
\begin{align*}
 dX(t) &= X(t) \left[ \left( r + \pi(t)(\mu -r) \right) dt + \pi(t) \sigma dW(t) \right] \\
  X(t) &= x
\end{align*}
eine eindeutige L�sung besitzt, f�r die $P$- fast sicher
\begin{equation}
 \int_0^T  \left( X(t) \pi(t) \right)^2 dt < \infty
\end{equation}
und
\begin{equation}
 X(t) \geq 0
\end{equation}
f�r alle $t \in [0,T]$ gilt, hei�t die Menge der zul�ssige Handelsstrategien.
\end{myDefinition}

\begin{myDefinition} \label{Handelsstrategie}
\standardsatz
Die Menge $\mathcal{M}$ aller Funktionen von $\txbereich$ nach $\left[-c,c \right]$ hei�t die Menge der zul�ssigen Markov-Handelstrategien, wenn
gilt:
F�r alle $(t,x)$ und alle $p \in \mathcal{M}$ besitzt die die stochastische Differentialgleichung
\begin{align*}
 dX(t) &= X(t) \left[ \left( r + p(t,X(t))(\mu -r) \right) dt + p(t,X(t)) \sigma dW(t) \right] \\
  X(t) &= x
\end{align*}
eine eindeutige L�sung und mit dieser L�sung gilt $p(t,X(t)) \in \mathcal{A}(t,x)$.
\end{myDefinition}
In dem gerade gegebenen Sinne k�nnen wir also eine Markovstrategie auch immer als eine normale Handelstrategie auffassen und wir werden in dem Fall nicht mehr zwischen diesen
beiden Begriffen unterscheiden.
\begin{myDefinition} \label{Verm�gensgleichung}
Es gelten die Vorrausetzungen unseres Standardfinanzmarktmodells. Die L�sung von
\begin{align} \label{eq:Verm�gensgleichung}
 dX(t) &= X(t) \left[ \left( r + \pi(t)(\mu -r) \right) dt + \pi(t) \sigma dW(t) \right] \\
  X(t_0) &= x_0
\end{align}
bezeichen wir mit $X_{\pi,t_0,x}$. Ist $t_0=0$, so schreiben wir auch $X_{\pi,x}$.
\end{myDefinition}
Wie in der vorhegenden Bemerkung schon bemerkt, lassen wir es auch zu, dass statt $\pi$ eine Markovstrategie $p$ eingesetzt wird. Streng genommen ist $X$ dann eine L�sung der
Differentialgleichung in der Definition der Markovstrategie.

\begin{myDefinition} \label{def:f}
Es gelten die Vorrausetzungen unseres Standardfinanzmarktmodells. Sei $f$ durch 
\begin{equation} \label{eq:fvonpi}
 f(\pi) =  \pi (\mu-r)-\einhalb \pi^2 \sigma^2
\end{equation}
definiert.
\end{myDefinition}

% Satz und definition
\begin{mySatz} \label{th:Eigenschaften_f}
$f$ besitzt die beiden Nullstellen
\begin{align}
 x_0 &= 0 \\
 x_1 &= 2 \frac{\pi-\mu}{\sigma^2},
\end{align}
nimmt ihr Maximum bei
\begin{equation} \label{eq:pimax}
 \pi_{max} =  \pistern
\end{equation}
an hat dort den Wert
\begin{equation} \label{eq:pimax}
 f(\pi_{max}) =  \einhalb \frac{(\pi-\mu)^2}{\sigma^2}
\end{equation}
\end{mySatz}


\begin{mySatz} \label{th:expliziteL�sungVerm�gensgleichung}
Die explizite L�sung von \ref{eq:Verm�gensgleichung} ist durch
\begin{align*} \label{eq:L�sVerm�gensgleichung}
                  & X_{\pi,t_0,x}(t) \\
  &= x \exp \left(
                                   \int_{t_0}^{t} r + \pi(u)(\mu -r) - \frac{1}{2} \pi(u)^2 \sigma^2 du 
                                  +\int_{t_0}^{t}\pi(u)\sigma dW(u)) 
                            \right) \\
                  &= x \exp \left(
                                   \int_{t_0}^{t} r + f(\pi(u)) du 
                                  +\int_{t_0}^{t}\pi(u)\sigma dW(u)) 
                            \right)  \\
                 &= x \exp \left(
                                   \int_{t_0}^{t} r + \pi(u)(\mu -r) du
                           \right) 
                     \exp \left(      
                                \int_{t_0}^{t}\pi(u)\sigma dW(u)) - \frac{1}{2}  \int_{t_0}^{t} \pi(u)^2 \sigma^2 du
                            \right) \\
\end{align*}
gegeben. In der letzten Gleichung ist dabei der letzte Faktor ein Martingal.

\end{mySatz}





% F�r einen konstanten Verm�gensprozess $\pi$ vereinfacht diese Gleichung sich zu:
% \begin{align} \label{eq:aaa}
%  X_{\pi,t_0,x}(t) &= x \exp \left(
%                                     (t-t_0) ( r + \pi (\mu -r) - \frac{1}{2} \pi^2 \sigma^2 )
%                                    + \pi \sigma (W(t) - W(t_0)) 
%                              \right) \\
%                   &= x \exp \left(
%                                     (t-t_0)f(\pi)
%                                    + \pi \sigma (W(t) - W(t_0))
%                              \right) 
% \end{align}

% Dessen erwartungswert berechnet sich zu 
% \begin{align} 
%  E(X(t)) &= x exp( r + \pi (\mu -r)) exp(\pi \sigma W(t) -\frac{1}{2} \pi^2\sigma^2) \\
%        &= x exp( r + \pi (\mu -r))
% \end{align}

\begin{mySatz} \label{th:ExpLogUtility}
Es gilt
 \begin{equation} \label{eq:ExpLogUtility}
E \left[ \log(X_{\pi,t,x}(T)) \right] = \log(x)  + E \left[ \int_{t}^{T} (r + \pi(u)(\mu -r)-\frac{1}{2}\pi(u)^2\sigma^2)du  \right]
                                      = \log(x)  + E \left[ \int_{t}^{T} f(\pi(u)) du  \right]
\end{equation}
\end{mySatz}

% Logaritmiert man \ref{eq:L�sVerm�gensgleichung} und nimmt den Erwartungswert so erh�lt man
% \begin{equation} \label{eq:ExpLogUtility}
% E(log(X_{\pi,x}(T))) = log(x)  + \int_{0}^{T} (r + \pi(u)(\mu -r)-\frac{1}{2}\pi(u)^2\sigma^2)du 
% \end{equation}
% und bezeichnet dies als erwarteten logaritmischen Nutzen des Endverm�gens. Unter der Annahme einer konstanten
% Handelsstrategie vereinfacht sich \ref{eq:ExpLogUtility} zu
% 
% \begin{equation} \label{eq:ExpLogUtilityKonst}
% E(log(X_{\pi,x}(T))) = log(x)  + (r + \pi(\mu -r)-\frac{1}{2}\pi^2 \sigma^2) T. 
% \end{equation}