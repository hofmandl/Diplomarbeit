\documentclass[pdftex,a4paper]{scrartcl}

% \usepackage{ngerman}
\usepackage[latin1]{inputenc}
\usepackage[T1]{fontenc}

\usepackage{amsmath, amsthm, amssymb}
\usepackage{mathtools}

\title{Portfoliooptimierung}
\author{Martin B�schen}

\begin{document}

\maketitle

\tableofcontents

\section{Einleitung}

Sei $X$ der Verm�gensprozess zur Handelsstrategie $\pi$ im Black-Scholes Modell. Sei $U$ eine Nutzenfunktion. Dann nennen wir
\begin{equation}
  \text{max}_{\pi} E(U(X^{\pi}(T)))
\end{equation}
das \emph{klassische Portfoliooptimierungsproblem}. Sei $\tau$ eine Stoppzeit, zu der der Markt crasht. Dann nennen wir 
\begin{equation}
  \text{max}_{\pi} \text{min}_{\tau} E(U(X^{\pi\tau}(T)))
\end{equation}
das \emph{Worst-Case Portfoliooptimierungsproblem}. In dieser Arbeit m�chte ich das klassische Problem und dessen 
L�sung mittels stochastischer Kontrolltehorie diskutieren. Danach betrachte ich die Erweiterung von Korn auf das 
Worst-Case-Problem, dass zun�chst mit ganz anderen Methoden als den HJB-Gleichungen gel�st wurde. Schliesslich
betrachten wir dann die L�sung mittels HJB-Gleichungen. Dabei m�chte ich die Ergebnisse der verschiedenen
Ans�tze vergleichen, welche Vorteile die jeweiligen Herangehensweisen bieten und insbesondere auch die Unterschiede in den
Beweisen vergleichen.

Als laufendes Beispiel soll zu jeder der drei Aussagen die logaritmische Nutzenfunktion diskutiert werden.

Schliesslich m�chte ich durch einige Zufallssimulationen die Ergebnisse nochmal veranschaulichen.

Weitere Aspekte die diskutiert werden k�nnten:

\begin{itemize}
 \item Further aspects in Korn, 2007, seite 16
 \item Further possible refinements, Korn/willmott, seite 183
 \item Durchgehen des Verifikationssatzes mit logaritmischen nutzen, Korn, 2007, seite 9
 \item Best constant portfolio process, Korn/Menkens, seite 133
 \item Formulierung des Worst-Case-Problem als optimal-stopping-problem
\end{itemize}



\section{Portfoliooptimierung in kontinuirlicher Zeit mit stochastischer Steuerung und HJB-Gleichungen}
\subsection{Stochastische Steuerung}
\subsection{Anwendung auf das Portfolioproblem}
\subsection{Explizite L�sung f�r logaritmischen Nutzen}

\section{Ein spieltheoretischer MaxMin-Ansatz zur L�sung des Portfolioproblems}
\subsection{L�sung des Problems}
\subsection{Explizite L�sung f�r logaritmischen Nutzen}

\section{�bertragung des MaxMin-Ansatzes in das HJB-Setting}
\subsection{L�sung des Problems}
\subsection{Explizite L�sung f�r logaritmischen Nutzen}

\section{Simulationen}

\begin{thebibliography}{999}
	\bibitem{Korn1} Korn: "Optimal Portfolios", 1997
	\bibitem{Korn2} Korn/Willmott: "Optimal Portfolios under the Threat of Crash", 2002
	\bibitem{Korn3} Korn/Menkens: "Worst-Case Scebario Portfolio Optimization", 2005
	\bibitem{Korn4} Korn/Menkens: "Worst-Case Scebario Portfolio Optimization and HJB-Systems", 2007
	\bibitem{Korn5} Korn/Korn: "Optionsbewertung und Portfoliooptimierung", 2001
	\bibitem{G�nther} G�nther/J�ngel: "Finanzderivate mit Matlab", 2003
	\bibitem{Oksendal} Oksendal: "Stochastic Differential Equations"
	\bibitem{Fleming} Fleming: Controlled Markov Processes and Viscosity Solutions, 1993 
\end{thebibliography}

\end{document}
