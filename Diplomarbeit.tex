\documentclass[a4paper]{report}

\usepackage[ngerman]{babel}
\usepackage[latin1]{inputenc}
\usepackage[T1]{fontenc}

\usepackage{amsmath, amsthm, amssymb}
\usepackage{cancel}

\usepackage{graphicx}
\usepackage{array}

\usepackage{bibgerm}


% Mathematikumgebungen
\newtheorem{mySatz}{Satz}[section]
\newtheorem{myDefinition}[mySatz]{Definition}

\title{L�sungsans�tze f�r das Worst-Case-Portfoliooptimierungsproblem}
\author{Martin B�schen}

\begin{document}

% Definition eigener Kommandos
\newcommand{\einhalb}{\frac{1}{2}}
\newcommand{\pistern}{\frac{\mu-r}{\sigma^2}}
\newcommand{\bruch}{ 
 \left( 
 \frac{\mu-r}{\sigma}
 \right)
 }
\newcommand{\pidach}{\hat{\pi}}
\newcommand{\vdach}{\hat{v}}
\newcommand{\xdach}{\hat{X}}
\newcommand{\pistar}{\pi^{\ast}}
\newcommand{\kstar}{k^{\ast}}
\newcommand{\letzterTermv}{\left(r+\einhalb \bruch^2\right)(T-t)}
\newcommand{\standardsatz}{Es gelten die Vorraussetzungen unseres Standard"-finanz"-markt"-modells. }
\newcommand{\txbereich}{\left[0,T \right] \times \mathbb{R}^+}
\newcommand{\RM}[1]{\MakeUppercase{\romannumeral #1}}
\newcommand{\Indikatorfunktion}{\mathbf{1}}
\newcommand{\reelleZahlen}{\mathbb{R}}

\maketitle

\tableofcontents

\chapter{Einleitung}
In dieser Arbeit betrachten wir das Problem der Zusammenstellung eines optimalen Aktienportfolios. Dabei betrachten wir das Black-Scholes-Modell eines Markt mit einem Bond und
einer Aktie mit konstanten Koeffizienten. Die Preisprozesse seien also durch die Gleichungen
\begin{align*}
 dB(t) &= r B(t) dt \\
  B(0) &= 1         \\
 dS(t) &= S(t)(\mu dt + \sigma dW(t)) \\
  S(0) &= s_0
\end{align*}
gegeben, wobei $W$ eine auf einem Wahrscheinlichkeitsraum definierte Brownsche Bewegung bezeichnet. Daraus leitet sich f�r einen Investor, der mit einem Anfangsverm�gen $x_0$ sein
Verm�gens gem�� eines Portfolioprozesses $\pi$ anlegt, der duch die stochastische Differentialgleichung 
\begin{align*}
 dX(t) &= X(t)(r dt + \pi(t)(\mu -r) dt + \sigma dW(t)) \\
  X(0) &= x_0
\end{align*}
gegebene Verm�gensprozess $X$ ab. Unter einen optimalen Portfolioprozess wollen wir in dieser Arbeit einen solchen verstehen, der einen optimalen erwarteten Nutzen aus dem
Endverm�gen zu einen gegebenen Zeithorizont $T$ bringt. Bezeichnet $U$ eine Nutzenfunktion, so wurde die Aufgabe
\begin{equation*}
  \max_{\pi} E(U(X_{\pi}(T)))
\end{equation*}
schon Ende der sechziger Jahre von Merton gel�st. Dabei hat er dieses Problem als stochastisches Steuerungsproblem betrachtet und es mit Standardmethoden aus diesem Feld gel�st.
Wir werden diese Ergebnisse zitieren und den dabei gewonnenenn Verifikationssatz am beispiel der logarithmischen Nutzenfunktion nachrechnen.

Aktien verhalten sich in der Realit�t aber nicht so wie im Black-Scholes-Modell modelliert. Sie machen, ohne dass man eine Regel dahinter erkennen kann, von Zeit zu Zeit gro�e
Kursspr�nge nach unten, sie crashen. Investoren wissen das: F�r viele Nutzenfunktionen (zum Beispiel f�r den Logarithmus) maximieren konstante Portfolioprozesse den erwarteten
Endnutzen im oben formulierten Portfoliooptimierungsproblem. Trotzdem haben Investoren die Tendenz bei Ann�herung an den Zeithorizont ihre Anlagen risikoloser zu gestalten. F�r
die pers�nliche Finanzplanung gilt zum Beispiel die Faustregel, dass man (100-Lebensalter) \% seines Verm�gens in Aktien investieren sollte. Diesem
Verhalten liegt die Angst vor einem Crash zugrunde.

Wie k�nnte man Crashes nun bei der Portfoliooptimierung ber�cksichtigen? Eine erste Idee w�re, die Aktienkurse anders zu modellierem, zum Beispiel durch Levy-Prozesse. In dieser
Arbeit wollen wir aber einen ganz anderen Ansatz betrachten, n�mlich den Worst-Case-Ansatz. Dabei wird die Verteilung der Crashzeit micht modelliert, sondern man sucht Strategien,
die selbst f�r die ung�nstigste Crashzeit eine untere Schranke f�r den erwarteten Endutzen garantieren. Dieser Ansatz ist in gewisser Weise nat�rlich, denn Crashes sind seltene
Ereignisse deren Parameter man bei einer Modellierung nur schwer sch�tzen k�nnte und das gemeine an Ihnen ist, dass sie so unerwartet kommen, dass selbst
Aussagen �ber ihre Verteilung eher Spekulation sind. 

Die Aufgabenstellung ist nun also die folgende: Wenn $\tau$ den Zeitpunkt des Crashes beschreibt $k$ den Faktor, um den der Wert der Aktie schrumpft, dann ist
$(1-\pi(\tau)k)X_{\pi}(T)$ das Endvem�gen, wobei $X_{\pi}$ den Verm�gensprozess des Black-Scholes-Modells bezeichnet. Wir suchen diejenige Handelstrategie $\pi$ die den
Ausdruck
\begin{equation*}
  \max_{\pi} \min_{\tau} E \left[ U( (1-\pi(\tau)k)X_{\pi}(T)) \right]
\end{equation*}
maximiert. Diese Aufgabe bezeichen wir als Worst"=Case"=Portfoliooptimierungsproblem. Der Ansatz ist also der aus der Spieltheorie bekannte MaxMin-Ansatz. Nach dem gleichen Prinzip
spielen Computer Schach. Man geht davon aus, dass einem Selbst immer das Schlimmste passiert und versucht dann durch seine Aktion den Nutzen noch m�glichst hoch zu halten.

Im wesentlichen werden wir in dieser Arbeit zwei Artikel diskutieren, die dieses Problem l�sen. Der Artikel \cite{Korn2002} von Ralf Korn und Paul Wilmott stellt den Begriff einer
Worst-Case-Schranke in den Vordergrund und zeigt wie man durch Gleichgewichtsargumente f�r logarithmischen Nutzen einen optimalen Portfolioprozess findet. Wir versuchen, diesen
Ansatz etwas formaler aufzuschreiben und bem�hen uns um gr��ere Ausf�hrlichkeit in den Beweisen.

Desweiteren diskutieren wir den Artikel \cite{Korn2007} von Ralf Korn und Mogens Steffensen. Dort wird das Optmierungsproblem als Spiel zwischen dem Investor und dem Markt
aufgefasst. Das Hauptergebnis ist ein Verfikationssatz, der die Wertfunktion und die optimale Strategie in einem System von Ungleichungen angibt. Es �hnelt dabei den typischen
Hamilton-Jacobi-Bellman-Gleichungen aus der stochastischen Kontrolltheorie. Auch hier beweisen wir die S�tze in gr��erer Ausf�hrlichkeit als im Artikel, insbesondere an den Stellen
die dem Autor dieser Diplomarbeit Schwierigkeiten bereitet haben. F�r die logaritmische Nutzenfunktion werden wir alle Voraussetzungen des Verifikationssatzes ganz genau
nachrechnen. In einem abschlie�enden Abschnitt vergleichen wir die beiden verschiedenen Abs�tze zur L�sung des Worst-Case-Portfoliooptimierungsproblems bez�glich ihrer
Modellierung, ihrer Ergebniss und ihrer Beweismethodik.

Um die Theorie anschaulicher zu machen, werden wir schliesslich mit dem Computer noch einige Beispiele rechnen. Dabei werden wir einige theoretisch erlangte Ergebnisse
best�tigt finden. Wir werden den erwarteten Endnutzen aus dem Verm�gen durch Monte-Carlo-Simulation berechnen. Damit geben wir auch einen kleinen Algorthmus an, wie man eine
optimale lineare Startegie findet. Welche Nutzenfunktion dann speziell gew�hlt wird, spielt dabei dann keine Rolle mehr.

Diese Arbeit ist nun wie folgt organisiert. In Kapitel \ref{chap:Grundlagen} wird in die Notation eingef�hrt und einige Grundlagen bereitgestellt.
Kapitel \ref{chap:Portfoliooptimierung} bespricht das Portfoliooptimierungsproblem im Black-Scholes-Modell und dessen L�sung mittels stochastischer Steuerung. In Kapitel
\ref{chap:Wilmott} besprechen wir dann den schon erw�hnten Artikel \cite{Korn2002} und in Kapitel \ref{chap:Steffensen} den Artikel \cite{Korn2007}. Die Zahlenbeispiele und
Simulationen folgen im abschlie�enden Kapitel \ref{chap:Zahlenbeispiele}.

Alle S�tze und Definition sind deutlich gekennzeichnet und unter einem Satz versteht der Autor einfach eine mathematische Aussage, sie sind nicht nach irgendeinen Schema nach
Wichtigkeit geordnet. Im Prinzip sollte es auch m�glich sein, nur die S�tze und Definitionen zu lesen, der restliche Text sind Bemerkungen und Kommentare die f�r den Aufbau der
Theorie nicht notwendig sind. Insbesondere beim Zur�ckbl�ttern sollte es also m�glich sein, sich auf die S�tze und Defintionen zu beschr�nken, jeder Begriff wurde darin und nicht
im Nebentext eingef�hrt. Dabei halten wir unsere Notation die ganze Zeit durch und kommen ohne Doppelbezeichnungen aus.

Diese Arbeit wurde vollst�ndig mit freier Software erstellt. Gesetzt wurde der Text mit \texttt{Latex}, die Grafiken wurden mit \texttt{gnuplot} erstellt und als Programmiersprache
f�r die Beispiele im letzen Kapitel wurde \texttt{octave} verwendet. 

 


\chapter{Grundlagen}
\label{chap:Grundlagen}
\section{Stochastische Grundlagen}
\begin{mySatz} \label{StochDgl}
Die stochastische Differentialgleichung
\begin{align*}
 dX(t) &= X(t)(A(t) dt + S(t)dW(t)) \\
  X(t_0) &= x_0
\end{align*}
besitzt die eindeutige L�sung

\begin{equation*}
 X(t) = x_0 \exp \left( 
                             \int_{t_0}^{t} (A(u)-\frac{1}{2}S(u)^2)du + \int_{t_0}^{t}S(u)dW(u)
                       \right) 
\end{equation*}
\end{mySatz}

\begin{mySatz} \label{th:Ito-Formel}
Sei $X$ ein It\^{o}-Prozess der Form
\begin{equation*}
 dX(t) = K(t)dt + H(t)dW(t)
\end{equation*}
 Sei $f:[0,\infty) \times R \rightarrow R$ eine $C^{1,2}$-Funktion. Dann gilt:
\begin{equation*}
\begin{split}
&df(t,X(t)) \\
= &\left( f_t(t,X(t)) K(t) + f_x(t,X(t)) K(t) + \einhalb f_{xx}(t,X(t)) H(t)^2 \right) dt \\
  &+ f_x(t,X(t))H(t) dW(t) 
\end{split}
\end{equation*}

\end{mySatz}



\section{Finanzmarktmodell}
Im folgenden geben wir die Definition unseres Finanzmarktmodells.
\begin{myDefinition} \label{Standardfinanzmarktmodell}
Sei $(\Omega,F,P) $ ein Wahrscheinlichkeitsraum. Sei $W$ eine auf diesen Raum definierte Brownsche
Bewegung. Seien $r$, $\mu$, $\sigma$ reelle Zahlen und sei $r < \mu$ und $\sigma>0$. Der Bond $B$ und die
Aktie $S$ seien die eindeutigen L�sungen der stochastischen Differentialgleichungen
\begin{align}
 dB(t) &= r B(t) dt \\
  B(0) &= 1         \\
 dS(t) &= S(t)(\mu dt + \sigma dW(t)) \\
  S(0) &= s_0
\end{align}
Dabei nennen wir $r$ den Zinssatz, $\mu$ die Aktienrendite und $\sigma$ die Volatilit�t.
Au�erdem sei eine Zahl $k$ mit $0<k<1$ gegeben, die wir als maximale Crashh�he bezeichnen.
Weiterhin sei eine Konstante $c>0$ gegeben mit $c>\max(1/k,\pistern)$, die wir Handelsschranke nennen. 
Desweiteren nennen wir $x_0>0$ unser Startverm�gen und $T>0$ unseren Zeithorizont.
Gelten alle diese Voraussetzungen, so sagen wir dass die Voraussetzungen unseres \textbf{Standardfinanzmarktmodells} gelten.
\end{myDefinition}
Die Interpretation unseres Standardfinanzmarktmodells ist allein durch die Namensgebung schon ziemlich klar. Mit der Handelsschranke dr�cken wir aus, dass es verboten sein soll,
mehr als das $c$-fache seines Verm�gens in Aktien anzulegen, also Aktien auf Kredit zu kaufen (falls $c>1$), als auch mehr als $(-c)$-fache sein Verm�gens in Aktien anzulegen, also
Aktien leer zu verkaufen. Wir brauchen dieses Handelsschranke aus beweistechnischen Gr�nden, aber man kann auch argumentieren, dass es eine vern�nftige Modellannahme ist in einer
Welt mit Gesetzesbeschr�nkungen und einer endlichen Geldmenge. Ein Investorin, die nicht diesen Beschr�nkungen unterliegt und die Resultate dieser Arbeit nutzen m�chte, kann sich
$c$ einfach als riesig gro� vorstellen, so da es praktisch keine Einschr�nkung mehr f�r sie darstellt.

Dieses Definition ist ein wenig auf Vorrat gemacht, sie dient dazu, dass wir mit dem rituellen Satz ``\standardsatz'' alle ben�tigten Vorausetzungen zusammen haben. Der Parameter
$k$ zum Beispiel wird erst in sp�teren Kapiteln ben�tigt.

Um die folgenden S�tze formulieren und beweisen zu k�nnen, h�tten wir im Prinzip nicht den Bond $B$ und die Aktie $S$ definieren m�ssen. Wir werden im folgenden eine
Verm�gensgleichung definieren, die die Modellparameter enth�lt, aber nicht direkt den Aktienkurs und den Bond. Dieses Arbeit besch�ftigt sich mit den Eigenschaften dieser
Verm�gensgleichung.

W�hrend die Parameter $\mu,r,\sigma$ und $k$ den Markt beschreiben, beschreiben $x_0$ und $T$ unsere pers�nliche Vorgaben.

Im folgende werden wir noch einige Trivialit�ten �ber eine bestimmte Parabel angeben. Wir geben ihr sogar einen Namen, denn sie wird in dieser Arbeit immer wieder als Abk�rzung
auftauchen und viele Argumente basieren auf der genauen Kenntnis ihres Verlaufes.


\begin{myDefinition} \label{def:f}
Es gelten die Voraussetzungen unseres Standardfinanzmarktmodells. Sei $f$ durch 
\begin{equation} \label{eq:fvonpi}
 f(\pi) =  \pi (\mu-r)-\einhalb \pi^2 \sigma^2
\end{equation}
definiert.
\end{myDefinition}

% Satz und definition
\begin{mySatz} \label{th:Eigenschaften_f}
Sei $\pistar:=\pistern$. Mit dieser Bezeichnung besitzt $f$ die beiden Nullstellen $0$ und $2\pistar$, nimmt sein Maximum an der Stelle $\pistar$ an und hat dort den Wert
\begin{equation} 
 f(\pistar) =  \einhalb \bruch^2
\end{equation}
\end{mySatz}
In Abbildung \ref{fig:graph_von_f} sind die Aussagen �ber $f$ nochmal veranschauchlicht.
\begin{figure}
 \begin{center}
  \input{Bilder/plotf.tex}
 \end{center}
\caption{Grapf von f}
\label{fig:graph_von_f}
\end{figure}


\begin{myDefinition} \label{def:Handelsstrategie}
\standardsatz
Die Menge $\mathcal{A}$ aller progressiv messbaren Prozesse $\pi$, f�r die die stochastische Differentialgleichung
\begin{align*}
 dX(t) &= X(t) \left[ \left( r + \pi(t)(\mu -r) \right) dt + \pi(t) \sigma dW(t) \right] \\
  X(0) &= x_0
\end{align*}
eine eindeutige L�sung besitzt, f�r die $P$- fast sicher
\begin{equation}
 \int_0^T  \left( X(t) \pi(t) \right)^2 dt < \infty
\end{equation}
und
\begin{equation}
 X(t) \geq 0
\end{equation}
f�r alle $t \in [0,T]$ gilt, hei�t die Menge der \textbf{zul�ssigen Handelsstrategien}.
\end{myDefinition}

\begin{myDefinition} \label{Handelsstrategie}
\standardsatz
Die Menge $\mathcal{M}$ aller Funktionen $p$ von $\txbereich$ nach $\reelleZahlen$ hei�t die Menge der \textbf{zul�ssigen Markov-Handelstrategien}, wenn die stochastische
Differentialgleichung
\begin{align*}
 dX(t) &= X(t) \left[ \left( r + p(t,X(t))(\mu -r) \right) dt + p(t,X(t)) \sigma dW(t) \right] \\
  X(0) &= x_0
\end{align*}
eine eindeutige L�sung besitzt und mit dieser L�sung  $p(\cdot,X(\cdot)) \in \mathcal{A}$ gilt.
\end{myDefinition}
In dem gerade gegebenen Sinne k�nnen wir also eine Markovstrategie auch immer als eine normale Handelstrategie auffassen und wir werden in dem Fall nicht mehr zwischen diesen
beiden Begriffen unterscheiden.

Nun geben wir die wichtige Definition der Verm�gensgleichung. Letztendlich handelt dieses Arbeit von den Eigenschaften einer solchen Verm�gensgleichung.
\begin{myDefinition} \label{Verm�gensgleichung}
\standardsatz
Sei $\pi$ eine Handelsstrategie.
 Die L�sung von
\begin{align*} \label{eq:Verm�gensgleichung}
 dX(t) &= X(t) \left[ \left( r + \pi(t)(\mu -r) \right) dt + \pi(t) \sigma dW(t) \right] \\
  X(0) &= x_0
\end{align*}
bezeichnen wir als \textbf{Verm�gensprozess} $X_{\pi}$. Als eine Abk�rzung vereinbaren wir weiter
\begin{equation*}
 X_{\pi,t,x}(s) = E \left[ X_{\pi}(s) | X_{\pi}(t)=x \right]
\end{equation*}
\end{myDefinition}
Wie in der vorhergehenden Bemerkung schon bemerkt, lassen wir es auch zu, dass statt $\pi$ eine Markovstrategie $p$ eingesetzt wird. Streng genommen ist $X$ dann eine L�sung der
Differentialgleichung in der Definition der Markovstrategie.


Die L�sung der Verm�gensgleichung l�sst sich auch ganz explizit als stochastisches Integral hinschreiben. Davon handelt der n�chste Satz.

\begin{mySatz} \label{th:expliziteL�sungVerm�gensgleichung}
Es gilt:
\begin{align*} \label{eq:L�sVerm�gensgleichung}
                  & X_{\pi,t_0,x}(t) \\
  &= x \exp \left(
                                   \int_{t_0}^{t} r + \pi(u)(\mu -r) - \frac{1}{2} \pi(u)^2 \sigma^2 du 
                                  +\int_{t_0}^{t}\pi(u)\sigma dW(u)) 
                            \right) \\
                  &= x \exp \left(
                                   \int_{t_0}^{t} r + f(\pi(u)) du 
                                  +\int_{t_0}^{t}\pi(u)\sigma dW(u)) 
                            \right)  \\
                 &= x \exp \left(
                                   \int_{t_0}^{t} r + \pi(u)(\mu -r) du
                           \right) 
                     \exp \left(      
                                \int_{t_0}^{t}\pi(u)\sigma dW(u)) - \frac{1}{2}  \int_{t_0}^{t} \pi(u)^2 \sigma^2 du
                            \right) \\
\end{align*}
In der letzten Gleichung ist dabei der letzte Faktor ein Martingal.

\end{mySatz}




Ein spezieller Augenmerk wird in dieser Arbeit auf dem sogennannten logaritmischen Endnutzen liegen. Daf�r stellen wir schon einmal ein Ergebnis bereit. Das folgt aus der
expliziten Darstellung des Verm�genprozesse aus dem vorangegangen Satz und der Tatsache dass unter unserem Bedingungen an die Handelsstrategie das stochastische Integral einen
Erwartungswert von 0 besitzt.


\begin{mySatz} \label{th:ExpLogUtility}
Es gilt
 \begin{equation} \label{eq:ExpLogUtility}
\begin{split}
 & E \left[ \log(X_{\pi,t,x}(T)) \right] \\
 = &\log x  + E \left[ \int_{t}^{T} (r + \pi(u)(\mu -r)-\frac{1}{2}\pi(u)^2\sigma^2)du  \right]  \\
 = & \log x  + E \left[ \int_{t}^{T} f(\pi(u)) du  \right] + r(T-t) \\
\end{split}
\end{equation}
\end{mySatz}



\chapter{Portfoliooptimierung im Black-Scholes-Modell: Der Ansatz �ber stochastische Steuerung}
\label{chap:Portfoliooptimierung}
\section{Das Portfoliooptimierungsproblem}
Wir stellen uns nun die Frage, wie wir zu unserem gegebenes Anfangskapital $x_0$ uns im Black-Scholes-Modell optimal verhalten, also welche Handelstrategie wir
verwenden. Dazu m�ssen wir erstmal pr�zise machen, was wir mit optimal meinen. Als Ziel setzen wir uns hier das Errreichen eines optimalen Endverm�gens. Um das quantifizieren zu
k�nnen, f�hren wir zun�chst den Begriff einer Nutzenfunktion ein.
\begin{myDefinition}
 Sei $U:(0,\infty) \rightarrow \reelleZahlen$ strikt konkav, stetig differenzierbar und es gelte
\begin{equation}
 \lim U'(x)=\infty, \qquad \lim U'(x)=0.
\end{equation}
Dann nennen wir $U$ eine \textbf{Nutzenfunktion}.
\end{myDefinition}

Im folgenden geben wir nun ein Funktional an, dass jeder Strategie einen erwarteten Nutzen zuweist. Optimales Verhalten hei�t dann, eine Strategie zu w�hlen, so dass dieses
Funktional maximiert wird.
\begin{myDefinition} \label{def:Standardportfoliooptimierungsproblem}
\standardsatz
Sei $U$ eine Nutzenfunktion, sei $\pi$ eine zul�ssige Handelstrategie und sei $X_{\pi}$ der Verm�gensprozess.
Wir bezeichnen mit
\begin{equation}
 J^0(x,t,\pi,U) = E \left[ U(X_{\pi}(T)) | X_{\pi}(t)=x \right] 
               := E_{t,x} \left[ U(X_{\pi}(T)) \right] 
\end{equation}
den \textbf{erwarteten Endnutzen} durch Verfolgung von  $\pi$ bez�glich $U$.
Die Aufgabe
\begin{equation}
 \text{maximiere} \qquad J^0(x_0,0,\pi,U)
\end{equation}
hei�t \textbf{(Standard-)Portfoliooptimierungsproblem}.
Die Funktion
\begin{equation}
 V^0(t,x,U) = \sup_{\pi \in \mathcal{A}} J^0(x,t,\pi,U)
\end{equation}
hei�t Wertfunktion des Portfoliooptimierungsproblem und eine Strategie f�r die das Surpremum angenommen wird hei�t \textbf{optimale Strategie} f�r $(t,x)$.
\end{myDefinition}
Die Bedeutung der $0$ in der Notation wird in sp�teren Kapiteln klar werden. Sie wurde hier nur schon in die Notation aufgenommen um die Notation in der ganzen Arbeit konsistent
zu halten. 
\section{Die L�sung des Portfoliooptimierungsproblem}
Mit Methoden der stochastischen Steuerung kann nun gezeigt werden:
\begin{mySatz} \label{th:Verifikationsatz}
\standardsatz 
Sei $v \in C^{1,2}$ und f�r alle $(t,x)$ gelte 
\begin{equation} \label{eq:HJBPortfolio} 
 \underset{\pi \in (-c,c)}{\text{max}}  \einhalb \pi^2 \sigma^2 x^2 v_{xx}(t,x) + (r+\pi (\mu-r) x)
v_{x}(t,x) + v_t(t,x)=0  
\end{equation}
mit
\begin{equation} \label{eq:HJBEndbedingung}
 v(T,x)=U(x)
\end{equation}
Sei au�erdem $p$ ein zul�ssiger Portfolioprozess sodass f�r alle $(t,x)$
\begin{equation}  \label{eq:optimaleHandelsstrategie}
  \einhalb p(t,x)^2 \sigma^2 x^2 v_{xx}(t,x) + (r+ p(t,x) (\mu-r) x)
v_{x}(t,x) + v_t(t,x)=0 
\end{equation}
gilt. Dann gilt 
\begin{equation}
 V^0(t,x,U)=v^0(t,x)
\end{equation}
und $p$ ist der optimale Portfolioprozess.
\end{mySatz}


\section{Explizite L�sung f�r logaritmischen Nutzen} \label{sec:log1}
Als Anwendung des vorangegangen Satzes berechnen wir das optimale Portfolio und die Wertfunktion f�r den Fall des logarithmischen Nutzens.
\begin{mySatz} \label{th:LogWertfunktion}
Die Wertfunktion f�r das Portfoliooptimierungsproblem f�r logarithmischen Nutzen ist gegeben durch
\begin{equation} \label{eq:LogWertfunktion}
 V^0(t,x,\log)= \log(x)+ f^0(t) 
\end{equation}
mit
\begin{equation*}
f^0(t) = \left(r + \einhalb \bruch^2 \right) (T-t)
\end{equation*}
und
\begin{equation*}
 \pistar=\pistern 
\end{equation*}
ist dabei die optimale Strategie.
\end{mySatz}

\begin{proof}
Wir m�ssen lediglich alle Vorausetzungen von Satz \ref{th:Verifikationsatz} �berpr�fen.
Sei dazu
\begin{equation} \label{eq:kanditatf�rlognutzen}
  v(t,x)= \log(x)+ f^0(t) 
\end{equation}
und wir berechnen zun�chst die ben�tigten Ableitungen.
\begin{align*}
v_t(t,x) &= - \left( r+\frac{1}{2} \bruch^2 \right) \\
v_x(t,x) &= \frac{1}{x} \\
v_x(t,x) &= -\frac{1}{x^2} \\
\end{align*}
Einsetzen in \eqref{eq:HJBPortfolio} ergibt:
\begin{equation*}
\begin{split}
    &\einhalb \pi^2 \sigma^2 x^2 v_{xx}(t,x) + (r+\pi(b-r)) x v_x(t,x) + v_t(t,x) \\
   =&\einhalb \pi^2 \sigma^2 x^2 \frac{1}{x^2} + (r+\pi(b-r)) x \frac{1}{x} + - \left( r+\frac{1}{2} \bruch^2 \right) \\
  =& -\pi^2\einhalb \sigma^2+\pi(b-r)-\einhalb \bruch^2 \\
  = & f(\pi) -\einhalb \bruch^2
\end{split}
\end{equation*}
Dabei haben wir in der letzten Gleichung das f aus Definition \ref{def:f} eingesetzt. Das Maximum von $f$ ist, wir im Satz \ref{th:Eigenschaften_f} angegeben,
$\einhalb \bruch^2$. Maximieren der vorhergehenden Gleichung ergibt also tats�chlich $0$. Weiter wissen wir aus Satz \ref{th:Eigenschaften_f}, dass dass Maximum von $f$ in
$\pistar$ angenommen wird, also ist auch die Gleichung \eqref{eq:optimaleHandelsstrategie}, die die optimale Strategie charakterisiert erf�llt. 
Au�erdem erf�llt \ref{eq:kanditatf�rlognutzen} auch die Endbedingung: 
\begin{equation}
 v(T,x)= \log x + \left( r+\frac{1}{2} \bruch^2  \right) (T-T)= \log x +0 = \log x
\end{equation}
Damit sind alle Bedingungen von Satz \ref{th:Verifikationsatz} erf�llt und die Behauptung ist bewiesen.
\end{proof}

Die Wertfunktion und die optimale Strategie fallen nat�rlich nicht vom Himmel. Normalerweise geht man umgekehrt vor und versucht die Differentialgleichung zu l�sen. Bei solch
einem Vorgehen besteht immer die Gefahr, dass man notwendige und hinreichende Bedingungen durcheinander wirft. Wir haben uns hier darauf beschr�nkt nur die L�sung anzugeben und zu
verifizieren, dass sie tats�chliche die Voraussetzungen erf�llt.


Zur Probe berechnen wir auch noch den besten konstanten Portfoliprozess, dabei erwarten wir nat�rlich das
gleiche Ergebnis. Wir m�ssen dazu nur das konstante $\pi$ finden, welches die Gleichung \ref{eq:ExpLogUtility} aus Satz \ref{th:ExpLogUtility} maximiert. F�r ein konstantes $\pi$
vereinfacht sich diese zu
\begin{equation*}
E \left[ \log(X_{\pi}(t)) \right] = \log x  +  (r+f(\pi))T 
\end{equation*}
und man erkennt wieder, dass wir das $\pi$ w�hlen m�ssen das $f$ maximiert. Das ist wieder
\begin{equation*}
 \pi=\pistern
\end{equation*}
und wir sind in unserer Probe best�tigt worden.

\chapter{Portfoliooptimierungerung unter Crashgefahr: Der Gleichgewichts"-ansatz}
\label{chap:Wilmott}

\section{Portfoliooptimierungerung unter Crash"-gefahr: Erste Formulierung}
Das Modell unseres Standardfinanzmarktes hat den Nachteil, dass es den Aktienkurs als stetige Funktion der
Zeit modelliert. In der Realit�t beobachtet man aber Kursspr�nge, insbesondere nach unten, die man als Crash
bezeichnet.
Das modellieren wir nun dadurch, indem wir annehmen, dass es innerhalb des Zeithorizonts zu einem
Fall des Aktienkuses um das k-fache seines Wertes kommen kann. Dabei gelte 
$k\in [0,\kstar]$ und $0<\kstar<1$. Bezeichnet $t$ den Crashzeitpunkt und $S$ den Aktienkurs, so l�sst sich dieser Sachverhalt
mittels der folgenden Formel  
\begin{equation*}
  k=\frac{S(t-)-S(t)}{S(t-)} \Leftrightarrow S(t)=(1-k)S(t-).
\end{equation*}
ausgedr�cken. Auf welchen Faktor schrumpft nun das Verm�gen $X$ bei solch einem Crash? Anders ausgedr�ckt, welches $a$ l�st die Gleichung
\begin{equation*}
  aX(t-)=X(t)?
\end{equation*}
Dazu f�hren wir die Rechnung
\begin{align*}
  X(t) &= \underbrace{(1-\pi(t))X(t-)}_{\text{Bondverm�gen}}  + \underbrace {\pi(t) (1-k)
X(t-)}_{\text{gecrashtes Aktienverm�gen}} \\
          &= (1-\pi(t)k) X(t-)
\end{align*}
durch, also gilt $a=1-\pi(t)k$. Wie sieht nun das zu einem Verm�gensprozess geh�rige Endverm�gen aus, wenn man in diesem
Chrashscenario handelt. Sei dazu $X(x,t_1,t_2)$ das Endvem�gen, wenn man beginnend mit dem Verm�gen
$x$ in $t_1$ bis zum Zeitpunkt $t_2$ gem�� dem Verm�gensprozess $\pi$ im Black-Scholes-Modell handelt. Den Wert von $X(x,t_1,t_2)$ haben wir in 
Satz \ref{th:expliziteL�sungVerm�gensgleichung} auch explizit angegeben. F�r das Endverm�gen ist dann der Ansatz  
\begin{equation*}
 \text{EV}= X((1-\pi(t)k) X(x_0,0,t),t,T)
\end{equation*}
sinnvoll, denn bis zum Crashzeitpunkt $t$ sammelt man das Verm�gen $X(x_0,0,t)$ an, der Crash verringert es auf
$(1-\pi(t)k) X(x_0,0,t)$ und dann hat man noch bis $T$ Zeit es weiter zu vermehren. Wir berechnen nun:
\begin{align*}
  X(T) &= (1-\pi(t)k) X(x_0,0,t) \exp(\int_t^T \dots ) \\
       &= (1-\pi(t)k) x_0 \exp(\int_0^t \dots )  \exp(\int_t^T \dots ) \\
       &= (1-\pi(t)k) x_0 \exp(\int_0^T \dots ) \\
       &= (1-\pi(t)k) X(x_0,0,T).
\end{align*}
Der Crash beeinflusst das Endverm�gen also nur um einen (zeitabh�ngigen) Faktor. Das h�ngt mit der
Exponentialgestalt des Verm�gensprozesses zusammen, der relative Verm�genszuwachs h�ngt n�mlich nicht von der
H�he der Verm�gens ab.

Diese mehr oder weniger heuristische �berlegungen motivieren nun das folgende Modell f�r das Endverm�gen.
Bezeichnen wir wieder mit $X_{\pi}$ unseren Verm�gensprozess im crashfreien Scenario und erlebt man zur Zeit $t$ einen Crash
der H�he $k$, dann hat man in $T$ das Verm�gen
\begin{equation*} 
  (1-\pi(t)k) X_{\pi}(T).
\end{equation*}
Wie sollte man sich in solchen einen Scenario nun verhalten, um einen m�glichst hohen Nutzen aus seinen Endverm�gen zu ziehen?
In \cite{Korn2002} wird vorgeschlagen eine Strategie $\pi$ zu suchen, die das Surpremum des Ausdrucks
\begin{equation}  \label{eq:kornmaxmin}
  \sup_{\pi} \inf_{0 \leq t \leq T, 0 \leq k \leq \kstar} E \left[ U((1-\pi(t)k) X_{\pi}(T)) \right]
\end{equation}
annimmt. Dieser Ansatz stammt aus der Spieltheorie, man versucht also den minimalen Nutzen zu maximieren.
Diesen MaxMin-Ansatz m�chte ich grunds�tzlich �bernehmen, kritisiere die Formulierung \ref{eq:kornmaxmin}
aber aus zweierlei Gr�nden. Zum einen wollen wir auch Strategien modellieren, die auf den Crash
reagieren k�nnen. Nach dem Crash sollte idealerweise ein Wechsel auf die optimale Stategie des crashfreien Scenarios erfolgen. Das wird hier
aber nicht mitmodelliert.
Wir k�nnen das ausdr�cken, indem wir
\begin{equation*}
   U((1-\pi(t)k) X_{\pi,x}(T))
\end{equation*}
durch
\begin{equation*} \label{eq:WCCrashScenario}
   V^0(t,(1-\pi(t)k) X_{\pi,x}(t),U)
\end{equation*}
ersetzen. Das wird in \cite{Korn2002} auch immer wieder benutzt. Zum anderen soll der Markt auch die M�glichkeit haben, gar nicht zu crashen. In der gegebenen
Formulierung \eqref{eq:kornmaxmin} k�nnte man argumentieren, dass sei durch einen Crash der H�he $k=0$ mitmodelliert. Wir wollen
das aber explizit machen, indem wir versuchen das Minimum aus \eqref{eq:WCCrashScenario} und dem dem erwarteten
Nutzen im Crashfreien Scenario zu maximieren. Das ist in der Modellierung sauberer und wird auch sp�ter die
Beweise durchsichtiger machen. Dann ist es auch nicht mehr n�tig, �ber verschieden Crashh�hen zu minimieren.
Ist unsere Aktienposition positiv, so schadet uns ein Crash der maximalen Gr��e am meisten, ist sie negativ
(wenn wir also Aktien leerverkauft haben), so schadet uns das Ausbleiben eines Crashes am meisten. 

Alle diese Gedanken ber�cksichtigend, gelangen wir so zur folgenden
\begin{myDefinition} \label{def:WCPortfolioproblem1}
\standardsatz 
Sei $\pi$ eine zul�ssige Handelsstrategie, sei $X_{\pi}$ der Verm�gensprozess und sei $U$ eine Nutzenfunktion.
Wir ordnen jeder Handelsstrategie die Worst-Case-Schranke
\begin{equation*}
 WC(\pi,t,x,U) =  \min \left( 
                   \inf_{t \leq s \leq T} E_{t,x} \left[ V^0(s,X_{\pi}(s)(1-\pi(s)k),U) \right] 
                  , 
                   E_{t,x} \left[ U(X_{\pi}(T)) \right] 
                  \right)
\end{equation*}
zu. Die Aufgabe 
\begin{equation*}
 \text{maximiere} \qquad  WC(\pi,0,x_0,U)
\end{equation*}
bezeichnen wir als \textbf{Worst-Case-Portfoliooptimierungsproblem \RM{1}}. 
Eine Strategie $\pi_{\text{opt}}$ f�r die
\begin{equation*}
 \sup_{\pi \in \mathcal{A}} WC(\pi,t,x,U) =  WC(\pi_{\text{opt}},t,x,U)
\end{equation*}
f�r alle $(t,x)$ gilt, bezeichnen wir als optimale Strategie.
\end{myDefinition}

Die Forderung an eine optimale Strategie, dass das Surprememum f�r alle $(t,x)$ angenommen werden muss, hat damit zu tun, dass wir wollen, dass unsere Strategie sich auch noch
optimal verh�lt, wenn bis kurz vor Ende des Zeithorizonts kein Crash geschehen ist. A priori k�nnen wir n�mlich nicht ausschlie�en, dass das Infimum der Worst-Case-Schranke weit
vorne angenommen wird und dann br�uchte sich die Handelstrategie ohne unsere Forderung danach nicht mehr optimal zu verhalten.

\section{Erste Eigenschaften}

Wir berechen zur Illustration die Worst-Case-Schranke f�r die reinen Bondstrategie bez�glich des logaritmischen Nutzen. Intuitiv ist klar, das
der Worst-Case das Ausbleiben eines Crashes ist. Denn ein Crash verursacht uns keinen Schaden, aber gibt uns
die M�glichkeit in die optimale Stategie des crashfreien Scenarios zu wechseln. Das rechnen wir im folgenden Satz nach.
\begin{mySatz}[Worst-Case-Schranke der reinen Bondstrategie, \cite{Korn2002}, 2.1.(i)] \label{th:worst-case-schranke-reine-bondstrategie}
Sei $\pi_0 \equiv 0$. Dann gilt:
\begin{equation*}
 WC(\pi_0,t,x) =  \log x + r(T-t).
\end{equation*}
\end{mySatz}
\begin{proof}


Wir berechnen zun�chst f�r beliebieges $t$ und $x$ und $t \leq s \leq T$ (Erkl�rungen folgen)
\begin{equation*}
\begin{split}
   &E_{t,x} \left[ V^0(s,X_{\pi_0}(s)(1-\pi_0k), \log) \right] \\
 =  &E_{t,x} \left[ V^0(s,X_{\pi_0}(s),\log) \right] \\
 = &E_{t,x} \left[ V^0(s,x \exp(r(s-t)),\log) \right] \\
 = &V^0(s,x \exp(r(s-t)),\log) \\
 = &\log x + r(s-t) + \left[ r + \einhalb \bruch^2 \right](T-s) \\
 = &\log x + r(T-t) + \left[\einhalb \bruch^2 \right](T-s)
\end{split}
\end{equation*}
Dabei haben wir in der ersten Gleichung den Faktor $1$ wegelassen, in der zweiten Gleichung haben $X_{\pi_0}$ eingesetzt, in der dritten Gleichung haben wir den Erwartungswert
weggelassen, da sein Argument nicht stochastisch ist, in der vierten Gleichung haben wir die explizite Form von $V^0(\cdot,\cdot,\log)$ aus Satz \ref{th:LogWertfunktion}
benutzt und in der vierten Gleichung haben wir die Terme zusammengefasst. Nimmt man in dieser Gleichung das Infimum �ber alle $s$, so f�llt der letzte Summand offenbar weg. Weiter
gilt: 
\begin{equation*}
  E_{t,x} \left[ \log(X_{\pi_0}(T) \right] = \log x + r(T-t)
\end{equation*}
Damit folgt die Behauptung.
\end{proof}

Als n�chstes berechen wir die Worst-Case-Schranke f�r die optimalen Strategie $\pistar=\pistern$ des crashfreien Scenarios. Hierbei wird der Portfolioprozess konstant gehalten,
auch nach dem Crash beh�lt er seinen Wert bei. Ein Crash beinflusst also, unabh�ngig von seinem
Zeitpunkt, das Endverm�gen um einen konstanten Faktor. Wir erwarten also als Worst-Case-Scenario einen Crash zu einem beliebigen Zeitpunkt. Das wird sich im Beweis des folgenden
Satzes auch so herausstellen.
\begin{mySatz}[Worst-Case-Schranke der optimalen Strategie im crasfreien Scenario, \cite{Korn2002}, 2.1.(ii)] \label{th:worst-case-schranke-optimal-ohne-crash}
Es gilt:
\begin{equation*} 
 WC(\pistar,t,x) = \log x +r(T-t) + \einhalb \bruch^2(T-t) + \log(1-\pistar k).
\end{equation*}
\end{mySatz}
\begin{proof}
Wir berechnen zun�chst f�r beliebieges $t$ und $x$ und ein $s$ mit $t \leq s \leq T$
\begin{equation*}
\begin{split}
  &  E_{t,x} \left[ V^0(s,X_{\pistar}(s)(1-\pistar k), \log) \right] \\
 =&E_{t,x} \left[ \log(X_{\pistar}(s)) + \log(1-\pistar k)  + \left( r+ \einhalb \bruch^2(T-s) \right) \right] \\
 =& \log x + \left( r+ \einhalb \bruch^2 \right)(s-t) \\
  &+ \log(1-\pistar k)  + \left( r+ \einhalb \bruch^2(T-s) \right)  \\
 =& \log x + \left( r+ \einhalb \bruch^2 \right)(T-t) + \log(1-\pistar k)  \\
\end{split}
\end{equation*}
Dabei haben wir in der ersten Gleichung die explizite Form von $V^0(\cdot,\cdot,\log)$ aus Satz \ref{th:LogWertfunktion} benutzt, in der zweiten Gleichung den erwarteten
logaritmischen Nutzen eingesetzt (Satz \ref{th:ExpLogUtility}) und in der dritten Gleichung haben wir die Terme zusammengefasst. Die Variable $s$ kommt also, wie erwartet, gar
nicht mehr darin vor.
Im crashfreien Scenario
\begin{equation*}
  E_{t,x} \left[ \log(X_{\pi_0}(T) \right] = \log x + r(T-t) + \einhalb \bruch^2 (T-t)
\end{equation*}
ist der erwartete Nutzen nat�rlich h�her. Damit folgt die Behauptung.
\end{proof}

Der Unterschied zwischen diesen beiden Schranken liegt also im Term 
\begin{equation*}
 \einhalb \bruch^2 T +\log(1-\pistar k).
\end{equation*}
Nur wenn der Zeithorizont gro� genug ist ist, wird dieser
Term positiv. Dann �berwiegt die h�here Rendite der Aktie den Schaden, der uns einmal durch den Crash zugef�gt wurde. Eine optimale Strategie sollte diese Effekte ausbalancieren
und je n�her man sich crashfrei dem Ende des Zeithorizonts gen�hert hat, desto vorsichtiger sollte man in Aktien investieren. Danach hat man n�mlich kaum noch Zeit, die erlittenen
Verluste wieder auszugleichen. Das ist im Prinzip die wesentliche Eigenschaft von optimalen Strategien auf M�rkten mit Crashgefahr. Das pr�zise zu machen und schliesslich auch zu
quantifizieren ist das Ziel dieser Arbeit.

Als n�chste Eigenschaft zeigen wir im folgenden noch, dass es niemals n�tig ist, Aktien leer zu verkaufen um einen optimalen Nutzen zu erreichen. Auf der Suche nach einer optimalen
Strategien d�rfen wir uns dann auf solche beschr�nken, die nach unten durch $0$ beschr�nkt sind. Wir zeigen dieses Aussage nur f�r logaritmischen Nutzen. Der Autor hat das starke
Gef�hl, dass die Aussage f�r beliebige Nutzenfunktionen richtig ist, kann es aber nicht streng beweisen. Wir werden auf die diesbez�glich kritischen Stellen im Beweis aufmerksam
machen.
\begin{mySatz}[Keine Notwendigkeit von Leerverk�ufen, \cite{Korn2002}, 2.3. (b)] \label{th:Unn�tzeLeerverk�ufe}
F�r logaritmischen Nutzen gilt: $WC(\pi^{+},t,x) \geq WC(\pi,t,x)$
\end{mySatz}
\begin{proof}
 Der Beweis ist so organisiert, dass wir erst seine Struktur zeigen, und dann die Details ausarbeiten. Sei
\begin{align*}
 A &:= \inf_{t\leq s \leq T} E_{t,x} \left[ V^0(t,X_{\pi}(s),U) \right] \\
 B &:= E_{t,x} \left[ U(X_{\pi}(T)) \right] \\
 A^+ &:= \inf_{t\leq s \leq T} E_{t,x} \left[ V^0(t,X_{\pi^+}(s),U) \right] \\
 B^+ &:= E_{t,x} \left[ (U(X_{\pi^+}(T)) \right] \\
\end{align*}
Wir haben also
\begin{equation*}
 \min(A,B) \leq \min(A^+,B^+)
\end{equation*}
zu zeigen.
Wir zeigen zun�chst, dass im crashfreien Scenario $\pi^+$ einen h�heren erwarteten Nutzen bringt, also
\begin{equation} \label{eq:eins}
 B\leq B^+
\end{equation}
Wenn $A\leq A^+$ sind wir wegen $min(A,B)\leq B \leq B^+$ und $min(A,B)\leq A \leq A^+$, also auch $min(A,B)
\leq min(A^+,B^+)$ fertig.
Sei also $A>A^+$. Dann gilt f�r alle $s$ mit $t \leq s \leq T$
\begin{equation} \label{eq:zwei}
  \pi(s) < 0.
\end{equation}
Ddann ist $\pi^+=0$ und wie bei der Berechnung der Worst-Case-Schranke f�r die reine Bondstrategie kann man
\begin{equation*} 
 A^+=B^+
\end{equation*}
zeigen. F�r $\pi$ ist aber das Worst-Case-Scenario das Ausbleiben eines Crashes, denn die leerverkauften Aktien werden billiger, also
\begin{equation} \label{eq:drei}
 A \geq B,
\end{equation}
Insgesamt folgt also
\begin{equation*}
 \min(A,B)=B \leq B^+ = A^+ = \min(A^+,B^+)
\end{equation*}
Bis hier hin haben wir noch nicht benutzt, dass die Nutzenfunktion der Logarithmus ist. Das wird aber massiv bei der Ausarbeitung der noch zu zeigenden Gleichungen
\eqref{eq:eins}, \eqref{eq:zwei} und \eqref{eq:drei} benutzt.
\begin{itemize}
 \item ad \eqref{eq:eins}:
      \begin{equation*}
       \begin{split}
         & E_{t,x}\left[ \log \left( X_{\pi}(T) \right) \right] \\
       =& \log x + E_{t,x}\left[ \int_t^T f(\pi(u)) du \right] + r(T-t) \\
      \leq & \log x + E_{t,x}\left[ \int_t^T f(\pi^+(u)) du \right] + r(T-t) \\
      =&  E_{t,x}\left[ \log \left( X_{\pi^+}(T) \right) \right] \\
       \end{split}
      \end{equation*}
 \item ad \eqref{eq:zwei}:
    \begin{equation*}
       \begin{split}
	        & A>A^+ \\
    \Rightarrow & \inf_s \log (1-\pi(s)k) + \log(X_{\pi}) + f^0(t) \\
                & > \inf_s \log (1-\pi^+(s)k) + \log(X_{\pi^+}) + f^0(t)  \\
    \Rightarrow & \inf_s \log (1-\pi(s)k)  \\
                & > \inf_s \log (1-\pi^+(s)k) + \log(X_{\pi^+}) -\log(X_{\pi}) \\
    \Rightarrow &  \inf_s \log (1-\pi(s)k) > \inf_s \log (1-\pi^+(s)k) \\
    \Rightarrow &  \log (1- \sup_s \pi(s)k) > \log (1-\sup_s pi^+(s)k) \\
    \Rightarrow &   1- \sup_s \pi(s)k > 1-\sup_s \pi^+(s)k \\
    \Rightarrow &   \sup_s \pi^+(s)k > \sup_s \pi(s)k \\
    \Rightarrow &   \forall s: \pi(s) < 0 \\
       \end{split}
      \end{equation*}
  \item ad \eqref{eq:drei}:
\begin{equation*}
       \begin{split}
	     B = & E_{t,x}\left[ \log(X_{\pi}(T)) \right] \\
                = & \inf_s E_{t,x} \left[ V^0(s,X_{\pi}(s),\log) \right] \\
                = & \inf_s E_{t,x} \log X_{\pi}(s) + ... \\
             \leq& \inf_s E_{t,x} + \log(1-\pi(s)k) + \log X_{\pi}(s) + ... \\
                =& \inf_s E_{t,x} \left[ V0 \right] \\
                =& A
       \end{split}
      \end{equation*}
    
\end{itemize}

\end{proof}

Wir schlie�en diesen Abschnitt mit der Aussage, dass eine optimale Strategie am Ende des Zeithorizonts kein Geld mehr in Aktien haben darf.
\begin{mySatz} [\cite{Korn2002}, Proposition 2.1 ]
Eine optimale Worst-Case Strategie $\pi_{\text{opt}}$ erf�llt
\begin{equation}
\pi_{\text{opt}}(T) \leq 0
\end{equation}
\end{mySatz}

\begin{proof}
Dazu  berechnen wir wie Worst-Case-Schranke am Ende des Zeithorizonts.
\begin{equation*}
\begin{split}
 &WC(\pi,T,x)  \\
=  & \min \left( 
                   \inf_{T \leq s \leq T} E_{T,x} \left[ V^0(s,X_{\pi}(s)(1-\pi(s)k)) \right] 
                  , 
                   E_{T,x} \left[ U(X_{\pi}(T)) \right] 
                  \right) \\
=  & \min \left( 
                  V^0(T,x(1-\pi(T)k)) 
                  , 
                   \log x 
                  \right) \\
=  & \min \left( 
                  \log x + \log(1-\pi(T)k) 
                  , 
                   \log x 
                  \right) \\
\end{split}
\end{equation*}
Handelsstrategien mit $\pi(T) > 0$ nehmen offenbar nicht die optimale Worst-Case-Schranke $\log x$ an. Also gilt die Behauptung.
\end{proof}

Aus den beiden vorangehenden S�tzen folgt also, dass wir uns bei der Suche nach optimalen Strategien auf Strategien mit positiven Aktienanteil, der am Ende des Zeithorizonts
verschwindet, beschr�nken k�nnen.



\section{Optimale Strategie f�r logaritmischen Nutzen}
In diesem Abschnitt wollen wir die optimale Strategie bez�glich des logaritmischen Nutzen angeben. Es wird sich herausstellen, dass sie sich als L�sung einer
Differentialgleichung ergibt. Zun�chst stellen wir einige Eigenschaften dieser Differentialgleichung bereit.
\begin{mySatz} \label{th:DGL}
Es gibt genau eine Funktion $\pidach: \left[0, T \right] \rightarrow \mathbf{R}$ die das Endwertproblem
\begin{align*}
\pi_t(t) & =\frac{1}{k}(1-\pi(t)k) 
\left( 
\pi(t)(\mu-r) - \einhalb 
  \left(
    \pi(t)^2 \sigma^2 + \bruch^2
  \right)
\right) \\
 \pi(T) &= 0
\end{align*}
l�st. Dabei gilt $\pidach(0) < \min(\pistar,1/k)$ und $\pidach$ ist streng monoton fallend.
\end{mySatz}
\begin{proof}
Mit dem Beweis folgen wir der Idee aus \cite{Korn2002}(Example 2.4.) und korrigieren die dort gegebenen fehlerhaften Rechnungen.

Wir formen die Differentialgleichung  zun�chst zu
\begin{equation*} 
\pi_t(t) =\frac{-\sigma^2}{2k}(1-\pi(t)k) \left( \pi(t)-\pistar \right)^2
\end{equation*}
um. Durch Trennen der Variablen erhalten wir
\begin{equation*} 
\frac{1}{(1-\pi k) \left( \pi-\pistar \right)^2 }d\pi =\frac{-\sigma^2}{2k} dt.
\end{equation*}
Durch Integartion gelangen wir nun zu
\begin{equation*} 
\frac{k}{(k \pistar -1)^2} \log \left( \frac{\pistar-\pi}{1-k\pi}\right) + \frac{1}{(k\pistar-1)(\pi-\pistar)} =\frac{-\sigma^2}{2k} t + C.
\end{equation*}
Dabei haben wir die linke Seite mit Hilfe eines Computeralgebrasystem integriert, durch (langwieriges) Ableiten und Termzusammenfassen kann das Ergebnis �berpr�ft werden.
Mit den Bezeichnungen
\begin{equation*} 
\alpha =\frac{k}{(k \pistar -1)^2}
\end{equation*}
und
\begin{equation*} 
\beta =\frac{1}{1-k \pistar }
\end{equation*}
k�nnen wir die erhaltene Gleichung als
\begin{equation} \label{eq:nichtlineareGleichung}
\alpha \log \left( \frac{\pistar-\pi}{1-k\pi}\right) + \frac{\beta}{(\pi-\pistar)} =\frac{-\sigma^2}{2k} t + C
\end{equation}
schreiben. Aus der Bedingung $\pi(T)=0$ folgt
\begin{equation*} 
C= \alpha \log \left( \pistar\right) - \frac{\beta}{\pistar} + \frac{\sigma^2}{2k} T.
\end{equation*}
F�r $t<T$ ist die rechte Seite von Gleichung \eqref{eq:nichtlineareGleichung} um $ \frac{\sigma^2}{2k}(T-t)$ gr��er als die linke Seite. Die Ableitung der linken Seite ist
\begin{equation*}
 \frac{1}{(1-\pi k) \left( \pi-\pistar \right)^2 },
\end{equation*}
(sie ist ja durch Integrieren dieses Termes entstanden), also ist sie f�r $\pi <\frac{1}{k}$ immer streng positiv. Die Gleichung \eqref{eq:nichtlineareGleichung} hat einen ersten
Pol bei $\min(\pi,1/k)$ mit dem Wert $\infty$, es gibt also genau einen Wert f�r $\pi$ der die Gleichung l�st. Das beweist die Existenz und Eindeutigkeit und die Behauptung
$\pidach(0)<\min(\pi,1/k)$. Das $\pidach$ eine monoton fallende Funktion ist folgt daraus, dass die rechte Seite von  Gleichung \eqref{eq:nichtlineareGleichung} in $t$ f�llt.
\end{proof}
Ein Bild von $\pidach$ f�r Beispielparameter findet man in Abbildung \ref{fig:optimaleStrategie}.
Verwendet man die Funktion $\pidach$ aus dem vorhergehendem Satz als Handelsstrategie, so h�lt man immer das Gleichgewicht zwischen dem erwarteten Nutzen nach einem sofortigen
Crash und dem erwarteten Nutzen, wenn �berhaupt kein Crash geschieht. Das ist die Aussage des folgenden Satzes.
\begin{mySatz}[Gleichgewichtsbedingung, \cite{Korn2002}, Corollary 2.2. (b)] \label{th:Gleichgewicht}
Es gilt:
\begin{equation} \label{eq:Gleichgewichtsbedingung}
V^0(t,x (1-\pidach(t)k), \log)= E_{t,x} \left[ \log(X_{\pidach}(T)) \right]
\end{equation}
\end{mySatz}

\begin{proof}
Da $\pidach$ die Differentialgleichung aus Satz \ref{th:DGL} erf�llt gilt:
\begin{equation*}
 \pidach_t(t) =\frac{1}{k}(1-\pidach(t)k) 
\left( 
\pidach(t)(\mu-r) - \einhalb 
  \left(
    \pidach(t)^2 \sigma^2 + \bruch^2
  \right)
\right)
\end{equation*}
Diese Gleichung formen wir zun�chst um:
\begin{equation*}
\frac{\pidach_t(t)}{(1-\pidach(t)k)} k =
\left( 
\pidach(t)(\mu-r) - \einhalb 
  \left(
    \pidach(t)^2 \sigma^2 + \bruch^2
  \right)
\right)
\end{equation*}
Integrieren von $t$ nach $T$ ergibt:
\begin{equation*}
\left[
- \log(1-\pidach(s) k)  
\right]^T_t
=
\int_t^T
\pidach(u)(\mu-r) - \einhalb 
  \left(
    \pidach(u)^2 \sigma^2 + \bruch^2
  \right)
du
\end{equation*}
Berechnung der linken Seite unter Beachtung von $\pidach(T)=0$ und Herausziehen des bez�glich $u$ konstanten Teil
des Integrals auf der rechten Seite ergibt
\begin{equation*}
\log(1-\pidach(t)k)
=
\int_t^T
\pidach(u)(\mu-r) - \einhalb 
  \left(
    \pidach(u)^2 \sigma^2 
  \right)
du
- \einhalb \bruch^2 (T-t)
\end{equation*}
vereinfachen.
Addieren wir  $\einhalb \bruch^2 (T-t) + \log x + r(T-t)$ zu dieser Gleichung, so erhalten wir
\begin{equation*} 
\begin{split}
 &\log(1-\pidach(t) k) + \log x + r(T-t) +  \einhalb \bruch^2 (T-t) \\
=& \log x + r(T-t)  + \int_t^T
\pidach(u)(\mu-r) - \einhalb 
  \left(
    \pidach(u)^2 \sigma^2
  \right)
du \\
\end{split}
\end{equation*}
Damit folgt aus der expliziten Form aus $V^0$ aus Satz \ref{th:LogWertfunktion} und der Formel f�r den erwarteten logaritmischen Nutzen aus Satz
\ref{th:ExpLogUtility} 
\begin{equation*}
V^0(t,x(1-\pidach(t)k)) = E_{t,x} \left[ \log(X_{\pidach}(T)) \right], 
\end{equation*}
wie behauptet.
\end{proof}
Wir bemerken noch, dass es auch legitim ist f�r $x$ die Zufallsvariable $X_{\pidach}(t)$ einzusetzen. Dann erh�lt man
\begin{equation*}
 V^0(t,X_{\pidach}(t) (1-\pidach(t)k)) = E(\log(X_{\pidach}(T)) | X_{\pidach}(t)). 
\end{equation*}
Nun formulieren und beweisen wir das Hauptergebnis dieses Kapitels

\begin{mySatz}[Optimale Strategie, \cite{Korn2002}, Theorem 2.3]
\label{th:hauptsatz_willmott}
Die L�sung $\pidach$ der Differentialgleichung aus Satz \ref{th:DGL} ist eine L�sung des Worst"=Case"=Portfoliooptimierungsproblems.
\end{mySatz}

\begin{proof}
Die Struktur dieses Beweises ist wie folgt. Wie betrachten in Fallunterscheidungen verschiedene
Handelsstrategien die sich bez�glich $\pidach$ wie folgt unterscheiden.
\begin{itemize}
 \item Handelstrategien mit einer h�herem erwarteten Nutzen im crashfreien Scenario ...
  \begin{itemize}
   \item ... und einem h�heren Aktienanteil zu Beginn.
   \item ... und einem niedrigeren Aktienanteil zu Beginn.
   \end{itemize}
 \item Handelsstrategien mit einem niedrigeren erwarteten Nutzen als im crashfreien Scenario.
\end{itemize}
F�r jeden Fall zeigen wir dann das $\pidach$ jeweils keine schlechtere Worstcaseschranke besitzt. Damit muss
$\pidach$ dann die optimale Stategie sein.

Nun zur Ausf�hrung unseres Planes. Sei $(t_0,x_0) \in \left[0,T \right] \times \mathbb{R}^+$ gegeben. Wir m�ssen dann zeigen, dass f�r alle $\pi \in \mathcal{A}$ 
\begin{equation*}
 WC(\pi,t_0,x_0) \leq WC(\pidach,t_0,x_0)
\end{equation*}
gilt. Wir gliedern diesen etwas l�ngeren Beweis nun in 5 Unterpunkte.
\begin{enumerate}
 \item 
Zun�chst betrachten wir eine Handesstrategie $\pi$ mit einer h�heren erwarteten Nutzen im crashfreien Scenario. 
Dann gilt (Erkl�rungen folgen)
\begin{equation*}
\begin{split}
   &\int_{t_0}^{T} f \left( E \left[ \pi(u) \right] \right)  du \\
 = &\int_{t_0}^T E \left[\pi(u)\right] (\mu -r) - \einhalb E \left[ \pi(u) \right] ^2 \sigma^2 du \\
 \geq &\int_{t_0}^T E \left[ \pi(u) \right] (\mu -r) - \einhalb E \left[ \pi(u)^2 \right] \sigma^2 du \\
 \geq  & E \left[ \int_{t_0}^{T} f(\pi(u)) du \right] \\
 >  & \int_{t_0}^{T}f(\pidach(u))du
\end{split}
\end{equation*}
Dabei haben wir bei der ersten Gleichung die Definition von $f$ benutzt, bei der ersten Ungleichung haben wir die f�r alle Zufallsvariablen $X$ g�ltige Beziehung
 $E\left[X^2 \right] \geq E\left[X\right]^2$ benutzt, in der zweiten Ungleichung haben wir Integral und Erwartungswert vertauscht und die letzte Ungleichung folgt aus der
expliziten Form des erwarteten logarithmischen Nutzens (Satz \ref{th:ExpLogUtility}) und der Voraussetzung des h�heren erwarteten Endnutzens von $\pi$ in diesem Fall. Dann folgt
die Existenz eines $s$ mit $t_0 \leq s \leq T$
mit
\begin{equation*}
 f(E(\pi(s))) > f(\pidach(s))
\end{equation*}
Da $\pidach \leq \pistar$ (Satz \ref{th:DGL}) und $f$ bis $\pistar$ streng monoton wachsend ist (Satz \ref{th:Eigenschaften_f}) folgt
\begin{equation} \label{forsomet}
 E(\pi(s)) > \pidach(s).
\end{equation}

\item
Wir definieren mit
\begin{equation}
 \vdach(t,x) := E \left[ \log X_{\pidach}(T) | X_{\pidach}(t) = x \right]
\end{equation}
 den erwarteten logaritmischen Nutzen durch Verfolgung der Handelsstrategie
$\pidach$. Wir geben ihn hier nochmal explizit an und berechnen die zur Anwendung der It\^{o}-Formel
ben�tigten partiellen Ableitungen:
\begin{align*}
\vdach(t,x) &= \log x  +  r(T-t) + \int_{t}^{T}  \pidach(u)(\mu -r)-\frac{1}{2}\pidach(u)^2\sigma^2du \\
\vdach_x(t,x) &= \frac{1}{x} \\
\vdach_{xx}(t,x) &= - \frac{1}{x^2} \\
\vdach_t(t,x) &= -\pidach(t)(\mu -r)+\frac{1}{2}\pidach(t)^2\sigma^2 -r
\end{align*}
Wir wiederholen hier auch nochmal die $X$ treibende Dynamik um im folgenden die Anwendung der It\^{o}-Formel
nachvollziehen zu k�nnen:
\begin{equation*}
 dX_{\pidach}(t) = X_{\pidach}(t) \left[ \left( r + \pi(t)(\mu -r) \right) dt + \pi(t) \sigma dW(t) \right]
\end{equation*}
Nun gehen wir davon aus dass $X(t_0)=x_0$ gilt (formal m�ssten wir in der folgende Gleichungskette jeden Ausdruch durch $E\left[\cdot | X(t_0)=x_0 \right]$ einschlie�en) und  wir
f�hren folgende Rechnung f�r ein $t \geq t_0$ durch (Erkl�rungen folgen)
\begin{equation*}
\begin{split}
 &V^0(t,X_{\pidach}(t)(1-\pidach(t)k)) \\
= & E \left[ \log ( X_{\pidach}(T)  | X_{\pidach}(t) \right] \\
= &\vdach(t,X_{\pidach}(t)) \\
= &\vdach(t_0,x_0) \\
+ &\int_{t_0}^t \vdach_x(s,X_{\pidach}(s)) X_{\pidach}(s) \pidach(s) \sigma dW(s) \\
+ &\int_{t_0}^t \vdach_t(s,X_{\pidach}(s)) +  \vdach_x(s,X_{\pidach}(s)) X_{\pidach}(s) (r+\pidach(s)(\mu-r)) \\
  &+ \einhalb \vdach_{xx}(s,X_{\pidach}(s))\pidach(s)^2 X_{\pidach}(s)^2 \sigma^2 ds
\end{split}
\end{equation*}

Dabei folgt die erste Gleichung aus Satz \ref{th:Gleichgewicht} (und der darauf folgenden Bemerkung), die zweite Gleichung ist die Definition von $\vdach$ und die letzte Gleichung
ist eine Anwendung der
It\^{o}-Formel (Satz \ref{th:Ito-Formel}).

Nehmen wir nun davon den Erwartungswert, so f�llt das Integral bez�glich der Brownschen Bewegung dank der Beschr�nktheit von $\pidach$ weg.
F�r den Integranten des Integrals bez�glich $ds$ berechnen wir unter Ausnutzung des zuvor f�r $\vdach$ berechneten Ableitungen
\begin{equation*}
\begin{split}
&\vdach_t(s,X_{\pidach}(s)) +  \vdach_x(s,X_{\pidach}(s)) X_{\pidach}(s) (r+\pidach(s)(\mu-r)) \\
  &+ \einhalb \vdach_{xx}(s,X_{\pidach}(s))\pidach(s)^2 X_{\pidach}(s)^2 \sigma^2 \\
 = &-\pidach(s)(\mu -r)+\frac{1}{2}\pidach(s)^2\sigma^2 -r + (r+\pidach(s)(\mu-r)) - \einhalb \pidach(s)^2   \sigma^2 \\
 =&0
\end{split}
\end{equation*}

Durch Anwenden des Erwartungswertes auf obige Gleichungsgruppe erhalten wir also f�r alle $t$ mit $t_0\leq t \leq T$
\begin{equation*}
E_{t_0,x_0}(V^0(t,X_{\pidach}(t)(1-\pidach(t)k))) = \vdach(t_0,x_0)  
\end{equation*}
Dabei ist $\vdach(t_0,x_0)$ der erwartete logarithmische nutzen im crashfreien Scenario bei Verfolgen der Handelsstrategie $\pidach$. Es herrscht also eine Indifferenz gegen�ber
den verschieden Scenarien. Ein Crash zu einem beliebigen Zeitpunkt als auch gar kein Crash f�hren stets zum gleichen erwarteten Endnutzen.
Also gilt f�r alle $t$ mit $t_0\leq t \leq T$:
\begin{equation} \label{worstcaseboundpidach}
  WC(\pidach,t_0,x_0) = E_{t_0,x_0}(V^0(t,X_{\pidach}(t)(1-\pidach(t)k))) =\vdach(t_0,x_0) 
\end{equation}

\item
Wir splitten nun unsere Fallunterscheidung weiter auf und betrachten eine Strategie $\pi$ mit $\pi(t_0) \geq \pidach(t_0)$.  Dann gilt:
\begin{align*}
 WC(\pi,t_0,x_0) 
         &\leq V^0(t_0,x_0 (1-\pi(t_0)k)) \\
         &\leq  V^0(t_0,x_0 (1-\pidach(t_0)k)) \\
         &= WC(\pidach,t_0,x_0)
\end{align*}
Dabei folgt die erste Ungleichung daraus, das die Worst-Case-Schranke kleiner als der erwartete Nutzen nach einem sofortigen Crash sein muss, die zweite Ungleichung folgt aus der
Monotonie von $V^0$ im zweiten Argument und die letzte Gleichung folgt aus den zuvor gezeigten Aussagen �ber $WC(\pidach,t_0,x_0)$ (Gleichung \eqref{worstcaseboundpidach}). 

\item
Als n�chsten Fall betrachten wir Strategien mit $\pi(t_0) < \pidach(t_0)$.
Wir betrachten dann
\begin{equation}
\tau = \inf \{t; E \left[ \pi(t) \right] > \pidach(t) \}
\end{equation}
Wegen Gleichung \eqref{forsomet} und den Voraussetzungen des gerade betrachteten Falles liegt dieses $\tau$ zwischen 0 und $T$.

Offensichtlich gilt dann auch:
\begin{align*}
 & E_{t_0,x_0} \left[ \log (X_{\pi}(\tau)) \right] \\
 &= E \left[ \int_{t_0}^{\tau} f(\pi(u)) du  \right] + r(\tau-t_0) + \log x_0 \\
 &= \int_{t_0}^{\tau} E \left[  f(\pi(u))  \right] du + r(\tau-t_0) + \log x_0 \\
 &\leq \int_{t_0}^{\tau} f(E \left[ \pi(u)  \right]) du + r(\tau-t_0) + \log x_0 \\
 &\leq \int_{t_0}^{\tau} f(\pidach) du + r(\tau-t_0) + \log x_0 \\
 &=  E_{t_0,x_0} \left[ \log(X_{\pidach}(\tau)) \right]
\end{align*}
Dabei folgt die erste Gleichung aus Satz \ref{th:ExpLogUtility}, die zweite folgt durch Vertauschen von Integral und Erwartungswert, die folgende Ungleichung folgt aus der
Konkavit�t von $f$ und der Jensenschen Ungleichung, die n�chste Ungleichung folgt aus der Definition von $\tau$ und die letzte Gleichung folgt wieder aus Satz
\ref{th:ExpLogUtility}. Wir halten als Ergebnis dieser Rechnung
\begin{equation} \label{damdam}
 E \left[ \log (X_{\pi}(\tau)) \right]  \leq E \left[ \log(X_{\pidach}(\tau)) \right]
\end{equation}
fest. Zum Zeitpunkt $\tau$ gilt wegen der Jensenschen Ungleichung und der Definition von $\tau$
\begin{equation} \label{dumdum}
 E \left[ \log(1-\pi(\tau)k) \right] \leq \log(1-k E(\pi(\tau))) \leq \log(1-k \pidach(\tau)) 
\end{equation}
Nun haben wir alles zusammen, um die Worst-Case-Schranke von $\pi$ nach oben abzusch�tzen(Erkl�rungen folgen): 
\begin{align*}
 WC(\pi,t_0,x_0) &\leq  E \left[ V^0(\tau,X_{\pi}(\tau) (1-\pi(\tau)k)) \right] \\
          &= E \left[ \log(X_{\pi}(\tau)) \right]  + E \left[ \log (1-\pi(\tau)k) \right] + f^0(\tau) \\ 
          &\leq E \left[ \log(X_{\pi}(\tau)) \right] + \log(1-k \pidach(\tau)) + f^0(\tau) \\ 
          &\leq E(\log(X_{\pidach}(\tau))) + \log(1-k \pidach(\tau)) + f^0(\tau) \\
          &= E(V^0(\tau,X_{\pidach}(\tau)(1-\pi(\tau)k))) \\
          &= WC(\pidach,t_0,x_0) 
\end{align*}
Die erste Absch�tzung folgt aus der Tatsache, dass die \wcs kleiner als der erwartete Nutzen bei einem Crash zur Zeit $\tau$ ist. Die darauf folgende Gleichung ist
die explizite Form von $V^0$ aus Satz \ref{th:LogWertfunktion}. Die n�chste Ungleichung folgt dann aus \eqref{dumdum} und
die folgende Absch�tzung aus
\eqref{damdam}. Dann verwenden wir wieder die explizite Form von $V^0$ und schliesslich die bekannte Worst-Case-Schranke von $\pidach$. Damit haben wir nun auch f�r diesen Fall
gezeigt, dass $\pi$ keine bessere Worst-Case-Schranke haben kann.

\item
Es bleibt noch der Fall, wo $\pi$ im crashfreien Scenario einen kleineren erwarteten Nutzen besitzt. Dann gilt aber:
\begin{equation*}
 WC(\pi,t_0,x_0) \leq  E_{t_0,x_0} \left[ \log(X_{\pi}(T)) \right]  \leq E_{t_0,x_0} \left[ \log(X_{\pidach}(T)) \right] = WC(\pidach,t_0,x_0)  
\end{equation*}
Dabei folgt die erste Ungleichung aus der Definition der Worst-Case-Schranke, die zweite aus der Annahme f�r diesen Fall und die letzte Gleichung ist die bekannte
Worst-Case-Schranke f�r $\pidach$. Damit sind alle F�lle behandelt und wir haben gezeigt dass jede beliebige Handelsstrategie $\pi$ keine bessere Worst-Case-Schranke als $\pidach$
besitzen kann.
\end{enumerate}
\end{proof}


Wir haben noch nicht gezeigt, dass $\pidach$ wirklich eine bessere Worst-Case Schranke besitzt als die
Bondstrategie. Dass tun wir nun, indem wir eine Strategie angeben, die eine h�here Worst-Case-Schranke als die
pure Bondstrategie hat. Da $\pidach$ optimal ist, muss es auch eine bessere Worst-Case-Schranke besitzen.

\begin{mySatz}[\cite{Korn2002}, Corollary 2.2. (a)]
F�r das Worst"=Case"=Portfoliooptimierungsproblem gibt es eine bessere Strategie als die reine Bondstrategie.
\end{mySatz}
\begin{proof}
\newcommand{\besser}{\frac{ 1-\exp \left( -\einhalb \bruch ^2(T-t) \right) }{k}  }
 Wir betrachten die Strategie
\begin{equation}
 \pi(t)=\einhalb \min\left( \besser,\pistern \right)
\end{equation}
Also gilt:
\begin{equation*}
 0 < \pi(t) \leq \pistern
\end{equation*}
Also folgt $f(\pi(t))>0$ (f�r $0\leq s<T$) und also auch 
\begin{equation} \label{eq:besserohnecrash}
 0 < E \left[ 
	      \int_t^s f(\pi(u)) du
       \right]
\end{equation}
f�r (f�r $0\leq t< s<T$). Diesen Wert sollte man sich als den Aktienvorteil vorstellen, das ist der zus�tzliche erwartete Nutzen im crashfreien Scenario, den man aus einem
Aktieninvestment gegen�ber einem reinen Bondinvestment zieht.
Weiter gilt:

\begin{equation*}
\begin{split}
 & V^0(t,x(1-\pi(t)k), \log) \\
 = &\log x  + \log \left[ 1-\pi(t)k \right] + \letzterTermv \\
 > &\log x + \log \left[ 1- \left(1-\exp \left( -\einhalb \bruch^2 (T-t) \right) \right) \right] \\
   &+ \letzterTermv \\
 = &\log x -\einhalb \bruch^2 (T-t) + \letzterTermv \\
 = &\log x  + r(T-t) \\
\end{split}
\end{equation*}
Dabei ergibt sich die Ungleichung, da $\log(1-x k)$ streng fallend in $x$ ist und
\begin{equation*}
 \pi(t) < 2 \pi(t) \leq \besser
\end{equation*}
Also gilt:
\begin{equation} \label{eq:besserbeicrash}
 V^0(t,x(1-\pi(t)) k, \log)
 > \log x + r(T-t) 
\end{equation}
Das bedeutet, dass ein sofortiger Crash immer noch besser ist als die reine Bondstrategie. Aber auch ein Crash zu einen sp�teren Zeitpunkt ist besser, wie wir im folgenden
berechnen werden
\begin{align*}
 & \inf_{t \leq s \leq T} E_{t,x} \left[ V^0(s,X_{\pi}(s)(1-\pi(s)k)) \right] \\
 &> \inf_{t \leq s \leq T} E_{t,x} \left[ \log(X_{\pi}(s)) + r(T-s) \right] \\
 &= \inf_{t \leq s \leq T} \log x + \int_t^s  f(\pi(u)) du + r(s-t) + r(T-s)  \\
 &=  \log x + r(T-t) 
\end{align*}
Dabei folgt die erste Ungleichung aus der gerade gezeigten Beziehung \eqref{eq:besserbeicrash}, in der folgenden Gleichung haben wir $X_{\pi}$ explizit hingeschrieben und in der
Ungleichung darauf haben wir die Positivit�t des Aktienvorteil aus \eqref{eq:besserohnecrash} ausgenutzt.
Andererseits gilt f�r das crashfreie Scenario:
\begin{align*}
 & E_{t,x} \left[ \log (X_{\pi}(T)) \right] \\
 &= \log x +  E \left[ 
	      \int_t^T f(\pi(s)) ds
       \right] + r(T-t)\\
 &> \log x + r(T-t)\\
\end{align*}
Dabei haben wir in der ersten Gleichung die explizite Form des logarithmischen Nutzens von $X_{\pi}$ hingeschrieben und bei der Ungleichung den Aktienvorteil
\eqref{eq:besserohnecrash} ausgenutzt.
Insgesamt folgt also aus den beiden vorrangegangen Rechnungen:
\begin{equation*}
 WC(\pi,t,x) > \log x + r (T-t) = WC(\pi_0,t,x)
\end{equation*}
Danmit ist $\pi$ tats�chliche eine bessere Strategie als die reine Bondstrategie.
\end{proof}


\section{Optimale konstante Strategie f�r logarithmischen Nutzen}
Wir geben zus�tzlich noch die beste konstante Strategie f�r das Worst-Case-Portfoliooptimierungsproblem an. Damit ist dann eine Strategie gemeint, die bis zum Crash konstant ist
und danach in die beste Strategie des crashfreien Scenarios wecheslt.
\begin{mySatz}[\cite{Korn2002}, Proposition 2.5]
Die Strategie 
\begin{equation}
 \pi= 
 \left(
  \einhalb  \left( \pistar + \frac{1}{k} \right) - \sqrt{ 
   \frac{1}{4}  \left( \pistar - \frac{1}{k} \right)^2 + \frac{1}{\sigma^2T}
   } 
\right)^+
\end{equation}
ist der beste konstante Portfolioprozess f�r das Worst-Case-Portfoliooptimierungsproblem ist.
Weiter gilt: $\pi \rightarrow \min \left( \pistar,\frac{1}{k} \right)$ f�r $T  \rightarrow \infty$.
\end{mySatz}

\begin{proof}
Zun�chst zeigen wir die Grenzwertaussage. Setzen wir $T$ formal $\infty$, so nimmt $\pi$ die folgende Gestalt
\begin{equation*}
 \pi= 
 \left(
  \einhalb  \left( \pistar + \frac{1}{k} \right) - 
   \einhalb  \left| \pistar - \frac{1}{k} \right|
\right)^+ = \min \left( \pistar,\frac{1}{k} \right)
\end{equation*}
an. Wenn wir $T$ also hinreichend gro� w�hlen, kommen wir diesem Wert beliebig nahe. Damit ist die Grenwertaussage gezeigt.

Nun zur eigentlichen Aussage: Wir geben zun�chst die Worst-Case-Schranke f�r einen konstanten
Portfolioprozess $\pi_c$ an. Das Worst-Case-Scenario ist eine Crash unmittelbar vor dem Zeithorizont. Intuitiv kann man sich das klarmachen, idem man sich �berlegt, das ein Crash
auf jeden Fall mehr schadet als kein Crash. Wegen des konstanten Portfolioprozess verliert man zu jedem Zeitpunkt den gleichen Anteil am Verm�gen, je sp�ter aber der Crash
geschieht, um so weniger Zeit hat man danach sich optimal zu verhalten. Damit ergibt sich die Worst-Case-Schranke:
\begin{equation} \label{eq:WorstCaseKonstanteStrategie}
  WC(\pi_c) = \log x  + rT + \left( \pi_c(\mu-r)-\einhalb \pi_c^2 \sigma^2 \right) T + \log(1-\pi_c k)
\end{equation}
Dieses intiutiv gewonnen Ergebnis rechnen wir nun nochmal nach:
\begin{equation*}
\begin{split}
   &E \left[ V^0(s,X_{\pi,x}(s)(1-\pi k)) \right] \\
 =& E \left[ \log(X_{\pi,x}(s)) + \log (1-\pi k)  + \left( r+ \einhalb \bruch^2(T-s) \right) \right] \\
 =& \log x + \left( r+ f (\pi) \right)s + \log(1-\pistar k)  + \left( r+ \einhalb \bruch^2(T-s) \right)  \\
 =& \log x + \left( r+ (\mu-r)(\pi) + \einhalb \bruch^2 \right)s + \log(1-\pistar k)  \\
 +& \left( r+ \einhalb \bruch^2(T-s) \right)  \\
 =& \log x + \left( r+ \einhalb \bruch^2 \right)(T-t) + \log(1-\pistar k)  \\
\end{split}
\end{equation*}

Wir wissen bereits aus Satz \ref{th:Unn�tzeLeerverk�ufe} dass wir uns bei der Suche nach optimalen Handelsstrategien auf solche besch�nken k�nnen, die nach unten durch $0$
beschr�nkt sind. Wir suchen also dasjenige $\pi$ aus $[0,1/k)$ das die Worst-Case-Schranke \eqref{eq:WorstCaseKonstanteStrategie} maximiert, f�r das also
\begin{align*}
  g(\pi) &:= \left( \pi(\mu-r)-\einhalb \pi^2 \sigma^2 \right) T + \log(1-\pi k) \\
         &= f(\pi)T  + \log(1-\pi k)
\end{align*}
maximal wird.
 
Ein Blick auf den Graphen von $f$ und $\log (1 - (\cdot) k)$ (Figur \ref{fig:graph_von_f_und_log})  macht sofort deren Konkavit�t�t klar, dewegen ist auch $g$ konkav.

\begin{figure}
 \begin{center}
  \input{GNUPlotprogramme/plotflog.tex}
 \end{center}
\caption{Graphen von f und log}
\label{fig:graph_von_f_und_log}
\end{figure}


 Sowohl f�r $\pi$ gegen $-\infty$ als auch f�r
$\pi$ gegen $1/k$ geht $g$ gegen $-\infty$. Dazwischen muss es also genau ein Maximum geben.

Ableiten nach $\pi$ ergibt:
\begin{equation*}
  \left( (\mu-r)- \pi \sigma^2 \right) T + \frac{-k}{1-\pi k}
\end{equation*}
Gleichsetzen mit 0 und Multiplikation mit $1-\pi k$ ergibt
\begin{equation*}
   (\mu-r)T - (\mu-r) T \pi k - \pi \sigma^2 T + \pi^2 \sigma^2 T k - k = 0
\end{equation*}

Nun sortieren wir anch den Koeffizienten von $\pi$.
\begin{equation*}
  \pi \sigma^2 T k -\pi T (( \mu-r) k + \sigma^2) + (\mu-r)T-k = 0
\end{equation*}

Und normieren noch:
\begin{equation*}
  \pi^2  - \pi \left( \frac{\mu-r}{\sigma^2} + \frac{1}{k} \right) + \frac{(\mu-r)}{\sigma^2 k} - \frac{1}{\sigma^2T} = 0
\end{equation*}

Mit der L�sungsformel f�r quadratischen Gleichungen in der Form $-\frac{p}{2} \pm \sqrt{ \frac{p^2}{4}-q}$ erhalten wir:
\newcommand{\meinp}{-\left( \frac{\mu-r}{\sigma^2} + \frac{1}{k} \right)}
\newcommand{\meinq}{\frac{(\mu-r)}{\sigma^2 k} - \frac{1}{\sigma^2T}}


\begin{equation*}
\begin{split}
  & \pi_{1,2} \\
 =& - \frac{\meinp}{2} \pm \sqrt{ \frac{(\meinp)^2}{4}- \left( \meinq \right) } \\
 =& \einhalb \left( \frac{\mu-r}{\sigma^2} + \frac{1}{k} \right) \pm \sqrt{\frac{(\pistern)^2}{4} 
                                           +  \frac{\pistern}{k 2} 
                                           + \frac{1}{4 k}
                                           -  \frac{\pistern}{k} + \frac{1}{\sigma^2T} } \\
  =& \einhalb \left( \frac{\mu-r}{\sigma^2} + \frac{1}{k} \right) \pm \sqrt{\frac{(\pistern)^2}{4} 
                                           -  \frac{\pistern}{k 2} 
                                           + \frac{1}{4 k}
                                           + \frac{1}{\sigma^2T} } \\
 =& \einhalb \left( \frac{\mu-r}{\sigma^2} + \frac{1}{k} \right) \pm
     \sqrt{ \frac{1}{4} \left( \frac{\mu-r}{\sigma^2} - \frac{1}{k} \right)^2  + \frac{1}{\sigma^2T} } 
\end{split}
\end{equation*}
Wir zeigen nun dass die weiter rechts liegende L�sung $>1/k$ ist. Ist n�mlich $\pistern \leq 1/k$, so folgt:
\begin{equation}
 \begin{split}
  &\einhalb \left( \frac{\mu-r}{\sigma^2} + \frac{1}{k} \right) +
     \sqrt{ \frac{1}{4} \left( \frac{\mu-r}{\sigma^2} - \frac{1}{k} \right)^2  + \frac{1}{\sigma^2T} }  \\
 \geq &\einhalb \left( \frac{\mu-r}{\sigma^2} + \frac{1}{k} \right) +
      \sqrt{ \frac{1}{4} \left( \frac{\mu-r}{\sigma^2} - \frac{1}{k} \right)^2} \\
 = &\einhalb \left( \frac{\mu-r}{\sigma^2} + \frac{1}{k} \right) +
      \frac{1}{2} \left( \frac{1}{k} - \frac{\mu-r}{\sigma^2} \right)\\
 = &  1/k
 \end{split}
\end{equation}

Ist andererseits $1/k \leq \pistern$ so folgt
\begin{equation}
 \begin{split}
  &\einhalb \left( \frac{\mu-r}{\sigma^2} + \frac{1}{k} \right) +
     \sqrt{ \frac{1}{4} \left( \frac{\mu-r}{\sigma^2} - \frac{1}{k} \right)^2  + \frac{1}{\sigma^2T} }  \\
 \geq &\einhalb \left( \frac{\mu-r}{\sigma^2} + \frac{1}{k} \right)  \geq \einhalb \left( \frac{1}{k} + \frac{1}{k} \right) = \frac{1}{k} \\
\end{split}
\end{equation}

Im Intervall $(-\infty,1/k)$ kann also nur in $\pi_1$ ein Maximum liegen, und da in diesem Intervall ein Maximum liegen muss, ist $\pi_1$ auch eine Maximumsstelle. Da die Funktion
konkav ist, ist sie rechts vom Maximum monoton fallend, das Maximum in  $[0,1/k)$ wird also entweder, wenn $\pi_1 \geq 0$, in $\pi_1$ angenommen, oder in $0$. Damit ist die
Behauptung bewiesen. 
\end{proof}

Wir wollen nochmal mit einer Grafik die zwei qualitativ unterschiedlichen F�lle betracten wo einmal die pure Bondstrategie die beste konstante Strategie ist und einmal die beste
konstante Strategie echt positiv ist.
Dazu verwenden wir jeweils die Parameter $\mu=0.2, r=0.05, \sigma = 0.4$ und $k=0.2$ und einmal mit $T=1$ und einmal mit $T=10$. Wir plotten die Worst-Case-Schranke aus Gleichung
\eqref{eq:WorstCaseKonstanteStrategie}.

\begin{figure}
 \begin{center}
  \input{GNUPlotprogramme/plotconstantwcb.tex}
 \end{center}
\caption{Worst-Case-Schranken konstanter Strategien}
\label{fig:vergleich_konstanter_strategien}
\end{figure}



\chapter{Portfoliooptimierung unter Crashgefahr: Der HJB-Ansatz}
\label{chap:Steffensen}
\section{Portfoliooptimierung unter Crashgefahr: Zweite Formulierung}
Wir werden im Folgenden einen anderen Ansatz als im vorherigen Kapitel besprechen, um ein optimales Portfolios unter Crashgefahr zu finden. Dieser Ansatz wurde zum ersten
Mal in \cite{Korn2007} diskutiert. Da die Modellierung dabei ganz anders ist als im vorherigen Kapitel, m�ssen wir das Portfolioproblem nochmal neu formulieren. Im letzten
Abschnitt dieses Kapitels werden wir die beiden Ans�tze vergleichen.

Wir gehen wieder von unserem Standardfinanzmarktmodell aus Definition \ref{Standardfinanzmarktmodell} aus. Wir betrachten dann zus�tzlich noch Sprungprozesse $N$, die auf dem
Intervall
$\left[0,T  \right]$ genau einen Sprung von 0 auf 1 machen. Damit modellieren wir die M�glichkeit des Aktienkurses zu crashen. Eine Aktienkurs stellen wir uns dann als einen
Prozess mit der folgenden Dynamik vor:
\begin{align*}
 dS(t) &= S(t-)(\mu dt + \sigma dW(t) - k dN(t)) \\
  S(0) &= s_0.
\end{align*}
Den Nutzen aus dem Endverm�gen zum Zeitpunkt $T$ werden wir wieder mittels eines MaxMin-Ansatzes maximieren. Das Infimum soll dabei aber nicht wieder, wie im letzten
Kapitel, �ber alle m�glichen Crashzeitpunkte, mathematisch betrachtet also reelle Zahlen, sonder �ber alle Crashprozesse genommen werden. Die benennen wir in der folgenden
Definitionen pr�zise.

\begin{myDefinition} 
\standardsatz
Die Menge der Crashprozesse $\mathcal{B}$ sind alle auf $\Omega$ definierten Prozesse $N$, die an die Filtrierung der Brownschen Bewegung adaptiert sind, f�r die $N(0)=0$ und
$N(T)=1$ gilt, und die, bis auf genau einen Sprung, konstant sind. 
\end{myDefinition}

Crashprozesse sind also als an die Brownsche Filtrierung adaptierte Prozesse definiert. Damit sind insbesondere alle konstanten Crashzeiten eingeschlossen, aber auch solche, die
auf eine (an die Brownsche Filtrierung adaptierte) Handelsstrategie reagieren. Bei unserem MinMax-Ansatz suchen wir eine Strategie, bei der der minimale erwartete Nutzen unter
allen m�glichen Crashzeiten maximiert wird. Unter diesem Gesichtspunkt ist unsere Menge von Crashzeiten also schon relativ gro� und wir verzichten auf die Ber�cksichtigung
von Crashzeiten, die ganz anderen Zufallsprinzipien folgen.

Jeder Crashprozess $N$ definiert eine Stoppzeit $\tau = \inf \{ t; N(t)=1 \}$ und jede Stoppzeit $\tau$ definiert einen Crashprozess $\mathbf{1}_{\left[\tau,T \right]}$ Wir werden
daher $\tau$ und $N$ im folgenden austauschbar benutzen.

\begin{myDefinition} 
\standardsatz
Wir bezeichnen mit $X_{\pi,\tau}$ die L�sung der von $\tau$ und $\pi$ kontrollierten Verm�gensgleichung.
\begin{align*}
 dX(t) &= X(t-)(r dt + \pi(t)(\mu -r) dt + \sigma dW(t) -k dN(t)) \\
  X(0) &= x_0
\end{align*}
\end{myDefinition}
Mit unserem Wissen �ber die L�sung dieses Differentialgleichung ohne Sprung k�nnen wir auch die L�sung dieser Differentialgleichung ganz konkret angeben. Es gilt n�mlich:
\begin{equation*}
 X_{\pi,\tau}(t) = X_{\pi, \tau}(t) \left( 1 - \Indikatorfunktion_{\left[\tau,T\right]} \pi(\tau) k \right)
\end{equation*} 
Der Prozess  $X_{\pi,\tau}$ folgt also im wesentlichen der Dynamik im crasfreien Scenario; vor dem Crash sind sie gleich, nach dem Crash unterscheiden sie sich um einen
Faktor. Insbesondere gilt:
\begin{equation*}
 X_{\pi, \tau}(\tau) = X_{\pi}(\tau) \left( 1 - \pi(\tau) k \right) = X_{\pi, \tau}(\tau-) \left( 1 - \pi(\tau) k \right)
\end{equation*} 

In der folgenden Definition geben wir noch einem oft ben�tigten Operator einen Namen.
\begin{myDefinition} 
\standardsatz
Sei $v \in C^{1,2}$ und $\pi \in \reelleZahlen$. Dann sei der Operator  $\mathcal{L}^{\pi} v$ durch 
\begin{equation} \label{eq:Generator}
 \mathcal{L}^{\pi} v(t,x) = v_t(t,x) + v_x(t,x)(r+\pi(\mu-r)) x + \einhalb v_{xx}(t,x) \pi^2 \sigma^2 x^2
\end{equation}
definiert.
\end{myDefinition}

Nun geben wir die dieser Modellierung angepassten Formulierung des Worst-Case-Portfoliooptimierungsproblem.


\begin{myDefinition} \label{Worst-Case-portfoliooptimerungsproblem2}
\standardsatz
Sei $N \in \mathcal{B}$ und $U$ eine Nutzenfunktion. Wir bezeichnen mit 
\begin{equation*} 
\begin{split}
 J^1(x,t,\pi,N) & := E_{t,x,1} \left[ U(X_{\pi,N}(T)) \right] \\
                & := E \left[ U(X_{\pi,N}(T)) | X(t)=x,N(t)=0 \right] 
\end{split}
\end{equation*}
den erwarteten Endnutzen durch Verfolgung von $\pi$ unter dem Crashscenario $N$. 
Wir nennen die Aufgabe
\begin{equation*}
  \text{maximiere}  \inf_{N \in \mathcal{B}} J^1(x_0,0,\pi,N)
\end{equation*}
das \textbf{Worst-Case-Portfoliooptimierunsproblem \RM{2}}. Die Funktion
\begin{equation*}
 V^1(t,x)= \sup_{\pi \in \mathcal{A}} \inf_{N \in \mathcal{B}} J^1(t,x,\pi,N)
\end{equation*}
hei�t Wertfunktion des Problems und eine Handelsstrategie f�r die das Surpremum angenommen wird hei�t
optimale Strategie f�r $(t,x)$.
\end{myDefinition}

Schon in der Notation erkennt man die Analogie zum Standard"=Portfoliooptimierungsproblem aus Definition \ref{def:Standardportfoliooptimierungsproblem}. Daf�r kennen wir schon 
einen Verifikationssatz f�r die Wertfunktion und die optimale Strategie. Im Folgenden wird es unser Ziel sein einen analogen Verifikationssatz f�r diese Erweiterung zu formulieren
und zu beweisen. Dieses Problem hat einen anderen Namen bekommen als das \emph{Worst-Case-Portfoliooptimierungsproblem \RM{1}} aus Definition \ref{def:WCPortfolioproblem1}. Zwar
wurde genau die gleiche Idee formuliert, aber die Modellierung ist ganz unterschiedlich. F�r den logarithmischen Nutzen werden wir dennoch zeigen, dass beide Formulierungen zur
gleichen optimalen Strategie f�hren.

Wir wollen noch ein paar Worte dar�ber verlieren, wie der Ausdruck 
\begin{equation*}
 V^1(t,x)= \sup_{\pi \in \mathcal{A}} \inf_{N \in \mathcal{B}} J^1(t,x,\pi,N)
\end{equation*}
zu verstehen ist. Als erste Idee w�rde man ihn wahrscheinlich als 
\begin{equation*}
 V^1(t,x)= \sup \{  \inf \{ J^1(t,x,\pi,N) : N \in \mathcal{B} \}  : {\pi \in \mathcal{A}} \}
\end{equation*}
interpretieren. Mit dieser Intepretation bezeichnet man das Maximierungsproblem als statisches Spiel, wo der eine Spieler erst ein $\pi \in \mathcal{A}$ ausw�hlt, und der andere
Spieler schliesslich ein $N \in \mathcal{B}$ ausw�hlt. Der erste Spieler kann sich also nicht bei der Wahl von $\pi$ auf das vom anderen Spieler zu w�hlende $N$ beziehen. Mit
solch einer Intepretation w�rden wir jedoch nicht zu unseren Ergebnissen kommen. Wir verstehen das Maximierungsproblem als Differentialspiel, dabei d�rfen die Spieler die Werte
der zu w�hlenden Prozesse abh�ngig von den Werten des jeweiligen anderen Prozesses aus der Vergangenheit machen. Wir werden den komplizierten Formalismus der Differentialspiele
in dieser Arbeit nicht ausf�hren, der Leser sollte sich dessen aber bewu�t sein und wir werden die zus�tzlichen M�glichkeiten eines Differentialspiels gegen�ber eines statischen
Spiel im folgenden naiv benutzen.


Zun�chst beweisen wir nun ein Lemma, dass verschieden M�glichkeiten angibt wie die Wertfunktion $V^1$ dargestellt werden kann.

\begin{mySatz} \label{th:Bellmanprinzip}
F�r die Wertfunktion $V^1$ des Worst-Case-Portfoliooptimierungspro"-blem \RM{2} gilt:
 \begin{align*}
 V^1(t,x) &=  \sup_{\pi \in \mathcal{A}} \inf_{N \in \mathcal{B}}    E_{t,x,1} \left[ U(X_{\pi, \tau}(T)) \right] \\
          &=  \inf_{N \in \mathcal{B}}   \sup_{\pi \in \mathcal{A}}  E_{t,x,1}  \left[ U(X_{\pi, \tau}(T)) \right] \\
%          &=  \sup_{\pi \in \mathcal{A}} \inf_{\tau}  E_{t,x,n}  \left[  V^0(\tau,X^{\pi \tau}(\tau-)(1-k \pi(\tau)))  \right]\\
%          &=  \inf_{\tau}   \sup_{\pi \in \mathcal{A}}  E_{t,x,n}  \left[  V^0(\tau,X^{\pi \tau}(\tau-)(1-k \pi(\tau)))  \right]\\
%          &=  \sup_{\pi \in \mathcal{A}} \inf_{\tau}  E_{t,x,n}  \left[  V^0(\tau,X^{\pi \tau}(\tau))  \right]\\
          &=  \inf_{\tau}   \sup_{\pi \in \mathcal{A}}  E_{t,x,1}  \left[  V^0(\tau,X_{\pi, \tau}(\tau))  \right].\\
\end{align*}
\end{mySatz}
In der Srache der Spieltheorie besagen die ersten beiden Gleichungen, dass das Spiel einen Sattelpunkt besitzt, die letzte Gleichung k�nnte man als ein Bellmanprinzip bezeichnen.



\begin{proof}
Sei ein beliebiges $\tau$ gegeben. Mit $\pi_1 \odot \pi_2$ bezeichnen wir dann eine Handelsstrategie, die bis $\tau$ sich wie $\pi_1$ verh�lt, danach wie $\pi_2$. Wegen der
$\epsilon$-Charakterisierung des Surpremums k�nnen wir ein $\pi_2'$ mit
\begin{equation} \label{eq:pistrichchar}
\begin{split}
       & \sup_{\pi_2} E_{\tau,X_{\pi_1,\tau}(\tau),0} \left[ U(X_{\pi_1 \odot \pi_2,\tau}(T)) \right] \\
 \leq &E_{\tau,X_{\pi_1,\tau}(\tau),0} \left[ U(X_{\pi_1 \odot \pi_2',\tau}(T)) \right]  + \frac{\epsilon}{2} 
\end{split}
\end{equation}
w�hlen. Es ist also $\pi_2'$ das fast optimale Verhalten nach den Crash.
Weiterhin k�nnen wir wegen der $\epsilon$-Charakterisierung des Infimums f�r einen beliebigen Portfolioprozess $\pi$ ein $\tau'$ mit
\begin{equation} \label{eq:taustrichchar}
\begin{split} 
     & \inf_{\tau}    E_{t,x,1} \left[ V^0(\tau,X_{\pi, \tau}(\tau)) \right] \\
\geq &  E_{t,x,1} \left[  V^0(\tau',X_{\pi, \tau'}(\tau'))  \right] - \frac{\epsilon}{2}
\end{split}
\end{equation}
w�hlen.
% Halte ich ebenfalls f�r �berfl�ssig	
% \begin{equation}
%  \inf_{N}    E \left[ U(X^{\pi,N}(T)) \right] 
% \geq  E \left[ V^0(\tau,X(\tau-)(1-k \pi(\tau))) \right] - \frac{\epsilon}{4}
% \end{equation}
Sei nun $\pi_1$ eine beliebige zul�ssige Handelsstrategie bis zum Crash $\tau$. Wir betrachten die folgende Ungleichungskette (Erkl�rungen folgen). 
\begin{equation*}
\begin{split}
      &   \sup_{\pi} \inf_{\tau}    E_{t,x,1} \left[ U(X_{\pi,\tau}(T)) \right] \\
\geq  &              \inf_{\tau}    E_{t,x,1} \left[ U(X_{\pi_1 \odot \pi_2',\tau}(T)) \right] \\
\geq  &              \inf_{\tau}    E_{t,x,1} \left[ E_{\tau,X_{\pi_1,\tau}(\tau),0} \left[  U(X_{\pi_1 \odot \pi_2',\tau}(T))
\right] \right] \\
\geq  &              \inf_{\tau}    E_{t,x,1} \left[ \sup_{\pi_2} E_{\tau,X_{\pi_1,\tau}(\tau),0} \left[  U(X_{\pi_1 \odot \pi_2,\tau}(T)) \right] \right] 
- \frac{\epsilon}{2} \\
\geq  &              \inf_{\tau}    E_{t,x,1} \left[ V^0(\tau,X_{\pi_1,\tau}(\tau)) \right]  - \frac{\epsilon}{2} \\
\geq  &                          E_{t,x,1} \left[ V^0(\tau',X_{\pi_1,\tau'}(\tau')) \right]  - \epsilon \\
\end{split}
\end{equation*}
Dabei folgt die erste Ungleichung durch Einsetzen von $\pi_1 \odot \pi_2'$ f�r das Surpremum. Die zweite Ungleichung folgt aus der Turmeigenschaft der bedingten Erwartung. Die
dritte Ungleichung folgt aus der Voraussetzung \eqref{eq:pistrichchar} an $\pi_2'$. Die vierte Ungleichung folgt aus der Definition von $V^0$. Die f�nfte Ungleichung folgt
schlie�lich aus der Voraussetzung \eqref{eq:taustrichchar} an $\tau'$. 

Die Handelsstrategie $\pi_1$ war aber beliebig gew�hlt. Auf der rechten Seite der letzten Ungleichungskette kommt $\pi_2$ nun gar nicht mehr vor, das Verhalten der Handelsstrategie
nach dem Crash beeinflusst diesen Term nicht. Deshalb k�nnen wir auf der rechten Seite das Surprememum �ber alle Handelsstrategien bilden und die
Ungleichung bleibt erhalten.
\begin{equation*}
\begin{split}
      &   \sup_{\pi} \inf_{\tau}    E_{t,x,1} \left[ U(X_{\pi \tau}(T)) \right] \\
\geq  &   \sup_{\pi}  E_{t,x,1} \left[ V^0(\tau',X_{\pi \tau'}(\tau')) \right]  - \epsilon \\
\end{split}
\end{equation*}
Nehmen des Infimum macht die rechten Seite nur kleiner und wir erhalten:
\begin{equation} \label{eq:Ungleichung1}
\begin{split}
      &   \sup_{\pi} \inf_{\tau}    E_{t,x,1} \left[ U(X_{\pi,\tau}(T)) \right] \\
\geq  &   \inf_{\tau}   \sup_{\pi}  E_{t,x,1} \left[ V^0(\tau,X_{\pi, \tau}(\tau)) \right]  - \epsilon \\
\end{split}
\end{equation}
Dies ist ein erstes wichtiges Zwischenergebnis in diesen Beweis.
Als n�chstes nehmen wir ein beliebiges $\tau$ und betrachten folgende Ungleichungskette (Erkl�rungen folgen):
\begin{equation*}
\begin{split}
      &  \sup_{\pi}     E_{t,x,1} \left[ U(X_{\pi,\tau}(T)) \right] \\
 =  & \  \sup_{\pi}     E_{t,x,1} \left[ E_{\tau,X_{\pi,\tau}(\tau),0} \left[  U(X_{\pi,\tau}(T))
\right] \right] \\
\leq &   \sup_{\pi}     E_{t,x,1} \left[  \sup_{\pi_2} E_{\tau,X_{\pi_1,\tau}(\tau),0} \left[  U(X_{\pi_1 \odot \pi_2,\tau}(T))
\right] \right] \\
\leq &   \sup_{\pi}     E_{t,x,1} \left[ V^0(\tau,X_{\pi,\tau}(\tau))   \right] \\
\end{split}
\end{equation*}
Dabei folgt die erste Gleichung mit der Turmeigenschaft der bedingten Erwartung, in der ersten Ungleichung haben wir unsere Notation $\pi = \pi_1 \odot \pi_2$ benutzt und den
Ausdruck in der Klammer durch Nehmen des Surpremums vergr��ert und in der letzten Ungleichung haben wir schlie�lich wieder die Definition von $V^0$ benutzt. Da $\tau$ beliebig
gew�hlt war, bleibt die Ungleichung durch Nehmen des Infimums auf der rechten Seite erhalten, und durch anschlie�enden Nehmen des Infimums auf der linken Seite nat�rlich auch.
\begin{equation} \label{eq:Ungleichung2}
\begin{split}
      & \inf_{\tau}  \sup_{\pi}     E_{t,x,1} \left[ U(X_{\pi,\tau}(T)) \right] \\
 \leq  &   \inf_{\tau}        \sup_{\pi}     E_{t,x,1} \left[ V^0(\tau,X_{\pi,\tau}(\tau))  \right]\\
\end{split}
\end{equation}
Das ist unser zweites wichtiges Zwischenergebnis und mit unseren beiden Zwischenergebnissen ist unser Satz schon fast bewiesen. Wir betrachten n�mlich nun die folgende
Ungleichungskette:
 \begin{equation*}
 \begin{split}
       &   \sup_{\pi} \inf_{\tau}    E_{t,x,1} \left[ U(X_{\pi, \tau}(T)) \right] \\
 \geq  &   \inf_{\tau}   \sup_{\pi}  E_{t,x,1} \left[ V^0(\tau,X_{\pi, \tau}(\tau)) \right]  - \epsilon \\
 \geq  &    \inf_{\tau}  \sup_{\pi}     E_{t,x,1} \left[ U(X_{\pi, \tau}(T)) \right]  - \epsilon\\
 \geq  &    \sup_{\pi}  \inf_{\tau}    E_{t,x,1} \left[ U(X_{\pi, \tau}(T)) \right]  - \epsilon
\end{split}
\end{equation*}
Dabei folgt die erste Ungleichung aus dem ersten Zwischenergebnis \eqref{eq:Ungleichung1}, die zweite Ungleichung folgt aus dem zweiten Zwischenergebnis \eqref{eq:Ungleichung2} und
die dritte Ungleichung aus der $ \sup \inf \leq \inf \sup$-Beziehung.

Da das $\epsilon$ beliebig gew�hlt war, bleibt hier jede Ungleichung auch ohne $\epsilon$ erhalten. Dann steht links und rechts von der Ungleichungskette das Gleiche, wir
erhalten also lauter Gleichheiten. Damit ist das Lemma bewiesen.
\end{proof}
Wir h�tten auch noch
 \begin{equation*}
 V^1(t,x) =  \sup_{\pi \in \mathcal{A}} \inf_{\tau}     E_{t,x,1}  \left[  V^0(\tau,X_{\pi,\tau}(\tau))  \right]
\end{equation*}
beweisen k�nnen. Da wir dieses Ergebnis im Folgenden nicht ben�tigen werden und der Beweis auch mit der gleichen Beweismethodik wie der gegebene gef�hrt werden kann verzichten wir
darauf. 

\section{Verifikationssatz f�r das Portfoliooptimierungsproblem}
Im folgenden formulieren und beweisen wir einen Verfikationssatz f�r $V^1$.

\begin{mySatz}
Seien die Voraussetzungen von Satz \ref{th:Verifikationsatz} erf�llt.
Sei $v^1 \in C^{1,2}$ und seien $\mathcal{A}'(t,x)$ und $\mathcal{A}''(t,x)$ durch
\begin{align}
\mathcal{A}'(t,x) & = \{ 
                \pi : \pi \in \reelleZahlen,  0 \leq \mathcal{L}^{\pi} v^1(t,x) 
                \} \\
\mathcal{A}''(t,x) & = \{ 
                  \pi : \pi \in\reelleZahlen, v^1(t,x) \leq V^0(t,x(1-k\pi)) 
                 \} 
\end{align}
definiert. Es gelte f�r alle $(t,x)$
\begin{equation} \label{eq:HJB1-Ungleichung}
  0 \leq \sup_{\pi \in \mathcal{A}''(t,x)} \left[ \mathcal{L}^{\pi} v^1(t,x) \right]
\end{equation}
\begin{equation} \label{eq:HJB2-crashrelation}
 0 \leq \sup_{\pi \in \mathcal{A}'(t,x)} \left[ V^0(t,x(1-k\pi))-v^1(t,x) \right] 
\end{equation}
\begin{equation} \label{eq:HJB3-complementarity}
 0 = \sup_{\pi \in \mathcal{A}''(t,x)} \left[ \mathcal{L}^{\pi} v^1(t,x) \right]  
     \sup_{\pi \in \mathcal{A}'(t,x)} \left[ V^0(t,x(1-k\pi))-v^1(t,x) \right]
\end{equation}
und
\begin{equation}
 v^1(T,x) = U(x)
\end{equation}
Existiert zus�tzlich eine zul�ssige Markov-Handelsstrategie mit
\begin{equation}
 p^1(t,x) \in \mathcal{A}''(t,x)
\end{equation}
und
\begin{equation} \label{eq:optimale_Strategie}
  \mathcal{L}^{p^1(t,x)} v^1(t,x) = \sup_{\pi \in \mathcal{A}''(t,x)} \left[ \mathcal{L}^{\pi} v^1(t,x) \right]
\end{equation}
f�r alle $(t,x)$, 
dann gilt $v^1(t,x)=V^1(t,x)$ und $p^1$ ist die optimale Strategie bis zum Crash.
\end{mySatz}
Dieser Satz macht also eine Aussage dar�ber, wann eine Markovstrategie eine optimale Strategie ist. Eine Markov-Strategie ordnet jedem m�glichen Prozesszustand, also ein bestimmtes
Verm�gen zu einer bestimmten Zeit, den Anteil des Verm�gens zu, der in die Aktie investiert wird, also eine reelle Zahl. Eine der Voraussetzungen die wir hier stellen ist, dass
diese Zahl in der Menge $\mathcal{A}''(t,x)$ liegt. Ausserdem soll die Strategie den Ausdruck $\mathcal{L}^{p^1(t,x)} v^1(t,x)$ in jedem Zustand $(t,x)$ maximieren. Das ist analog
zu der Forderung \eqref{eq:HJBPortfolio} aus dem Verifikationsatz \ref{th:Verifikationsatz} f�r das Portfoliooptimierungsproblem ohne Crash, allerdings muss das Surpremum nicht
$0$ sein und die Menge �ber die das Surpremum genommen wird ist auch eingeschr�nkt.


\begin{proof}
Wir beschreiben zun�chst die Struktur des Beweises. Mit Hilfe der It\^{o}-Formel k�nnen wir eine Relation zwischen $v^1(t,x)$ und den Werten von $v^1$ bis zum Crash herleiten
(vergleiche \eqref{itov}). Durch Angabe einer optimalen Strategie und aus den Voraussetzungen des Satzes gelingt es dann eine obere Schranke f�r $v^1(t,x)$ (vergleiche
\eqref{eq:erstewichtigeUngleichung}) anzugeben, und genauso gelingt es durch Angabe eines optimalen Crashes eine untere Schranke f�r $v^1(t,x)$ (vergleiche
\eqref{eq:zweitewichtigeUngleichung}) herzuleiten. Diese Schranken fallen zusammen und charakterisieren nach Satz \ref{th:Bellmanprinzip} die optimale Wertfunktion. Dass die
optimale Strategie wirklich optimal ist folgt dann aus einem auf dem Weg gewonnenen Ergebnis. Diesen Plan f�hren wir nun aus.
 
Wir gehen davon aus, das in $(t,x)$ f�r das Verm�gen $X(t)=x$ gilt und noch ein Crash m�glich ist.
Anwendung der It\^{o}-formel ergibt dann f�r $t \leq s < \tau$
\begin{equation*} 
\begin{split}
 &dv^1(s,X_{\pi,\tau}(s)) \\
=&\mathcal{L}^{\pi(s)}v(s,X_{\pi,\tau}(s))ds + v_x^1(s,X_{\pi,\tau}(s)) \pi(s) \sigma X_{\pi,\tau}(s) dW(s), \\
\end{split}
\end{equation*}
% und
% \begin{equation}
%  dv^1(\tau,X(\tau))= v^1(\tau,X(\tau))(1-\pi(\tau)k) - v^1(\tau-,X(\tau-))
% \end{equation}
also
\begin{equation} \label{itov}
\begin{split}
 & v^1(\tau-,X_{\pi,\tau}(\tau-)) - v^1(t,x) \\
= & \int_t^{\tau} \mathcal{L}^{\pi(s)}v(s,X_{\pi,\tau}(s))ds
                            +   \int_t^{\tau} v_x^1(s,X_{\pi,\tau}(s)) \sigma X_{\pi,\tau}(s) \pi(s) dW(s) 
\end{split}
\end{equation}
Wir fixieren nun die Strategie $p^1$ aus den Vorausetzungen des Satzes. Dann folgt aus der HJB-Ungleichung \eqref{eq:HJB1-Ungleichung}  und der Vorrausetzung
\eqref{eq:optimale_Strategie} an $p^1$ f�r $t \leq s \leq \tau$:
\begin{equation} \label{rauch}
 0 \leq \sup_{\pi \in \mathcal{A}''(s,X_{p^1,\tau}(s))} \left[ \mathcal{L}^{\pi} v^1(s,X_{p^1,\tau}(s)) \right] = 
    \mathcal{L}^{p(s,X_{p^1,\tau}(s))} v^1(s,X_{p^1,\tau}(s)) 
\end{equation}
und da $p(s,X_{p^1,\tau}(s)) \in \mathcal{A}''(s,X_{p^1,\tau}(s)) $
\begin{equation} \label{schall}
  v^1(s,X_{p^1,\tau}(s)) \leq V^0(s,X_{p^1,\tau}(s)(1-k  p^1(s,X_{p^1,\tau}(s))))
\end{equation}
Wenn wir nun $p^1$ in \eqref{itov} einsetzen und die Gleichung nach $v^1(t,x)$ umstellen, erhalten wir:
\begin{equation}
\begin{split}
  v^1(t,x) &= v^1(\tau-,X_{p^1,\tau}(\tau-)) - \int_t^{\tau} \mathcal{L}^{p^1(s,X_{p^1,\tau}(s))}v(s,X_{p^1,\tau}(s))ds \\
           &-   \int_t^{\tau} v_x^1(s,X_{p^1,\tau}(s)) \sigma X_{p^1,\tau}(s) p^1(s,X_{p^1,\tau}) dW(s). 
\end{split}
\end{equation}
Mit \eqref{schall} k�nnen wir den ersten Summanden auf der rechten Seite nach oben absch�tzen, dabei beachten wir noch die Beziehung 
\begin{equation*}
X_{\tau,\pi}(\tau) = X_{\pi,\tau}(\tau-) \left( 1 - \pi(\tau) k \right).
\end{equation*}
 Den zweiten
Summanden k�nnen wir wegen \ref{rauch} durch Weglassen nach oben absch�tzen und erhalten:
\begin{equation}
 v^1(t,x) \leq V^0(\tau,X_{p,\tau}(\tau)) -  \int_t^{\tau} v_x^1(s,X(s)) \sigma X_{p,\tau}(s) dW(s) 
\end{equation}
Bilden des Erwartungswertes f�hrt dann zu:
\begin{equation} \label{ungleichung}
 v^1(t,x) \leq E_{t,x,1} \left[ V^0(\tau,X_{p,\tau}(\tau))  \right] 
\end{equation}
Durch Bilden des Surpremums �ber alle Handelsstrategien vergr��ern wir nun die rechte Seite:
\begin{equation} 
 v^1(t,x) \leq \sup_{\pi} E_{t,x,1} \left[ V^0(\tau,X_{\pi,\tau}(\tau))  \right]  
\end{equation}
Da das $\tau$ beliebig gew�hlt war erhalt nun auch das Bilden des Infimums auf der rechten Seite die
Ungleichung:
\begin{equation} \label{eq:erstewichtigeUngleichung} 
 v^1(t,x) \leq \inf_{\tau} \sup_{\pi} E_{t,x} \left[ V^0(\tau,X(\tau)  \right] 
\end{equation}
Das ist das erste wichtige Zwischenergebnis und stellt die obere Schranke dar, von der schon in der Beschreibung des Beweisplans die Rede war.

Unser n�chstes Ziel ist es nun eine untere Schranke zu finden.
Sei dazu $\pi$ wieder eine beliebige Handelsstrategie. Dann fixieren wir dazu die Stoppzeit
\begin{equation}
 \theta = \inf \{s \in R; V^0(s,X_{\pi}(s)(1-\pi(s))) \leq v^1(s,X_{\pi}(s)) \}
\end{equation}
Nach Definition von $\theta$ gilt f�r $t\leq s < \theta$
\begin{equation} \label{super}
 V^0(s,X_{\pi,\theta}(s)(1-k \pi(s))) > v^1(s,X_{\pi,\theta}(s))
\end{equation}
und
\begin{equation} \label{duper}
 V^0(\theta,X_{\pi,\theta}(\theta-)(1-k \pi(\theta))) \leq v^1(\theta,X_{\pi}(\theta))
\end{equation}
F�r ein festes $s$ muss nun entweder
\begin{equation} \label{hanni}
 \mathcal{L}^{\pi(s,X_{\pi,\theta}(s))} v_1(s,X_{\pi,\theta}(s)) < 0 
\end{equation}
oder 
\begin{equation} \label{nanni}
 0 \leq \mathcal{L}^{\pi(s,X_{\pi,\theta}(s))} v_1(s,X_{\pi,\theta}(s))
\end{equation}
gelten.
Gilt letzteres, so ist $\pi(s,X_{\pi, \theta}(s)) \in \mathcal{A}'(s,X_{\pi,\theta}(s)) $ und aus \eqref{super} folgt
\begin{equation} 
 \sup_{\pi \in \mathcal{A}'(s,X_{\pi,\theta}(s))} \left[ V^0(s,X_{\pi,\theta}(s)(1-k\pi))-v^1(s,X_{\pi,\theta}(s)) \right] > 0 
\end{equation}
Wegen der Voraussetzung \eqref{eq:HJB3-complementarity} des Satzes folgt dann 
\begin{equation}
 0 = \sup_{\pi \in \mathcal{A}''(s,X_{\pi,\theta}(s))} \left[ \mathcal{L}^{\pi} v^1(s,X_{\pi,\theta}(s)) \right]. 
\end{equation}
Daraus, und da wir aus \eqref{super} wissen, dass $\pi(s,X_{\pi,\theta}(s)) \in \mathcal{A}''(s,X_{\pi,\theta}(s))$ gilt, folgt
\begin{equation}
 \mathcal{L}^{\pi(s,X_{\pi,\theta}(s))} v_1(s,X_{\pi,\theta}(s)) \leq 0 
\end{equation}
Also unabh�ngig ob \eqref{hanni} oder \eqref{nanni} gilt, es folgt
\begin{equation} \label{shappi}
 \mathcal{L}^{\pi(s,X_{\pi,\theta}(s))} v_1(s,X_{\pi,\theta}(s)) \leq 0 
\end{equation}
Wenn wir nun wieder von \eqref{itov} ausgehen, dort $\theta$ einsetzen und die Gleichung nach $v^1(t,x)$ umstellen erhalten wir 
\begin{equation}
\begin{split}
v^1(t,x) &= v^1(\theta,X_{\pi,\theta}(\theta)) - \int_t^{\theta} \mathcal{L}^{\pi(s)}v(s,X_{\pi,\theta}(s))ds \\
         & -   \int_t^{\theta} v_x^1(s,X_{\pi,\theta}(s)) \sigma X_{\pi,\theta}(s) dW(s)
\end{split}
\end{equation}
Mit \eqref{duper} k�nnen wir den ersten Summanden auf der rechten Seite verkleinern, dabei beachten wir $X_{\pi,\theta}(\theta) = X_{\pi,\theta}(\theta-)(1-k\pi(\tau))$. Wegen
\eqref{shappi}
wird die rechte Seite durch Weglassen des zweiten Summanden ebenfalls nur kleiner. So erhalten wir
\begin{equation}
 v^1(t,x) \geq V^0(\theta,X_{\pi,\theta}(\theta))  -   \int_t^{\theta} v_x^1(s,X(s)) \sigma X_{\pi,\theta}(s) dW(s). 
\end{equation}
Bildung des Erwartungwertes f�hrt zu
\begin{equation} \label{ungleichung2}
  v^1(t,x) \geq E_{t,x,1} \left[ V^0(\theta,X_{\pi,\theta}(\theta)) \right]
\end{equation}
% n�chster unn�tiger Teil
% Wie im ersten Teil des Beweises nehmen wir nun surprema und infina:
% Bilden des Infimums �ber alle Crashzeiten macht die rechte Seite von \ref{ungleichung2} nur kleiner.
%  \begin{equation} 
%   v^1(t,x) \geq \inf_{\tau} E_{t,x} \left[ V^0(\tau-,X(\tau-)) \right]
%  \end{equation}
%  Da $\pi$ beliebig gew�hlt war, bleibt die Ungleichung durch Bildung des Surpremums �ber alle Handelstrategie
% erhalten:
% \begin{equation} 
%  v^1(t,x) \geq \sup_{\pi} \inf_{\tau} E_{t,x} \left[ V^0(\tau-,X(\tau-)) \right]
% \end{equation}
Da die Handelstrategie beliebig gew�hlt war, k�nnen wir das Surpremum �ber alle Handelsstrategien nehmen und die Ungleichung bleibt erhalten.
\begin{equation} 
 v^1(t,x) \geq \sup_{\pi}  E_{t,x,1} \left[ V^0(\theta,X_{\pi,\theta}(\theta)) \right]
\end{equation}
Bilden des Infimums macht die rechte Seite nur kleiner.
\begin{equation} \label{eq:zweitewichtigeUngleichung}
 v^1(t,x) \geq \inf_{\tau} \sup_{\pi}  E_{t,x,1} \left[ V^0(\tau,X_{\pi,\tau} (\tau)) \right]
\end{equation}
Das war unser zweites Ziel, die untere Schranke f�r  $v^1(t,x)$.

Nun folgt aus der oberen Schranke aus \eqref{eq:erstewichtigeUngleichung} und der gerade angegebenen unteren Schranke \eqref{eq:zweitewichtigeUngleichung} wegen deren Gleichheit
die Gleichung
\begin{equation} 
 v^1(t,x) = \inf_{\tau} \sup_{\pi}  E_{t,x} \left[ V^0(\tau,X_{\pi,\tau}(\tau)) \right]
\end{equation}
Nach dem vorangegangen Satz \ref{th:Bellmanprinzip} folgt dann also $V^1 = v^1$.

Es bleibt noch die Optimalit�t der Strategie zu zeigen: Ausgehend von \eqref{ungleichung} k�nnen wir, da $\tau$ beliebig gew�hlt war, auf der rechten Seite das Infimum bilden und
wir
erhalten
\begin{equation}
 v^1(t,x) \leq \inf_{\tau} E_{t,x,1} \left[ V^0(\tau,X_{p^1,\tau}(\tau))  \right] 
\end{equation}
Das bedeutet aber gerade, dass die Strategie optimal ist.
\end{proof}

Bei der Formulierung des Satzes war etwas vage von "`optimale Strategie bis zum Crash"' die Rede. Was damit gemeint war, wurde dann im Beweis klar, n�mlich
\begin{equation*}
 \inf_{\tau} \sup_{\pi \in \mathcal{A}}  E_{t,x,1}  \left[  V^0(\tau,X_{\pi,\tau}(\tau))  \right] = \inf_{\tau} E_{t,x} \left[ V^0(\tau,X_{p^1,\tau}(\tau))  \right] 
\end{equation*}
Damit wurde strenggenommen noch nicht das Portfoliooptimierungsproblem gel�st, dort ist n�mlich ein $p$ gesucht mit
\begin{equation*}
 \inf_{\tau} \sup_{\pi \in \mathcal{A}}  E_{t,x,1}  \left[  U(X_{\pi, \tau}(T))  \right] = \inf_{\tau} E_{t,x,1} \left[ U(X_{p,\tau}(T))  \right] 
\end{equation*}
Solche ein p k�nnen wir aber einfach konstruieren. Ist $\tau$ die Crashzeit, so betrachten wir
\begin{equation*}
 p = p^1 \Indikatorfunktion_{[0,\tau]} + p^0 \Indikatorfunktion_{(\tau,T]}
\end{equation*}
Solch eine Handelsstrategie ist nat�rlich nicht mehr stetig. Mit den gleichen Methoden wie in Satz \ref{th:Bellmanprinzip} und dem Wissen das $p^0$ im crashfreien Scenario
optimal ist, kann  man dann die obige Gleichung zeigen.

Als n�chstes wollen wir noch bemerken, dass man ganz leicht zeigen kann, dass aus den Voraussetzungen des Verfikationsatzes folgt, dass die optimale Strategie am Ende des
Zeithorizonts kein Geld mehr in Aktien haben darf, wenn man den logaritmischen Nutzen betrachtet. Dazu berechen wir zun�chst
\begin{equation*}
\begin{split}
\mathcal{A}''(T,x) & = \{ 
                  \pi : \pi \in \reelleZahlen, v^1(T,x) \leq V^0(T,x(1-k\pi)) 
                 \} \\
& = \{ 
                  \pi : \pi \in \reelleZahlen , \log x \leq \log x + \log (1-k\pi) 
                 \}  = \reelleZahlen^-\\
\end{split}
\end{equation*}
Da $p^1(T,x) \in \mathcal{A}''(T,x)$ gefordert ist, ist die Behauptungs damit schon gezeigt.




\section{Logaritmischer Nutzen}
\label{sec:log_utility_hjb}
Wir zeigen nun, dass wenn $\pidach$ und $f$ L�sungen der Differentialgleichungen
\begin{align*}
 \pidach_t(t) &=\frac{1}{k} (1- \pidach(t) k) 
  \left(
   \pidach(t) (\mu-r) - \einhalb \left[ \bruch^2 + \pidach(t)^2 \sigma^2 \right] 
  \right) \\
 \pidach(T) &=0
\end{align*}
und
\begin{align*} 
f^1_t(t) &=  -(r+\pidach(t) (\mu-r)) + \einhalb \pidach(t)^2 \sigma^2 \\
f^1(T) &=0
\end{align*}
sind, dann erf�llen die Wertfunktion
\begin{equation*} 
 V^1(t,x) = \log(x) + f^1(t)
\end{equation*}
und die optimale Strategie $\pidach$ alle Voraussetzungen des Verifikationssatzes.

Zun�chst berechnen wir, wie der Operator $\mathcal{L}$ auf unter diesen speziellen Voraussetzungen aussieht. Es gilt
\begin{equation*}
\begin{split}
 \mathcal{L}^{\pi}V^1(t,x) & =f^1_t(t) +r + \pi (\mu-r) + \einhalb \pi^2 \sigma^2 \\
                           & =f^1_t(t) +r  + f(\pi),
\end{split} 
\end{equation*}
wobei wir das $f$ aus Definition \ref{def:f} vewendet haben.

Nun zeigen wir, dass
\begin{equation} \label{eq:logGleichgewicht}
 V^1(t,x)=V^0(t,x(1-k \pidach(t,x))
\end{equation}
gilt. Dazu halten wir $x$ fest und betrachten beide Seiten als Funktionen von $t$. F�r $t=T$ gilt:
\begin{equation*}
 V^1(T,x)= \log x = V^0(T,x) = V^0(T,x(1-k \pi(T,x))
\end{equation*}
Ableiten der rechten Seite von \eqref{eq:logGleichgewicht} ergibt
\begin{equation*}
\begin{split}
& \frac{d}{dt} \left[ ( V^0(t,x(1-k \pi(t,x)) ) )  \right] \\
= & \frac{d}{dt} \left[ \log x + \log (1-k\pi(t)) + \left( r + \einhalb \bruch^2 \right) (T-t) \right] \\
= & \frac{-k \pidach_t(t)}{1-k \pidach(t)} - r - \einhalb \bruch^2 \\
= & \einhalb \left[ \bruch^2 + \pidach(t)^2 \sigma^2 \right] - \pidach(t) (\mu-r)- r - \einhalb \bruch^2 \\
= & \einhalb \pidach(t)^2 \sigma^2  - \pidach(t) (\mu-r)- r = f^1_t(t),  \\
\end{split}
\end{equation*}
und das ist offensichtlich die Ableitung der linken Seite von \eqref{eq:logGleichgewicht}. Da wir deren Gleichheit in einem Punkt und die Gleichheit der Ableitungen gezeigt
haben, gilt auch in jedem Punkt die Gleichheit. Daraus folgt dann
\begin{equation*}
 \pidach(t,x) \in \mathcal{A}''(t,x)
\end{equation*}
und die Gleichgewichtsbeziehung \eqref{eq:HJB2-crashrelation} aus den Voraussetzungen des Verifikationssatzes ist gezeigt. 

Die Beziehung
\begin{equation} 
 V^1_t(t,x) = -V^1_x(t,x)(r+\pidach(\mu-r))x - \einhalb V^1_{xx}(t,x) \pidach^2 \sigma^2 x^2
\end{equation}
kann man durch Einsetzen der entsprechenden Ableitungen sofort �berpr�fen.
Also gilt f�r alle $(t,x)$
\begin{equation*}
 \pidach(t,x) \in \mathcal{A}'(t,x)
\end{equation*}
und es gilt die HJB-Ungleichung \eqref{eq:HJB1-Ungleichung} aus den Voraussetzungen des Verifikationssatzes. Es
bleibt noch die Bedingung \eqref{eq:HJB3-complementarity} zu �berpr�fen.
Dazu werden wir
\begin{equation*}
 \sup_{\pi \in \mathcal{A}'(t,x)} \left[ V^0(t,x(1-k\pi))-V^1(t,x) \right] =0
\end{equation*} 
zeigen. Wir berechnen zun�chst
\begin{equation*}
\begin{split}
  & \mathcal{A}'(t,x)  \\
 =& \{ 
     \pi :  0 \leq \mathcal{L}^{\pi} V^1(t,x) 
     \} \\
=& \{ 
     \pi :  0 \leq -(r+\pidach(t) (\mu-r)) + \einhalb \pidach(t)^2 \sigma^2 + r - (\mu -r) \pi + \einhalb \pi^2 \sigma^2
     \} \\
=& \{ 
     \pi :  0 \leq -f(\pidach(t)) + f(\pi)
     \} \\
\subset & \{ 
     \pi :  \pidach(t) \leq  \pi
     \} \\
\end{split}
\end{equation*}
Dabei haben wie wieder unsere Funktion $f$ aus Definition \ref{def:f} verwendet und die letzte Mengeninklusion folgt aus der Tatsache dass $\pidach(t)<\pistar$ gilt und aus der
Gestalt von f.
Die schon gezeigte Gleichung \eqref{eq:logGleichgewicht} besagt nun
\begin{equation*}
 \log x + f^1(t) = \log x + \log(1 - k \pidach(t)) + f^0(t).
\end{equation*}
Ersetzen wir $\pidach(t)$ durch ein $\pi$ aus $\mathcal{A}'(t,x)$, so wird die rechte Seite kleiner, und wir erhalten 
\begin{equation*}
V^1(t,x) \geq V^0(t,x(1-\pi(t) k)). 
\end{equation*}
Es gilt also tats�chlich
\begin{equation*}
 \sup_{\pi \in \mathcal{A}'(t,x)} \left[ V^0(t,x(1-k\pi))-V^1(t,x) \right] =0.
\end{equation*}
Damit ist nun auch die Bedingung \eqref{eq:HJB3-complementarity} aus den Voraussetzungen des Verifikationssatzes gezeigt. Es bleibt noch zu zeigen, dass f�r $\pidach$ die
Gleichung \eqref{eq:optimale_Strategie} gilt. Wir berechnen daf�r:
\begin{equation*}
\begin{split}
  & \mathcal{A}''(t,x)  \\
 =& \{ 
     \pi :   V^1(t,x) \leq V^0(t,x(1-k \pi)) 
     \} \\
=& \{ 
     \pi :  f^1(t) \leq \log(1-k \pi) + f^0(t)
     \} \\
=& \{ 
     \pi :  \pi \leq \pidach(t)
     \}.
\end{split}
\end{equation*}
Dabei ergibt sich die letzte Gleichheit, da genau f�r $\pidach$ in der die Menge definierenden Ungleichung Gleichheit herrscht und die rechte Seite monoton falllend in $\pi$ ist.
Nun gilt
\begin{equation*}
\begin{split}
 & \mathcal{L}^{\pi} V^1(t,x) \\
 = & f^1_t(t) + r + \pi (\mu -r) - \einhalb \pi^2 \sigma^2 \\
 = & f^1_t(t) + r + f(\pi), \\
\end{split}
\end{equation*}
mit unserer Funktion $f$ aus Definition \ref{def:f} und da $f$  monoton wachsend bis $\pistar$ ist und $\pidach(t) \leq \pistar$ gilt wird das Surpremum �ber alle $\pi \in
\mathcal{A}''(t,x)$ tats�chlich mit $\pidach(t)$ angenommen und damit ist auch \eqref{eq:optimale_Strategie} gezeigt. Damit sind alle Voraussetzungen des Verifikationssatzes
erf�llt.


Wie auch schon im Abschnitt \ref{sec:log1} scheint die L�sung hier vom Himmel gefallen zu sein. In \cite{Korn2007} wird in einem dem Autor dieser Diplomarbeit schwer
verst�ndlichen Abschnitt versucht mit heuristischen Methoden aus den Voraussetzungen des Verifikationsatzes auf Kandidaten f�r die Wertfunktion und die optimale Strategie zu
schlie�en. Dort gelangen die Autoren dann schliesslich auch zu der hier angegeben L�sung. Wieder besteht bei solch einem Vorgehen die gro�e Gefahr notwendige und hinreichende
Bedingungen durcheinander zu werfen und wieder haben wir in dieser Arbeit den umgekehrten Weg beschritten und haben die L�sung einfach nur, wenn auch m�hselig, verifiziert.

\section{Vergleich der beiden Ans�tze}
Zun�chst einige Bemerkungen zum stochastischen Modell. 
Im ersten Ansatz wurde der Crash nicht in das stochastische Modell des Aktienkurses mit aufgenommen. Es wurde anschaulich argumentiert, wie das Endverm�gen aussieht, wenn ein Crash
zur Zeit $t$ geschieht und dann einfach das Infimum �ber alle m�glichen Zeitpunkte genommen. Damit war es auch nicht
m�glich zu formulieren wie eine Handelstrategie auf einem Crash reagieren kann und wie mussten das Problem mit $V^0$ formulieren. 
Im zweiten Ansatz hingegen war der Crash im Aktienkurs modelliert, das Endverm�gen ergibt sich aus der L�sung einer stochastischen Differentialgleichung, die Formulierung des
Optimierungsproblems war viel weniger schwerf�llig und das Bellmanprinzip aus Satz \ref{th:Bellmanprinzip} konnten wir beweisen anstatt er voraussetzen zu m�ssen. Allerdings haben
wir dabei auch Sprungprozesse im Integrator des stochastischen Integrals gehabt, w�hrend wir beim ersten Ansatz nicht �ber das It\^{o}-Kalk�l hinausgehen mussten.
Mit dem ersten Ansatz konnten wir eine L�sung nur f�r den Fall des logarithmischen Nutzen angeben, an wichtigen Stellen im Beweis werden Eigenschaften des Logarithmus ausgenutzt.
Daf�r wird im Beweis genau klar, was eigentlich passiert, wie n�mlich das Ausbalancieren funktioniert und dass jede Strategie die nicht mit der Gleichgewichtsbedingung
ausbalanciert ist eine niedrigere Worst-Case-Schranke besitzt.
Beim zweiten Ansatz wird dieses Motiv viel weniger deutlich, auch wenn in der einen Gleichung die Gleichgewichtsbedingung steht. Daf�r kann man das Ergebnis als eine
Verallgemeinerung des gew�hnlichen HJB-Ansatz sehen und wir mussten auch nicht die Nutzenfunktion spezifizieren.


\chapter{Zahlenbeispiele und Simulation}
\label{chap:Zahlenbeispiele}
\section{Worst-Case-Schranke der optimalen Strategie}
In diesem Kapitel wollen wir konkret ein paar Zahlenbeispiele rechnen. Dazu spezifizieren wir unsere Marktkoeffizienten.
\begin{align*}
 \mu &=0.2 \\
   r&=0.05 \\
 \sigma &= 0.4
\end{align*}
Au�erdem spezifizieren wir unsere pers�nlichen Parameter zu
\begin{align*}
 x_0 &=1 \\
  T&=10 
\end{align*}
Der erwartete Nutzen durch Verfolgen der reinen Bondstrategie ist dann
\begin{equation*}
 rT=0.5
\end{equation*}
Die optimale Strategie im crasfreien Scenario f�hrt zu einem erwarteten Endverm�gen von
\begin{equation*}
 V^0(0,1,\log)=1.203
\end{equation*}
Als optimalen erwarteten Nutzen im Markt mit Crash erwarten wir nach den Ergebnissen dieser Arbeit dann einen Wert dazwischen. Dazu ben�tigen wir die Werte der L�sung $\pidach$ von
der Differentialgleichung aus Satz \ref{th:DGL}.  Wir haben diese numerisch mit der Mathematiksoftware  \texttt{octave} gel�st. Der Graph dieser L�sung ist in Abbildung
\ref{fig:optimaleStrategie} abgebildet.

\begin{figure}[htbp]
\centering
\includegraphics[type=eps,ext=.eps,read=.eps,width=1\linewidth]{Bilder/foo}
\label{fig:optimaleStrategie}
\caption{Optimale Handelstrategie $\pidach$}%
\end{figure}


Wir haben nun verschieden M�glichkeiten den erwarteten Nutzen bei Verfolgen dieser Strategie. Im Beweis von Satz \ref{th:hauptsatz_willmott} haben wir gelernt, dass es reicht wenn
wir den Wert von der numerisch ermittelten L�sung von $\pidach$ an der Stelle $0$ nehmen und wie folgt in die Wertfunktion einsetzen
\begin{equation*}
 V^0(0,1-\pidach(0)k,\log)=1.0472
\end{equation*}
Genauso ist es aber auch m�glich den erwarten Nutzen im crashfreien Scenario zu berechnen. Satz \ref{th:ExpLogUtility} liefert daf�r die Formel und wir erhalten 
\begin{equation*}
 \int_0^T f(\pidach(t)) dt + rT = 1.0471.
\end{equation*}
\section{Berechnung von Worst-Case-Schranken durch Monte"=Carlo"=Simulation}

Bei den bisherigen Berechnungen haben wir die Ergebnisse aus den S�tzen des theoretischen Teils dieser Arbeit benutzt, insbesondere aus Satz \ref{th:hauptsatz_willmott}, der nur
eine Aussage f�r den logarithmischen Nutzen macht. Im folgenden berechnen wir \wcs ohne diesen theoretischen Unterbau, die Berechnungen beschr�nken sich dabei nicht
auf den logarithmischen Nutzen.

Wir stellen zun�chst unsere in  \texttt{octave} programmierte Funktion \texttt{worstcase} vor.
Diese Funktion berechnet die \wcs nach dem Vorbild von Definition \ref{def:WCPortfolioproblem1}. Dort wurde die \wcs wie folgt berechnet:
\begin{equation*}
  \min \left( 
                   \inf_{0 \leq s \leq T} E \left[ V^0(s,X_{\pi}(s)(1-\pi(s)k),U) \right] 
                  , 
                   E \left[ U(X_{\pi}(T)) \right] 
                  \right)
\end{equation*}
Die Erwartungswerte werden wir mit einer Monte"=Carlo"-Simulation ermitteln. Mit einem Computer k�nnen wir nat�rlich nicht das Infimum �ber unendlich viele Werte nehmen. Da eine
Monte"=Carlo"-Simulation eine sehr teure Operation ist, k�nnen wir das Infimum auch nur �ber eine relativ kleine Menge von Zeitpunkten berechnen. Deshalb �bergeben wir der Funktion
eine Liste von Werten, f�r die dann die jeweilige Erwartung f�r ein Crash zu diesem Zeitpunkt berechnet wird. Die folgende Tabelle beschreibt die Schnittstelle zu dieser Funktion.

\begin{table}[htbp]
% \centering
\begin{tabular}{|l|l|}
\hline
\multicolumn{2}{|l|}{Funktion: worstcase} \\
\hline
\multicolumn{2}{|l|}{Eingaben} \\
\hline
 v &  Wertfunktion im crashfreien Scenario \\
 p &  Handelstrategie  \\
 t &  Liste von Crashzeitpunkten  \\
\hline
\multicolumn{2}{|l|}{Ausgaben} \\
\hline
crashUtilities & Erwarteter Nutzen zu dem jeweiligen Crashzeitpunkten \\
noCrashUtility & Erwarteter Nutzen ohne Crash\\
worstcasebound & \wcs\\
\hline
\end{tabular}
\caption{Beschreibung der Funktion worstcase}
\end{table}

Wir illustrieren die Benutzung dieser Funktion, in dem wir die Beispiele besprechen, die wir auch schon im theoretischen Teil behandelt haben. Dazu haben wir eine feste Liste von
Crashzeitpunkten, die Wertfunktion f�r den logaritmischen Nutzen und jeweils verschiedene Handelsstrategien als Eingaben an die Funktion �bergeben und haben in Tabelle
\ref{tab:Ausgaben_von_worstcase} die Ergebnisse zusammengefasst.
Als Erstes betrachten wir die Ausgaben f�r die reine Bondstrategie $\pi_0$ deren Worst-Case-Schranke wir in Satz \ref{th:worst-case-schranke-reine-bondstrategie} berechnet haben.
Dort hatten wir schon erw�hnt dass das Ausbleiben eines Crashes das Worst-Case-Scenario ist. Dass wird auch an der Programmausgabe deutlich. Je sp�ter der Markt crasht, umso
weniger Zeit hat der Investor die optimale Strategie im crashfreien Markt zu verfolgen.
Als n�chstes betrachten wir die optimale Strategie $\pistar$ im crashfreien Scenario. Hier hatten wir in Satz \ref{th:worst-case-schranke-optimal-ohne-crash} behauptet, dass ein
Crash mehr schadet als kein Crash, aber der Zeitpunkt egal ist. Auch dieses Verhalten l�sst sich aus der Ausgabe herauslesen.
Schlie�lich betrachten wir noch die Ausgabe f�r die Handelsstrategie $\pidach$. Hier erwarten wir, dass wir die Gleichgewichtsbedingung wiederfinden. Wir sollten f�r jeden
Crashzeitpunkt als auch f�r den chrashfreien Fall den gleichen erwarteten Nutzen haben. Auch dieses Verhalten finden wir in der Ausgabe best�tigt.

\begin{table}[htbp]
% \centering
\begin{tabular}{|l|l|}
\hline
\multicolumn{2}{|p{11cm}|}{Ausgaben der Funktion worstcast mit den Eingaben $v=V^0(\cdot,\cdot,\log)$, $t=[0,2,5,8,10]$ und $p$ wie angegeben } \\
\hline
\hline
\multicolumn{2}{|l|}{Eingabe:  $p=p_0$  } \\
\hline
\multicolumn{2}{|l|}{Ausgaben} \\
\hline
crashUtilities & $[1.20312 ,1.06248,0.85150,0.64053,0.49988]$ \\
noCrashUtility & $0.49938$ \\
worstcasebound & $0.49938$ \\
\hline
\hline
\multicolumn{2}{|l|}{Eingaben: $\pistar$} \\
\hline
\multicolumn{2}{|l|}{Ausgaben} \\
\hline
crashUtilities & $[0.99549,0.99599,0.99116,1.00101,0.99856]$ \\
noCrashUtility & $1.2125$ \\
worstcasebound & $0.99116$ \\
\hline
\hline
\multicolumn{2}{|l|}{Eingaben: $\pidach$} \\
\hline
\multicolumn{2}{|l|}{Ausgaben} \\
\hline
crashUtilities & $[1.0472,1.0436,1.0486,1.0559,1.0476]$ \\
noCrashUtility & $1.0351$ \\
worstcasebound & $1.0351$ \\
\hline
\end{tabular}
\caption{Berechnung von Worst-Case-Schranken f�r verschiedene Handelsstrategien}
\label{tab:Ausgaben_von_worstcase}
\end{table}

\section{Beste lineare Strategie}

Mit unsere Funktion k�nnen wir nun also jede Handelsstrategie beurteilen. Damit k�nnen wir nun zum Beispiel einen einfachen Algorithmus angeben um eine beste lineare Strategie zu
berechnen. Mit linearer Strategie meinen wir eine lineare Funktion, die beim Zeithorizont den Wert $0$ besitzt. Die Suche nach solch einer Strategie ist motiviert durch die
Vorstellung, dass je weiter man sich dem Zeithorizont n�hert, desto vorsichtiger sollte man in Aktien investieren. Eine lineare Funktion ist f�r solch ein Verhalten
das einfachste Modell. Eine lineare Strategie ist durch einen Punkt auf der y-Achse festgelegt. Wenn wir nun zu einer Menge von Punkten auf der y-Achse jeweils die \wcs
berechnen, und dann den mit der h�chsten \wcs w�hlen, m�ssten wir die beste lineare Strategie gefunden haben.

In der Tabelle \ref{tab:lineare_Strategien} haben wir die Ergebnisse dieser Rechnung dargestellt. Dabei haben wir durch vorherige Experimente diesen Bereich als den
kritischen erkannt. Man erkennt dass die beste lineare Strategie, die mit dem Startwert $1.0$ ist. Die dazugeh�rige Worst-Case-Schranke ist mit $0.97409$ nur geringf�gig kleiner
als die Worst-Case-Schranke der optimalen Strategie und deutlich gr��er als der erwartete Nutzen der reinen Bondstrategie. In der Abbildung \ref{fig:optimale_lineare_Strategie}
haben wir die optimale lineare Strategie und die optimale Strategie nochmal im Vergleich dargestellt.

\begin{table}[!htbp]
% \centering
\begin{tabular}{|lllllll|}
\hline
\multicolumn{7}{|l|}{Startpunkte} \\
\hline
 0.7 &  0.8 & 0.9 & 1.0 & 1.1 & 1.2 & 1.3 \\
\hline
\multicolumn{7}{|l|}{Worst-Case-Schranken} \\
\hline
0.88871 &  0.91934 &  0.96035 &  0.97409 &  0.95466 &  0.92869 &  0.90202 \\
\hline
\end{tabular}
\caption{Worst-Case-Schranke f�r lineae Strategie mit unterschiedlichen Startwerten}
\label{tab:lineare_Strategien}
\end{table}




\begin{figure}[htbp]
\centering
\includegraphics[type=eps,ext=.eps,read=.eps]{Bilder/bar}
\caption{Optimale und optimale lineare Strategie im Vergleich}%
\label{fig:optimale_lineare_Strategie}
\end{figure}

\section{Programmcode}
Alle Ergebnisse wurden auch schon in den vorherigen Abschnitten besprochen. Inhaltlich wird der Leser hier also nichts Neues finden, dieser Abschnitt ist f�r den an der
Implementierung interessierten Leser gedacht.

Der Code ist in zwei Dateien aufgeteilt. Zum einen haben wir das Script \texttt{diplomarbeit.m}, dass alle Ergebnisse aus den vorherigen Abschnitten produziert. Dieser Code kann
gut parallel zu den dort gegebenen Kommentaren gelesen werden. Zum anderen haben wir noch das M-file \texttt{worstcase}, in der die Funtion zur Berechnung der \wcs implementiert
ist.

\subsection{Der Code von diplomarbeit.m}
\lstinputlisting{Octaveprogramme/diplomarbeit.m} 

\subsection{Der Code von worstcase.m}
\lstinputlisting{Octaveprogramme/worstcase.m} 
% \chapter{Symbolverzeichnis}

% Literaturverzeichnis
\bibliographystyle{geralpha}
\nocite{*} % auch alle nicht im Text zitierten Eintr�ge werden mit aufgenommen
\bibliography{Literatur}

\end{document}
