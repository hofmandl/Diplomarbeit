
\chapter{�bertragung des MaxMin-Ansatzes in das HJB-Setting}
Wir erweitern unser standardfinazmarktmodell aus \ref{Standardfinanzmarktmodell} wie folgt.
Sei $(\Omega,F,P)$ ein Wahrscheinlichkeitsraum und die Parameter $r$, $\mu$, $\sigma$ wie gehabt. Auf dem
W-Raum sei wieder eine Brownsche Bewegung $W$ definiert und ein Sprungprozess $N$, der auf dem Intervall
$(0,T)$ h�chstens einen Sprung von 0 auf 1 machen kann.
�hnlich wie beim vorher betrachteten L�sungsanstz f�hren wir wieder einen Parameter $k$ ein, der die H�he
des Crashes angibt.
Die Preise sind dann gegeben durch:
\begin{align}
 dB(t) &= r B(t) dt \\
  B(0) &= 1         \\
 dS(t) &= S(t-)(\mu dt + \sigma dW(t) - k dN(t)) \\
  S(0) &= s_0
\end{align}
Ist nun eine Handelsstrategie $\pi$ gegeben, so folgt der Verm�gensprozess der folgenden Dynamik
\begin{align}
 dX(t) &= X(t)(r dt + \pi(t)(\mu -r) dt + \sigma dW(t) -k dN(t)) \\
  X(t_0) &= x_0
\end{align}
Ohne $N$ ist $X$ on  Ito-Diffusion. Den Generator $(t,X(t))$ bezeichen wir mit
\begin{equation} \label{eq:Generator}
 \mathcal{L}^{\pi} v(t,x) = v_t(t,x) + v_x(t,x)(r+\pi(t,x)(\mu-r)) x + \einhalb v_{xx}(t,x) (\pi(t,x))^2
\sigma^2 x^2
\end{equation}

Wir stellen uns also das folgende Optimierungsproblem:
\begin{equation}
\sup_{\pi \in \mathcal{A}} \inf_{N \in \mathcal{B}} E \left[ U(X(T)) \right]
\end{equation}



Wir f�hren nun eine Wertfunktion ein
\begin{equation}
 J^n(t,x,\pi,N) = E \left[ U(X(T)) | X(t) = x, N(t)=1-n \right] = E_{t,x,n} \left[ U(X(T)) \right]
\end{equation}

Die optimale Wertfunktion ist dann:
\begin{equation}
 V^n(t,x)= \sup_{\pi \in \mathcal{A}} \inf_{N \in \mathcal{B}} J^n(t,x,\pi,N)
\end{equation}

Sei
\begin{equation}
 G=\left[0,T \right] \times \mathbb{R}^+
\end{equation}


Wir formulieren nun einen Verifikationsatz. Sei $v^0$ eine L�sung von
\begin{align}
 0 &= \sup_{\pi \in R } \left[ \mathcal{L}^{\pi} v^0(t,x) \right] \\
 v^0(T,x) &= U(x)
\end{align}
und sei $p$ eine zul�ssige Handeslstrategie mit
\begin{equation}
 \mathcal{L}^{p(t,x)} v^0(t,x) = 0
\end{equation}

Dann gilt $V^0(t,x) = v^0(t,x)$ und $p$ ist die optimale Strategie.

Sei 
\begin{equation}
 \mathcal{A}'(t,x) = \{ 
                \pi : \pi \in R,  0 \leq \mathcal{L}^{\pi(t)} v^n(t,x) 
                \}
\end{equation}

\begin{equation}
 \mathcal{A}''(t,x) = \{ 
                  \pi : \pi \in R, v^1(t,x) \leq v^0(t,x(1-k\pi)) 
                 \}
\end{equation}
Es gelte nun
\begin{equation} \label{eq:HJB-Ungleichung}
  0 \leq \sup_{\pi \in \mathcal{A}''(t,x)} \left[ \mathcal{L}^{\pi} v^0(t,x) \right]
\end{equation}
\begin{equation} \label{crashrelation}
 0 \leq \sup_{\pi \in \mathcal{A}'(t,x)} \left[ v^0(t,x(1-k\pi))-v^1(t,x) \right] 
\end{equation}
\begin{equation} \label{complementarity}
 0 = \sup_{\pi \in \mathcal{A}''(t,x)} \left[ \mathcal{L}^{\pi} v^0(t,x) \right]  
     \sup_{\pi \in \mathcal{A}'(t,x)} \left[ v^0(t,x(1-k\pi))-v^1(t,x) \right]
\end{equation}
und
\begin{equation}
 v^1(T,x) = U(x)
\end{equation}

Existiert weiterhin eine zul�ssige Handelstrategie mit
\begin{equation}
 p(t,x) \in \mathcal{A}''(t,x)
\end{equation}
und
\begin{equation} \label{eq:optimale_Strategie}
  \mathcal{L}^{p(t,x)} v^1(t,x) = \sup_{\pi \in \mathcal{A}''(t,x)} \left[ \mathcal{L}^{\pi} v^0(t,x) \right]
\end{equation}


% \begin{equation} \label{eq: optimaler_Crash}
%  \theta = \inf_s \left[ v^0(s,x(1-k \pi)) \leq v^1(t,x) \right]
% \end{equation}

Dann gilt $v^1(t,x)=V^1(t,x)$ und $p$ ist die optimale Strategie.

Wir brauchen noch ein Lemma:
Die Wertfunktion kann auf die folgenden Weisen ausgedr�ckt werden:
\begin{align}
 V^1(t,x) &=  \sup_{\pi \in \mathcal{A}} \inf_{N \in \mathcal{B}}    E \left[ U(X(T)) \right] \\
          &=  \inf_{N \in \mathcal{B}}   \sup_{\pi \in \mathcal{A}}  E \left[ U(X(T)) \right] \\
          &=  \sup_{\pi \in \mathcal{A}} \inf_{\tau}    V^0(\tau,X(\tau-)(1-k \pi(\tau))) \\
          &=  \inf_{\tau}   \sup_{\pi \in \mathcal{A}}    V^0(\tau,X(\tau-)(1-k \pi(\tau))) \\
\end{align}

Beweis des Verifikationssatzes:
Wir gehen davon aus, das im Punkte $(t,x)$ $X(t)=x$ gilt und noch ein Crash m�glich ist.
Anwendung der Ito-formel ergibt dann:
\begin{equation}
 dv^1(s,X(s))=\mathcal{L}^{\pi(s)}v(0,X(s))ds + v_x^1(s,X(s)) \sigma X(s) dW(s)
\end{equation}
und
\begin{equation}
 dv^1(\tau,X(\tau))= v^1(\tau,X(\tau))(1-\pi(\tau)k) - v^1(\tau-,X(\tau-))
\end{equation}
also
\begin{equation} \label{itov}
v^1(\tau-,X(\tau-)) - v^1(t,x) = \int_t^{\tau} \mathcal{L}^{\pi(s)}v(s,X(s))ds
                            +   \int_t^{\tau} v_x^1(s,X(s)) \sigma X(s) p(s) dW(s) 
\end{equation}
Wir fixieren nun die optimale Strategie $p$. Dann folgt aus der der Vorrausetzung \ref{eq:optimale_Strategie}
an $p$ und der HJB-ungleichung \ref{eq:HJB-Ungleichung} f�r $t \leq s \leq \tau$:
\begin{equation} \label{rauch}
 0 \leq \sup_{\pi \in \mathcal{A}''(s,X(s))} \left[ \mathcal{L}^{\pi} v^1(s,X(s)) \right] = 
  \left[ \mathcal{L}^{p(s,X(s)} v^1(s,X(s)) \right]
\end{equation}

und da $p(s,X(s)) \in \mathcal{A}''(s,X(s))$
\begin{equation} \label{schall}
  v^1(s,X(s)) \leq v^0(s,X(s)(1-k\pi(s))
\end{equation}

Wenn wir nun $p$ in \ref{itov} einsetzen und die Gleichung nach $v^1(t,x)$ umstellen, erhalten wir:
\begin{equation}
 v^1(t,x) = v^1(\tau-,X(\tau-)) - \int_t^{\tau} \mathcal{L}^{p(s,X(s))}v(s,X(s))ds
                                  -   \int_t^{\tau} v_x^1(s,X(s)) \sigma X(s) p(s) dW(s) 
\end{equation}
Mit \ref{schall} k�nnen wir den ersten Summanden auf der rechten Seite nach oben absch�tzen, den zweiten
Summanden k�nnen wir wegen \ref{rauch} durch Weglassen nach oben absch�tzen und erhalten:
\begin{equation}
 v^1(t,x) \leq v^0(\tau,X(1-k\pi(\tau)) -  \int_t^{\tau} v_x^1(s,X(s)) \sigma X(s) dW(s) 
\end{equation}
Bilden des Erwartungswertes f�hrt dann zu:
\begin{equation} \label{ungleichung}
 v^1(t,x) \leq E_{t,x} \left[ v^0(\tau,X(1-k\pi(\tau))  \right] 
\end{equation}
Ausgehend von \ref{ungleichung} vergr��ern wir die rechte Seite durch Bilden des Surprememums �ber alle
Handelsstrategien:
\begin{equation} 
 v^1(t,x) \leq \sup_{\pi} E_{t,x} \left[ v^0(\tau,X(1-k\pi(\tau))  \right] 
\end{equation}
Da da das $\tau$ beliebig gew�hlt war erha�lt nun auch das bilden des infimums auf er rechten Seite die
Ungleichung:
\begin{equation} 
 v^1(t,x) \leq \inf_{\tau} \sup_{\pi} E_{t,x} \left[ v^0(\tau,X(1-k\pi(\tau))  \right] 
\end{equation}
Wieder ausgehend von \ref{ungleichung} k�nnen wir die beiden letzten Schritte auch in umgekehrter Reihenfolge
durchf�hren. Da $\tau$ beliebig gew�hlt war erh�lt das Bilden des Infimums auf der rechten Seite von
\ref{ungleichung} die Ungleichung und wir erhalten
\begin{equation} 
 v^1(t,x) \leq \inf_{\tau} E_{t,x} \left[ v^0(\tau,X(1-k\pi(\tau))  \right] 
\end{equation}
und bilden des Surprememus �ber alle Handelstrategien macht die rechte Seite davon nur gr��er

\begin{equation} 
 v^1(t,x) \leq \sup_{\pi} \inf_{\tau} E_{t,x} \left[ v^0(\tau,X(1-k\pi(\tau))  \right] 
\end{equation}

Sein nun $\pi$ wieder eine beliebige Handelsstrategie. Dann fixieren wir dazu die Stoppzeit
\begin{equation}
 \theta(\pi) = \inf \{s \in R; v^0(s,X(s)(1-\pi(s))) \leq v^1(s,X(s)) \}
\end{equation}
wobei $X$ der Dynamik ohne $N$ folgt.


Wir fixieren nun $\theta$. Nach definition von $\theta$ gilt f�r $t\leq s < \theta$
\begin{equation} \label{super}
 v^0(s,X(s)(1-k \pi(s))) > v^1(s,X(s))
\end{equation}
und
\begin{equation} \label{duper}
 v^0(\theta,X(\theta)(1-k \pi(\theta))) \leq v^1(\theta,X(\theta))
\end{equation}
Nun muss entweder
\begin{equation} \label{hanni}
 \mathcal{L}^{\pi(s,X(s))} v_1(s,X(s)) < 0 
\end{equation}
oder 
\begin{equation} \label{nanni}
 0 \leq \mathcal{L}^{\pi(s,X(s))} v_1(s,X(s))
\end{equation}
gelten.
Gilt letzteres, so ist $\pi(s,X(s)) \in \mathcal{A}'(s,X(s)) $ und aus \ref{super} folgt

\begin{equation} 
 \sup_{\pi \in \mathcal{A}''(s,X(s))} \left[ v^0(s,X(s)(1-k\pi))-v^1(s,X(s)) \right] > 0 
\end{equation}

Wegen \ref{complementarity} folgt dann 
\begin{equation}
 0 = \sup_{\pi \in \mathcal{A}''(s,X(s))} \left[ \mathcal{L}^{\pi} v^1(s,X(s)) \right] 
\end{equation}

Aber wegen \ref{super} ist $\pi(s,X(s)) \in \mathcal{A}''(s,X(s))$ und es folgt

\begin{equation}
 \mathcal{L}^{\pi(s,X(s))} v_1(s,X(s)) \leq 0 
\end{equation}

Also unabh�ngig ob \ref{hanni} oder \ref{nanni} gilt, es folgt
\begin{equation} \label{shappi}
 \mathcal{L}^{\pi(s,X(s))} v_1(s,X(s)) \leq 0 
\end{equation}

Wenn wir nun $\theta$ in \ref{itov} einsetzen und die Fleichung nach $v(t,x)$ umstellen erhalten wir 
\begin{equation}
 v^1(t,x) = v^1(\theta,X(\theta)) - \int_t^{\theta} \mathcal{L}^{\pi(s)}v(s,X(s))ds
                                  -   \int_t^{\theta} v_x^1(s,X(s)) \sigma X(s) dW(s) 
\end{equation}
Mit \ref{duper} k�nnen wir den ersten Summanden auf der rechten Seite verkleinern, und wegen \ref{shappi}
wird die rechte Seite durch Weglassen des zweiten Summanden ebenfalls nur kleiner. So erhalten wir
\begin{equation}
 v^1(t,x) \geq v^0(\tau-,X(\tau-))  -   \int_t^{\tau} v_x^1(s,X(s)) \sigma X(s) dW(s) 
\end{equation}
Bildung des erwartungwertes f�hrt zu
\begin{equation} \label{ungleichung2}
 v^1(t,x) \geq E_{t,x} \left[ v^0(\tau-,X(\tau-)) \right]
\end{equation}
Wie im ersten Teil des Beweises nehmen wir nun surprema und infina:
Bilden des Infimums �ber alle Crashzeiten macht die rechte Seite von \ref{ungleichung2} nur kleiner.
 \begin{equation} 
  v^1(t,x) \geq \inf_{\tau} E_{t,x} \left[ v^0(\tau-,X(\tau-)) \right]
 \end{equation}
 Da $\pi$ beliebig gew�hlt war, bleibt die Ungleichung durch Bildung des Surpremums �ber alle Handelstrategie
erhalten:
\begin{equation} 
 v^1(t,x) \geq \sup_{\pi} \inf_{\tau} E_{t,x} \left[ v^0(\tau-,X(\tau-)) \right]
\end{equation}
Wieder ausgehen von \ref{ungleichung2} k�nnen wir mit den gleichen Argument wie  in der verherigen Gleichung
das Surpremum �ber alle handelsstrategien bilden.
\begin{equation} 
 v^1(t,x) \geq \sup_{\pi}  E_{t,x} \left[ v^0(\tau-,X(\tau-)) \right]
\end{equation}
Und das Bilden des Infimums macht die rechte Seite nur kleiner.
\begin{equation} 
 v^1(t,x) \geq \inf_{\tau} \sup_{\pi}  E_{t,x} \left[ v^0(\tau-,X(\tau-)) \right]
\end{equation}

\section{Charakteriesierung der L�sung}
In $\mathcal{N}$ gilt
\begin{equation} \label{eq:N1}
 \pi(t,x) = \frac{-V^1_x(t,x)}{V^1_xx(t,x) x} \pistar
\end{equation}
und
\begin{equation} \label{eq:N2}
 V^1_t(t,x) = -V^1_x(t,x)(r+\pi(\mu-r))x - \einhalb V^1_{xx}(t,x) \pi^2 \sigma^2 x^2
\end{equation}
und auf dem Komplement von N gilt
\begin{equation} \label{eq:NC1}
V^1(t,x) = V^0(t,x(1-k\pi))
\end{equation}
und
\begin{equation} \label{eq:NC2}
 V^1_t(t,x) = -V^1_x(t,x)(r+\pi(\mu-r))x - \einhalb V^1_{xx}(t,x) \pi^2 \sigma^2 x^2
\end{equation}



\section{Log Utility}
Inspiriert duch die L�sung
\begin{equation}
 V^0(t,x) = \log(x) + \left(r+\einhalb \pistern \right)(T-t)
\end{equation}
im crasfreien Scenario machen wir den Ansatz  
\begin{equation}
 V^1(t,x) = \log(x) + f^1(t)
\end{equation}
Wir berechen die ben�tigten Ableitungen:
\begin{equation}
 V^1_t(t,x)= f^1_t(t),
\end{equation}
\begin{equation}
 V^1_x(t,x)= \frac{1}{x}
\end{equation}
und
\begin{equation}
 V^1_{xx}(t,x)= -\frac{1}{x^2}
\end{equation}

Aus $V^1(t,x) = V^0(t,x(1-k\pi^1))$ folgt
\begin{equation}
\log(x) + f^1(t) = \log(x) + \log(1-k\pi^1) + f^0(t)
\end{equation}
Umstellen f�hrt zu 
\begin{equation} \label{zwischenergebnis}
\exp(f^1(t) - f^0(t)) = 1-k\pi^1 
\end{equation}

und schliesslich
\begin{equation}
\pi^1 = \frac{1-\exp(f^1(t)-f^0(t))}{k}
\end{equation}

Setzen wir die Ableitungen von $V$ in \ref{eq:NC2} ein erhalten wird
\begin{equation} \label{eq:f1}
f^1_t(t) =  -(r+\pi^1(\mu-r)) + \einhalb (\pi^1)^2 \sigma^2
\end{equation}

Ableiten von $\pi^1$ ergibt
\begin{equation}
\pi_t^1=  \frac{\exp(f^1(t)-f^0(t)) (f_t^0(t)-f_t^1(t))}{k}
\end{equation}

und druch Einsetzen von \ref{zwischenergebnis}

\begin{equation} \label{eq:zack}
 \pi^1_t=\frac{1}{k} (1- \pi^1 k) 
  \left(
   (f_t^0(t)-f_t^1(t)) 
  \right)
\end{equation}

Mit \ref{eq:f1} und dem bekannten wert f�r $f^0$ berechnen wir
\begin{align}
   (f_t^0(t)-f_t^1(t)) &= -r - \einhalb \pistern - \left[ -r + \pi^1 (\mu-r) + \einhalb (\pi^1)^2 \sigma^2
 \right] \\
  &= \pi^1 (\mu-r) - \einhalb \left[ \pistern + (\pi^1)^2 \sigma^2 \right]
\end{align}
und setzen wir das wieder in \ref{eq:zack} ein so ergibst sich die schon aus \ref{eq:dgl} bekannte
Differentialgleichung.
\begin{equation}
 \pi^1_t=\frac{1}{k} (1- \pi^1 k) 
  \left(
   \pi^1 (\mu-r) - \einhalb \left[ \pistern + (\pi^1)^2 \sigma^2 \right] 
  \right)
\end{equation}





