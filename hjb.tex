
\chapter{�bertragung des MaxMin-Ansatzes in das HJB-Setting}
Wir erweitern unser standardfinazmarktmodell aus \ref{Standardfinanzmarktmodell} wie folgt.
Sei $(\Omega,F,P)$ ein Wahrscheinlichkeitsraum und die Parameter $r$, $\mu$, $\sigma$ wie gehabt. Auf dem
W-Raum sei wieder eine Brownsche Bewegung $W$ definiert und ein Sprungprozess $N$, der auf dem Intervall
$(0,T)$ h�chstens einen Sprung von 0 auf 1 machen kann.
�hnlich wie beim vorher betrachteten L�sungsanstz f�hren wir wieder einen Parameter $k$ ein, der die H�he
des Crashes angibt.
Die Preise sind dann gegeben durch:
\begin{align}
 dB(t) &= r B(t) dt \\
  B(0) &= 1         \\
 dS(t) &= S(t-)(\mu dt + \sigma dW(t) - k dN(t)) \\
  S(0) &= s_0
\end{align}

Ist nun eine Handelsstrategie $\pi$ gegeben, so folgt der Verm�gensprozess der folgenden Dynamik
\begin{align}
 dX(t) &= X(t)(r dt + \pi(t)(\mu -r) dt + \sigma dW(t) -k dN(t)) \\
  X(t_0) &= x_0
\end{align}

Wir stellen uns also das folgende Optimierungsproblem:
\begin{equation}
\sup_{\pi \in \mathcal{A}} \inf_{N \in \mathcal{B}} E \left[ U(X(T)) \right]
\end{equation}

Wir f�hren nun den folgenden Operator ein:
\begin{equation}
 \mathcal{L}^{\pi} v(t,x) = v_t(t,x) + v_x(t,x)(r+\pi(\mu-r)) x + \einhalb v_{xx}(t,x) \pi^2 \sigma^2 x^2
\end{equation}

Wir f�hren nun eine Wertfunktion ein
\begin{equation}
 J^n(t,x,\pi,N) = E \left[ U(X(T)) | X(t) = x, N(t)=1-n \right]
\end{equation}

Die optimale Wertfunktion ist dann:
\begin{equation}
 V^n(t,x)= \sup_{\pi \in \mathcal{A}} \inf_{N \in \mathcal{B}} J^n(t,x,\pi,N)
\end{equation}

Verification Thereom
Sei $v^0$ eine L�sung von
\begin{align}
 0 &= \sup_{\pi \in \mathcal{A}} \left[ \mathcal{L}^{\pi} v^0(t,x) \right] \\
 v^0(T,x) &= U(x)
\end{align}
und
\begin{equation}
 p(t) = arg \sup_{\pi \in \mathcal{A}} \left[ \mathcal{L}^{\pi} v^0(t,x) \right] 
\end{equation}

Dann gilt $V^0(t,x) = v^0(t,x)$ und $p$ ist die optimale Strategie.

Sei 
\begin{equation}
 \mathcal{A}' = \{ 
                \pi : \pi \in \mathcal{A}, \forall x : 0 \leq \mathcal{L}^{\pi(t)} v^n(t,x) 
                \}
\end{equation}

\begin{equation}
 \mathcal{A}'' = \{ 
                  \pi : \pi \in \mathcal{A}, \forall x : v^1(t,x) \leq v^0(t,x(1-k\pi(t))) 
                 \}
\end{equation}
Es gelte nun
\begin{equation} \label{hjbinequality}
  0 \leq \sup_{\pi \in \mathcal{A}''} \left[ \mathcal{L}^{\pi} v^0(t,x) \right]
\end{equation}
\begin{equation} \label{crashrelation}
 0 \leq \sup_{\pi \in \mathcal{A}''} \left[ v^0(t,x(1-k\pi))-v^1(t,x) \right] 
\end{equation}
\begin{equation} \label{complementarity}
 0 = \sup_{\pi \in \mathcal{A}''} \left[ \mathcal{L}^{\pi} v^0(t,x) \right]  
     \sup_{\pi \in \mathcal{A}''} \left[ v^0(t,x(1-k\pi))-v^1(t,x) \right]
\end{equation}


und
\begin{equation}
 v^1(T,x) = U(x)
\end{equation}
und
\begin{equation} {eq:optimale_Strategie}
 p(t) = arg \sup_{\pi \in \mathcal{A''}} \left[ \mathcal{L}^{\pi} v^1(t,x) \right]
\end{equation}
und
\begin{equation} {eq: optimaler_Crash}
 \theta = \inf_s \left[ v^0(s,x(1-k \pi)) \leq v^1(t,x) \right]
\end{equation}
Dann ist alles optimal.


Lemma:
Die Wertfunktion kann auf die folgenden Weisen ausgedr�ckt werden:
\begin{align}
 V^1(t,x) &=  \sup_{\pi \in \mathcal{A}} \inf_{N \in \mathcal{B}}    E \left[ U(X(T)) \right] \\
          &=  \inf_{N \in \mathcal{B}}   \sup_{\pi \in \mathcal{A}}  E \left[ U(X(T)) \right] \\
          &=  \sup_{\pi \in \mathcal{A}} \inf_{\tau}    V^0(\tau,X(\tau-)(1-k \pi(\tau))) \\
          &=  \inf_{\tau}   \sup_{\pi \in \mathcal{A}}    V^0(\tau,X(\tau-)(1-k \pi(\tau))) \\
\end{align}

Beweis des Verifikationssatzes:
Wir gehen davon aus, das im Punkte $(t,x)$ $X(t)=x$ gilt und noch ein Crash m�glich ist.
Anwendung der Ito-formel ergibt dann:
\begin{equation}
 dv^1(s,X(s))=\mathcal{L}^{\pi(s)}v(0,X(s))ds + v_x^1(s,X(s)) \sigma X(s) dW(s)
\end{equation}
und
\begin{equation}
 dv^1(\tau,X(\tau))= v^1(\tau,X(\tau))(1-\pi(\tau)k) - v^1(\tau-,X(\tau-))
\end{equation}
also
\begin{equation} \label{itov}
v^1(\tau-,X(\tau-)) - v^1(t,x) = \int_t^{\tau} \mathcal{L}^{\pi(s)}v(s,X(s))ds
                            +   \int_t^{\tau} v_x^1(s,X(s)) \sigma X(s) dW(s) 
\end{equation}
Wir fixieren nun die optimale Strategie $p$. Dann folgt aus der Definition f�r $t \leq s \leq \tau$  hjb
inequality (\ref{hjbinequality}) 
\begin{equation} \label{rauch}
 0 \leq \sup_{\pi \in \mathcal{A}''} \left[ \mathcal{L}^{p} v^0(t,x) \right]
\end{equation}

und da $p \in \mathcal(A)''$
\begin{equation} \label{schall}
  v^1(s,X(s)) \leq v^0(s,X(1-k\pi(t))
\end{equation}

Wenn wir nun $p$ in \ref{itov} einsetzen und die Gleichung nach $v^1(t,x)$ umstellen, erhalten wir:
\begin{equation}
 v^1(t,x) = v^1(\tau-,X(\tau-)) - \int_t^{\tau} \mathcal{L}^{\pi(s)}v(s,X(s))ds
                                  -   \int_t^{\tau} v_x^1(s,X(s)) \sigma X(s) dW(s) 
\end{equation}
Mit \ref{schall} k�nnen wir den ersten Summanden auf der rechten Seite nach oben absch�tzen, den zweiten
Summanden k�nnen wir wegen \ref{rauch} durch Weglassen nach oben absch�tzen und erhalten:
\begin{equation}
 v^1(t,x) \leq v^0(s,X(1-k\pi(t)) -  \int_t^{\tau} v_x^1(s,X(s)) \sigma X(s) dW(s) 
\end{equation}
Bilden des Erwartungswertes f�hrt dann zu:
\begin{equation} \label{ungleichung}
 v^1(t,x) \leq E_{t,x} \left[ v^0(s,X(1-k\pi(t))  \right] 
\end{equation}
Ausgehend von \ref{ungleichung} vergr��ern wir die rechte Seite durch Bilden des Surprememums �ber alle
Handelsstrategien:
\begin{equation} 
 v^1(t,x) \leq \sup_{\pi} E_{t,x} \left[ v^0(s,X(1-k\pi(t))  \right] 
\end{equation}
Da da das $\tau$ beliebig gew�hlt war erha�lt nun auch das bilden des infimums auf er rechten Seite die
Ungleichung:
\begin{equation} 
 v^1(t,x) \leq \inf_{\tau} \sup_{\pi} E_{t,x} \left[ v^0(s,X(1-k\pi(t))  \right] 
\end{equation}
Wieder ausgehend von \ref{ungleichung} k�nnen wir die beiden letzten Schritte auch in umgekehretr Reihenfolge
durchf�hren. Da $\tau$ beliebig gew�hlt war erh�lt das Bilden des Infimums auf der rechten Seite von
\ref{ungleichung} die Ungleichung und wir erhalten
\begin{equation} 
 v^1(t,x) \leq \inf_{\tau} E_{t,x} \left[ v^0(s,X(1-k\pi(t))  \right] 
\end{equation}
und bilden des Surprememus �ber alle Handelstrategien macht die rechte Seite davon nur gr��er

\begin{equation} 
 v^1(t,x) \leq \sup_{\pi} \inf_{\tau} E_{t,x} \left[ v^0(s,X(1-k\pi(t))  \right] 
\end{equation}

Wir fixieren nun $\theta$. Nach definition von $\theta$ gilt f�r $t\leq s < \theta$
\begin{equation} \label{super}
 v^0(s,X(s)(1-k \pi(s))) > v^1(s,X(s))
\end{equation}
und
\begin{equation}
 v^0(\theta,X(\theta)(1-k \pi(\theta))) \leq v^1(\theta,X(\theta))
\end{equation}
Nun muss entweder
\begin{equation} \label{hanni}
 \mathcal{L}^{\pi} v_1(s,X(s)) < 0 
\end{equation}
oder 
\begin{equation} \label{nanni}
 0 \leq \mathcal{L}^{\pi} v_1(s,X(s))
\end{equation}
gelten.
Gilt letzteres, so ist $\pi \in \mathcal{A}_1' $ und aus folgt

\begin{equation} 
 \sup_{\pi \in \mathcal{A}''} \left[ v^0(t,x(1-k\pi))-v^1(t,x) \right] > 0 
\end{equation}

Wegen \ref{complementarity} folgt dann 
\begin{equation}
 0 = \sup_{\pi \in \mathcal{A}''} \left[ \mathcal{L}^{\pi} v^0(t,x) \right] 
\end{equation}

Aber wegen \ref{super} ist $\pi \in \mathcal{A}''$ und es folgt

\begin{equation}
 \mathcal{L}^{\pi} v_1(s,X(s)) \leq 0 
\end{equation}

Also unabh�ngig ob \ref{hanni} oder \ref{nanni} gilt, es folgt
\begin{equation}
 \mathcal{L}^{\pi} v_1(s,X(s)) \leq 0 
\end{equation}
