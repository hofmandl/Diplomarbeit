\chapter{Zahlenbeispiele und Simulation}
\label{chap:Zahlenbeispiele}
\section{Worst-Case-Schranke der optimalen Strategie}
In diesem Kapitel wollen wir konkret ein paar Zahlenbeispiele rechnen. Dazu spezifizieren wir unsere Marktkoeffizienten.
\begin{align*}
 \mu &=0.2 \\
   r&=0.05 \\
 \sigma &= 0.4
\end{align*}
Au�erdem spezifizeiren wir unsere pers�nlichen Parameter zu
\begin{align*}
 x_0 &=1 \\
  T&=10 
\end{align*}
Der erwartete Nutzen durch Verfolgen der reinen Bondstrategie ist dann
\begin{equation*}
 rT=0.5
\end{equation*}
Die optimale Strategie im crasfreien Setting f�hrt zu einem erwarteten Endverm�gen von
\begin{equation*}
 V^0(0,1,\log)=1.203
\end{equation*}
Als optimalen erwarteten Nutzen im Markt mit Crash erwarten wir nach den Ergebnissen dieser Arbeit dann einen Wert dazwischen. Dazu ben�tigen wir die Werte der L�sung $\pidach$ von
der Differentialgleichung aus Satz \ref{th:DGL}.  Wir haben diese numerisch mit Matlab gel�st. Der Graph dieser L�sung ist in Abbildung \ref{fig:optimaleStrategie} abgebildet.

\begin{figure}[htbp]
\centering
\includegraphics[type=eps,ext=.eps,read=.eps,width=1\linewidth]{Bilder/foo}
\label{fig:optimaleStrategie}
\caption{Optimale Handelstrategie $\pidach$}%
\end{figure}


Wir haben nun verschieden M�glichkeiten den erwarteten Nutzen bei Verfolgen dieser Strategie. Im Beweis von Satz \ref{th:hauptsatz_willmott} haben wir gelernt, da� es reicht wenn
wir den Wert von der numerisch ermittelten L�sung von $\pidach$ an der Stelle $0$ nehmen und wie folgt in die Wertfunktion einsetzen
\begin{equation*}
 V^0(0,1-\pidach(0)k,\log)=1.0472
\end{equation*}
Genauso ist es aber auch m�glich den erwarten Nutzen im crashfreien Scenario zu berechen und mit Satz \ref{th:ExpLogUtility} 
\begin{equation*}
 \int_0^T f(\pidach(t)) dt + rT = 1.0471
\end{equation*}
zu berechnen.
Eine weitere M�glichkeit w�re die Verwendung der Wertfunktion aus $V^1(\cdot,\cdot,\log)$ aus Abschnitt \ref{sec:log_utility_hjb}.

\section{Worst-Case-Schrankenberechnung durch Monte-Carlo-Simulation}

Bei diesen Berechnungen haben wir die Ergebnisse aus den S�tzen f�r den logaritmischen Nutzen benutzt. Im folgenden berechnen wir Worst-Case-Schranke ohne diesen theoretischen
Hinterbau, die Berechnungen beschr�nken sich dabei nicht auf den logaritmischen Nutzen.

Wir stellen zun�chst unsere in Matlab programmierte Funktion worstcase vor.
Diese Funktion berechnet die WorstcaseSchranke nach dem Vorbild von Definition \ref{def:WCPortfolioproblem1}. Dort wurde die Worst-Case-Schranke wie folgt berechnet:
\begin{equation*}
  \min \left( 
                   \inf_{0 \leq s \leq T} E \left[ V^0(s,X_{\pi}(s)(1-\pi(s)k),U) \right] 
                  , 
                   E \left[ U(X_{\pi}(T)) \right] 
                  \right)
\end{equation*}
Mit einem Computer k�nnen wir nat�rlich nicht das Infimum �ber unendlich viele Werte nehmen. Deswegen �bergeben wir der Funktion eine Liste von Werten, f�r die dann die jeweilige
Erwartung f�r ein Crash zu diesem Zeitpunkt berechnet wird. Die folgende Tabelle beschreibt die Schnittstelle zu dieser Funktion.

\begin{table}[htbp]
% \centering
\begin{tabular}{|l|l|}
\hline
\multicolumn{2}{|l|}{Funktion: worstcase} \\
\hline
\multicolumn{2}{|l|}{Eingaben} \\
\hline
 v &  Wertfunktion im crashfreien Scenario \\
 p &  Handelstrategie  \\
 t &  Liste von Crashzeitpunkten  \\
\hline
\multicolumn{2}{|l|}{Ausgaben} \\
\hline
crashUtilities & Erwarteter Nutzen zu dem jeweiligen Crashzeitpunkten \\
noCrashUtility & Erwarteter Nutzen ohne Crash\\
worstcasebound & Worst-Case-Schranke\\
\hline
\end{tabular}
\caption{Beschreibung der Funktion worstcase}
\end{table}

Wir illustrieren die Benutzung dieser Funktion, in dem wir die Beispiele besprechen, die wir auch schon im theoretischen Teil behandelt haben. Dazu haben wir eine feste Liste von
Crashzeitpunkten, die Wertfunktion f�r den logaritmischen Nutzen und jeweils verschiedene Handelsstrategien als Eingaben an die Funktion �bergeben und haben in Tabelle
\ref{tab:Ausgaben_von_worstcase} die Ergebnisse zusammengefasst.
Als Erstes betrachten wir die Ausgaben f�r die reine Bondstrategie $\pi_0$ deren Worst-Case-Schranke wir in Satz \ref{th:worst-case-schranke-reine-bondstrategie} berechnet haben.
Dort hatten wir schon erw�hnt dass das Ausbleiben eines Crashes das Worst-Case-Scenario ist. Dass wird auch an der Programmausgabe deutlich. Je sp�ter der Markt crasht, umso
weniger Zeit hat der Investor die optimale Strategie im crashfreien Markt zu verfolgen.
Als n�chstes betrachten wir die optimale Strategie $\pistar$ im crashfreien Scenario. Hier hatten wir in Satz \ref{th:worst-case-schranke-optimal-ohne-crash} behauptet, dass ein
Crash mehr schadet als kein Crash, aber der Zeitpunkt egal ist. Auch dieses Verhalten l�sst sich aus der Ausgabe herauslesen.
Schliesslich betrachten wir noch die Ausgabe f�r die Handelsstrategie $\pidach$. Hier erwarten wir, dass wir die Gleichgewichtsbedingung wiederfinden. Wir sollten f�r jeden
Crashzeitpunkt als auch f�r den chrashfreien Fall den gleichen erwarteten Nutzen haben. Das spiegelt sich auch in der Ausgabe wieder.

\begin{table}[htbp]
% \centering
\begin{tabular}{|l|l|}
\hline
\multicolumn{2}{|p{11cm}|}{Ausgaben der Funktion worstcast mit den Eingaben $v=V^0(\cdot,\cdot,\log)$, $t=[0,2,5,8,10]$ und $p$ wie angegeben } \\
\hline
\hline
\multicolumn{2}{|l|}{Eingabe:  $p=p_0$  } \\
\hline
\multicolumn{2}{|l|}{Ausgaben} \\
\hline
crashUtilities & $[1.20312 ,1.06248,0.85150,0.64053,0.49988]$ \\
noCrashUtility & $0.49938$ \\
worstcasebound & $0.49938$ \\
\hline
\hline
\multicolumn{2}{|l|}{Eingaben: $\pistar$} \\
\hline
\multicolumn{2}{|l|}{Ausgaben} \\
\hline
crashUtilities & $[0.99549,0.99599,0.99116,1.00101,0.99856]$ \\
noCrashUtility & $1.2125$ \\
worstcasebound & $0.99116$ \\
\hline
\hline
\multicolumn{2}{|l|}{Eingaben: $\pidach$} \\
\hline
\multicolumn{2}{|l|}{Ausgaben} \\
\hline
crashUtilities & $[1.0472,1.0436,1.0486,1.0559,1.0476]$ \\
noCrashUtility & $1.0351$ \\
worstcasebound & $1.0351$ \\
\hline
\end{tabular}
\caption{Berechnung von Worst-Case-Schranken f�r verschiedene Handelsstrategien}
\label{tab:Ausgaben_von_worstcase}
\end{table}

\section{Beste lineare Strategie}

Mit unsere Funktion k�nnen wir nun also jede Handelstrategie beurteilen. Damit k�nnen wir nun zum Beispiel einen einfachen Algorithmus angeben um eine beste lineae Strategie zu
berechnen. Mit linearer Strategie meinen wir eine lineare Funktion, die am Ende des Zeithorizonts $0$ ist. Damit ist solch eine Strategie durch einen Punkt auf der y-Achse
festgelegt. Das passt auch ganz gut zu unserer Vorstellung, dass man je n�her man sich dem Ende des Zeithorizonts n�hert, desto vorsichtiger sollte man in Aktien investieren. Eine
linear Funktion ist f�r solch ein Verhalten das einfachste Modell. Wenn wir nun zu einer Menge von Punkten auf der y-Achse jeweils die Worstcase-Schranke berechnen, und dann den
mit der h�chsten Worstcase-Schranke w�hlen, m�ssten wir dieses beste Strategie gefunden haben.

In der Tabelle \ref{tab:lineare_Strategien} haben wir die Ergebnisse dieser Rechnung dargestellt. Man erkennt dass die beste lineare Strategie, die mit dem Startwert $1.0$ ist. Die
dazugeh�rige Worst-Case-Schranke ist mit $0.97409$ nur geringf�gig kleiner als die
Worst-Case-Schranke der optimalen Strategie und deutlich gr��er als der erwartete nutzen der reinen Bondstrategie. In der Abbildung \ref{fig:optimale_lineare_Strategie} haben wir
die optimale lineare Strategie und die optimale Strategie nochmal im Vergleich dargestellt.

\begin{table}[!htbp]
% \centering
\begin{tabular}{|lllllll|}
\hline
\multicolumn{7}{|l|}{Startpunkte} \\
\hline
 0.7 &  0.8 & 0.9 & 1.0 & 1.1 & 1.2 & 1.3 \\
\hline
\multicolumn{7}{|l|}{Worstcaseschranken} \\
\hline
0.88871 &  0.91934 &  0.96035 &  0.97409 &  0.95466 &  0.92869 &  0.90202 \\
\hline
\end{tabular}
\caption{Worst-Case-Schranke f�r lineae Strategie mit unterschiedlichen Startwerten}
\label{tab:lineare_Strategien}
\end{table}




\begin{figure}[htbp]
\centering
\includegraphics[type=eps,ext=.eps,read=.eps]{Bilder/bar}
\caption{Optimale und optimale lineare Strategie im Vergleich}%
\label{fig:optimale_lineare_Strategie}
\end{figure}

