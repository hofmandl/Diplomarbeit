\chapter{Einleitung}

DDDD Sei $ (\Omega,F,P) $ ein Wahrscheinlichkeitsraum. Sei $W$ eine auf diesen Raum definierte Brownsche
Bewegung. Wir betrachten das Black-Scholes-Marktmodell mit einer Aktie und konstanten Koeffizienten, die
Preisprozesse seien also durch die folgenden Gleichungen gegeben:
\begin{align*}
 dB(t) &= r B(t) dt \\
  B(0) &= 1         \\
 dS(t) &= S(t)(\mu dt + \sigma dW(t)) \\
  S(0) &= s_0
\end{align*}

Sei $X$ der Verm�gensprozess zur Handelsstrategie $\pi$. Er ist durch folgende Differentialgleichung gegeben:
\begin{align*}
 dX(t) &= X(t)(r dt + \pi(t)(\mu -r) dt + \sigma dW(t)) \\
  X(0) &= x_0
\end{align*}
Sei $U$ eine Nutzenfunktion. Dann nennen wir
\begin{equation*}
  \underset{\pi}{\text{max}} E(U(X^{\pi}(T)))
\end{equation*}
das \emph{klassische Portfoliooptimierungsproblem}. Wir diskutieren seine L�sung mittels stochastischer
Steuerung, welche auf Merton zur�chgeht.

Das Black-Scholes-Modell wurde vielfach daf�r kritisiert, dass seine Preisprozesse stetig sind. In Realit�t
beobachtet man aber Crashes von Aktienm�rkten. Um dies zu modellieren machen wir nun einen anderen Ansatz als
den vielfach diskutierten Vorschlag den Aktienprozess $S$ durch einen unstetigen Prozess zu modellieren: Wir
betrachten eine Stoppzeit $\tau$, zu der die Aktie um den Faktor $k$ f�llt. Das Endverm�gen hat in
diesem Modell dann die folgende Form:
\begin{equation*}
  X^{\pi \tau}(T)=(1-\pi(\tau)k)X(T)
\end{equation*}

Nun betrachten wir den Markt als Gegenspieler des Investors und versuchen das folgende Maximierungsproblem
\begin{equation*}
  \underset{\pi}{\text{max}} \underset{\tau}{\text{min}} E(U(X^{\pi\tau}(T)))
\end{equation*}
zu l�sen, das wir als \emph{Worst-Case Portfoliooptimierungsproblem} bezeichnen. Wir diskutieren
zwei von Ralf Korn vorgeschlagene L�sungen f�r dieses Problem. Die erste L�sung bestimmt die
Handelstrategie durch Gleichgewichts�berlegungen. Die zweite L�sung macht einen ganz anderen
Ansatz, durch Aufstellung eines Systems von HJB-Gleichungen kann eine Wertfunktion ermittelt
werden, aus der dann auf die optimale Handelstrategie geschlossen werden kann. Dabei werden wir
insbesondere untersuchen, inwiefern sich diese beiden Ans�tze in ihren Ergebnissen,in ihrer
Beweismethodik und ihren weiteren Anwendungsm�glichkeiten unterscheiden.

Um konkret rechen zu k�nnen, werden wir sowohl das \emph{klassische Portfoliooptimierungsproblem} als auch
die beiden Ans�tze beim \emph{Worst-Case Portfoliooptimierungsproblem} nach Wahl der logarithmischen
Nutzenfunktion diskutieren. 

Schliesslich soll noch untersucht werden, inwieweit wir das Problem als ein optimales Stoppproblem, welches
der Markt als Gegenspieler des Investors zu l�sen hat, formuliert werden kann.
